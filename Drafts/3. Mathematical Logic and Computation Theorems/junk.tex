
% Here I tried to explain a first order language but later seemed to me not relevant to the entirety of the draft.
\newpoint{A First-order Language:} $\curveL$ is defined as an infinite collection of symbols, separated into the following categories:
\begin{itemize}
    \item \textit{Parentheses:} $(,)$.
    \item \textit{Connectives:} $\land, \lor, \neg$.
    \item \textit{Quantifier:} $\forall, \exists$.
    \item \textit{Variables:} one for each positive integer $n$ denoted: $v_n$ for $n$th number.
    \item \textit{Equality:} $=$.
    \item \textit{Constant:} We can have a new symbol for each positive number or any other method that we distinguish between two numbert (we can use | for 1, || for 2 etc)
    \item \textit{Functions:} For each positive integer $n$, some set of zero or more $n$-ary function symbols.
    \item \textit{Relation:} For each positive integer $n$, some set of zero or more $n$-ary relation symbols.
\end{itemize}
\begin{callout}
    Having $n$ arity means that it is intended to represent a function of $n$ variables.
\end{callout}
Notice that by defining such language one can avoid the process of finiding an algorithm of grammar (the one that differenciates between nonesense and meaningful words) since we defined all the possible functions and etc. This way we have to only
\\
\\
\newpoint{Terms and Formulas:} As we mentioned earlier not all words of a set $\Sigma^\infty$ is meaningful. Since any combination of the alphabet is a word there has to be distinctions between what are meaningful words and what are not. We would consider two kinds of words as \textit{terms \& formulas} as follow:
\begin{define}
    \textit{If $\curveL$ is a language, a \textbf{term of $\curveL$} is a nonempty finite string $t$ of symbols from $\curveL$ such that either:}
    \begin{enumerate}
        \item $t$ is a variable, or
        \item $t$ is a constant symbol, or 
        \item $t:\equiv ft_1t_2t_3\dots t_n$, where $f$ is an $n$-ary function symbol of $\curveL$ and each of the $t_i$ is a term of $\curveL$.
    \end{enumerate}

\end{define}



% Plato and forms
\subsection{Plato and Forms}
\newpoint{Platonic Definitions and Forms:} According to Aristotle, as cited in Metaph. I.6.987a29–b14, XIII.4.1078b12–32 and 9.1086a24–b4, Socrates was primarily interested in defining ethical matters and Plato continued this interest. This is evident in many dialogues of Plato labeled as "Socratic" or "early," which reveal a preoccupation with defining various concepts. Despite some objections, such as those presented by Kahn (1996), I find no compelling reason to doubt Aristotle's account and believe that the Socratic dialogues offer us a glimpse into the nature of Socratic discourse, albeit in the form of historical fiction, particularly in regard to definitions. Even Xenophon's writings depict Socrates occasionally pursuing definitions, as shown in Mem. I.i.16 and IV.vi, although they do not provide the same level of detail as Plato's Socratic dialogues in reconstructing Socrates' method.
\\
\\
Additionally, Aristotle notes that Plato's adoption of Socrates' pursuit for definitions took a unique direction in which the objects of definition were distinguished as "forms," separate from perceptible things. This development occurred not in the Socratic dialogues but rather in works such as the Phaedo and Republic, which are typically classified as Plato's "middle" dialogues.
\\
\\
The Socratic dialogues that are under consideration here include Charmides, Euthyphro, Hippias Major, Laches, Lysis, the Protagoras, and Book I of the Republic. However, there is some controversy over the historical accuracy of certain works such as the Hippias Major, Lysis, and Republic I, which have been discussed in Dancy (2004: 7-9). Those who doubt the historical authenticity of these dialogues may view my interpretation as pertaining solely to Plato himself. Therefore, references to "Socrates" should only be understood as referring to the character portrayed in Plato's dialogues.
\\
\\
The same principle applies even more so to the use of the name "Socrates" in the following discussion of the middle dialogues, including Phaedo, Symposium, Republic, and Meno (which I consider to be transitional). In my opinion, which is shared by many others, these later dialogues feature much more of Plato's own views and less of Socrates' than the Socratic dialogues do. According to this view, the portrayal of the historical Socrates is more accurate in the Socratic dialogues, while the depiction of Socrates in the later works tends to reflect the philosophical development of Plato himself over time (for further details, see sections 2: INTERPRETING PLATO and 3: THE SOCRATIC PROBLEM).
\\
\\
It is not necessary to endorse any of the aforementioned viewpoints in order to follow this chapter. The developmental process I describe primarily pertains to logic: Socrates' arguments in the Socratic dialogues do not necessitate the "Theory of Forms," a commonly accepted metaphysical position, whereas his arguments in the later "middle" dialogues do rely on this theory. However, the latter arguments build upon the former ones, and one crucial argument that emerges in this process is what I refer to as the "Argument from Relativity" (AR). In the middle dialogues, AR establishes a contrast between a Form, such as Beauty, and its ordinary manifestations, arguing that while the latter are beautiful only in relation to other things, the Form itself is inherently beautiful (for more elaboration on this point, see section 12: THE FORMS AND THE SCIENCES IN SOCRATES AND PLATO).
\begin{align*}
    \text{(ARE) There is such a thing as the Beautiful.}&\\
    \text{(ARO) Any ordinary beautiful is also ugly.}&\\
    \text{(ARBeautiful) The Beautiful is never ugly.}&\\
    \therefore \text{(ARC) The Beautiful is not the same as any ordinary beautiful.}&
\end{align*}
In this context, the "Argument from Relativity" (AR) comprises four distinct components. First, there is the assumption of the existence of Beauty itself (ARE). Second, there is a premise (ARO) that needs to be supported by argument, which states that ordinary beautiful things are only relatively beautiful. Third, there is a statement about the Form of Beauty (ARBeautiful), which asserts that it is inherently beautiful and not merely beautiful in relation to other things. Finally, there is the conclusion (ARC) that draws upon these elements to establish a contrast between the Form of Beauty and its ordinary manifestations, with the former being an immutable and perfect embodiment of beauty while the latter are only beautiful in a relative sense.
\\
\\
The Argument from Relativity does not feature in the Socratic dialogues, though there is a notable foreshadowing of it in the Hippias Major (which will be discussed later). Rather, this argument emerges in the middle dialogues and represents the primary development that I am referring to, regardless of any debates regarding the precise chronology or authorship of these works.
\\
\\
The purpose of this chapter is to construct a Theory of Definition for Socrates, though it should be noted that this theory is not intended to represent either Socrates' or Plato's own theories of definition. The reason for this is that there is no explicit Theory of Definition presented in the Socratic dialogues, unlike later works such as Phaedrus, Sophist, Statesman, and Philebus, which introduce a more systematic approach known as the "Method of Collection and Division." Instead of attempting to discern Socrates' personal beliefs on what constitutes a definition, our approach involves examining his criticisms of specific attempts to define terms. For each such criticism, we analyze the specific ways in which the attempted definition fails, and then proceed to identify the characteristics that a successful definition would need to possess in order to avoid those shortcomings.
\\
\\
The Theory of Definition that we will construct in this chapter will include three primary conditions of adequacy: the Substitutivity Requirement, the Explanatory Requirement, and the Paradigm Requirement. The first condition is relatively straightforward, while the latter two are more complex and will have implications for the Theory of Forms. It should be noted, however, that Socrates' interest in definitions does not necessarily imply an engagement with metaphysical questions (though some scholars, such as Allen 1970, have argued to the contrary). The turn towards metaphysics only becomes apparent later on, particularly in works like the Phaedo.
\\
\\
The absence of a direct term for "definition" in Socrates' vocabulary is worth noting. Although one of the terms he uses refers to "boundary," his discussions mainly revolve around inquiries such as "what is the pious?" (Euthyphro), "temperance?" (Charmides), or "the beautiful?" (Hippias Major).
\\
\\
Before presenting our theory, it is important to understand why Socrates seeks definitions and answers to his "what is . . . ?" questions.
\\
\\
In Book I of the Republic, Socrates inquires about the nature of justice to determine if being just leads to greater happiness than being unjust. This practical concern is evident in other dialogues where definitions are sought. For example, the first portion of the Laches deals with whether combat training in heavy armor builds character, particularly courage. The question "what is courage?" arises in 190d to resolve this issue. In the Lysis, the dialogue shifts to defining a friend (212a8-b2) after discussing how friends ought to treat each other, which occupies half of the conversation. Similarly, in the Euthyphro, the query "what is piety?" emerges at 5c-d after Euthyphro claims his father committed murder based on what he believed was pious behavior. The Protagoras raises the inquiry of whether studying with a sophist like Protagoras promotes virtue or excellence and concludes with Socrates admitting that the discussion's difficulty arose from everyone failing to answer the question "what is excellence?" due to their preoccupation with other matters.
\\
\\
Socrates believes that definitions are crucial in determining the correct way to live, which is why he seeks them out. In these dialogues, the preliminary discussions leading up to the definition question often take up more space than the discussion of the definition itself.
\\
\\
Despite this, the question of definition remains highly significant, and Socrates provides a reason for its importance when it comes to determining how to live. He operates on the "Intellectualist Assumption," as he explicitly states, which some refer to as the "Socratic Fallacy" or the "Principle of the Priority of Definition." This assumption is discussed in detail elsewhere, particularly in Benson (1990, 2000) and Dancy (2004: 35-64), and we can express it as follows:
\\
\\
The phrase ". . . F -" refers to any declarative sentence that contains "F," "F-ness," or "the F." For example, if "F" is "pious," a possible ". . . F -" sentence could be "this action is pious" or "piety is a good thing." Defining what the F or F-ness is constitutes stating it. Therefore, in order to determine if prosecuting one's father for murder under circumstances like Euthyphro's is pious, one must define piety (Euthphr. 4d9-e8, 5c8-d5, 6d6-e7, 15c11-e1); to evaluate whether something is fine or beautiful (synonymous translations of the Greek term "kalon"), defining beauty is necessary (Hp.Ma. 286c5-d2, 298b11-c2, 304d4-e3). Similarly, to establish whether just individuals are happier than unjust ones, justice must first be defined (R.I, 354a12-c3).
\\
\\
Up to this point, I've employed lowercase letters (e.g., "the beautiful") to refer to the subject matter of Socrates' "what is it?" inquiries. In the middle dialogues, the beautiful is redefined as a Form, namely, "the Beautiful." To abide by this convention, capital letters are used when discussing Forms.
\\
\\
This convention also applies to the term "form" itself. Socrates occasionally refers to what he seeks as an "idea" or a "form" (eidos), with little discernible difference between the two words, both derived from the root "id---," which signifies sight. For consistency, the term "form" will be used henceforth. In Greek, these terms were commonly used to describe the characteristics or qualities of objects, particularly visual ones, and they carried no profound ontological implications. Therefore, in the Socratic dialogues, the word "forms" will be used, while in the middle dialogues, "Forms" will be more appropriate.
\\
\\
Socrates often begins his quest for definitions by determining if he and his counterpart agree that there is something worth investigating. In the Hippias Major (287c8-d2), for example, he asks if there is such a thing as beauty, and Hippias promptly agrees. Such agreements, according to the Theory of Forms, pertain to Forms and assert the existence of the Beautiful, the Form. However, Hippias' quick assent does not imply any significant metaphysical weight to his statement. When Socrates receives such an endorsement, he immediately proceeds to ask for the definition: "Say then, friend, what is the beautiful?" (287d2-3). He does not elaborate on the ontology of beauty. The dialogue's focus is on defining rather than ontology. Similarly, when we talk about animals and distinguish lions from tigers, we are usually unconcerned with the metaphysical question of whether lions are more than just ordinary lions. Socrates appears to be uninterested in this parallel inquiry about beauty being anything beyond beautiful things—at least not yet. This distinction between the Socratic and middle dialogues is critical for this chapter's perspective.
\\
\\
The following is the proposed Theory of Definition for Socrates. It begins with a potential definiens, which is an expression that aims to define a term, the definiendum. The Socratic dialogues require that an adequate definition satisfies the following:
\begin{itemize}
    \item The Substitutivity Requirement: This means that the definiens should be substitutable for the definiendum without altering the truth or falsehood of the sentence containing the definiens (salva veritate).
    \item The Paradigm Requirement: The definiens should provide a standard or model by which instances of the definiendum may be determined.
    \item The Explanatory Requirement: The definiens must elucidate the application of the definiendum.
\end{itemize}