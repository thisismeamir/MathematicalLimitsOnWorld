\documentclass[9pt,a4paper,twocolumn]{article}
\usepackage[utf8]{inputenc}
\usepackage{amsmath}
\usepackage{amsfonts}
\usepackage{amssymb}
\usepackage{url}
\usepackage{makeidx}
\usepackage{graphicx}
\usepackage{graphicx, adjustbox}
\usepackage{lmodern}
\usepackage{fourier}
\usepackage{float}
\usepackage{caption}
\usepackage{wrapfig}
\usepackage{mhchem}
\usepackage[left=1.5cm,right=1.5cm,top=1cm,bottom=3cm]{geometry}
\usepackage{multicol}
\usepackage{soul}



%Colors
\usepackage[dvipsnames]{xcolor}


\definecolor{black}{RGB}{0, 0, 0}
\definecolor{richblack}{RGB}{7, 14, 13}
\definecolor{charcoal}{RGB}{45, 67, 77}
\definecolor{delectricblue}{RGB}{93, 117, 131}
\definecolor{cultured}{RGB}{245, 245, 245}
\definecolor{lightgray}{RGB}{211, 216, 218}
\definecolor{silversand}{RGB}{190, 194, 198}
\definecolor{spanishgray}{RGB}{148, 150, 157}
\definecolor{darkliver}{RGB}{64, 63, 76}

\colorlet{lightdelectricblue}{delectricblue!30}
\colorlet{lightdarkliver}{darkliver!30}


%ColorDefines
\newcommand{\trueblack}[1]{\textcolor{black}{#1}}
\newcommand{\rich}[1]{\textcolor{richblack}{#1}}
\newcommand{\lightblack}[1]{\textcolor{charcoal}{#1}}
\newcommand{\lightrich}[1]{\textcolor{delectricblue}{#1}}
\newcommand{\liver}[1]{\textcolor{darkliver}{#1}}

%Boxes
\usepackage{tcolorbox}
\newtcolorbox{calloutbox}{center,%
    colframe =red!0,%
    colback=cultured,
    title={Callout},
    coltitle=richblack,
    attach title to upper={\ ---\ },
    sharpish corners,
    enlarge by=0.5pt}

\newtcolorbox[use counter=equation]{eq}{center,
	colframe =red!0,
	colback=cultured,
	title={\thetcbcounter},
	coltitle=richblack,
	detach title,
	after upper={\par\hfill\tcbtitle},
	sharpish corners,
    enlarge by=0.5pt }
    
\newtcolorbox{qt}{center,
	colframe=delectricblue,
	colback=white!0,
	title={\large "},
	coltitle=delectricblue,
	attach title to upper,
	after upper ={\large "},
	sharp corners,
	enlarge by=0.5pt,
	boxrule=0pt,
	leftrule=2pt}
	
\newtcolorbox{exc}{center,%
    colframe =red!0,%
    colback=darkliver!15,
    title={Excercise},
    coltitle=richblack,
    attach title to upper={\ ---\ },
    sharpish corners,
    enlarge by=0.5pt}
    
\newcounter{theo}
\newtcolorbox[use counter=theo]{theobox}
	{center,%
    colframe =red!0,%
    colback=cultured,
    title={Theorem \thetcbcounter},
    coltitle=richblack,
    attach title to upper={\ ---\ },
    sharpish corners,
    enlarge by=0.5pt}

\newcounter{def}
\newtcolorbox{define}
{center,
	colframe=darkliver!50,
	colback=white!0,
	title={\large "},
	coltitle=darkliver!50,
	attach title to upper,
	after upper ={\large "},
	sharp corners,
	enlarge by=0.5pt,
	boxrule=0pt,
	leftrule=2pt}

\newtcolorbox{argue}
{center,
	colframe=darkliver!50,
	colback=white!0,
	title={},
	coltitle=darkliver!50,
	attach title to upper,
	after upper ={},
	sharp corners,
	enlarge by=0.5pt,
	boxrule=2pt,
	leftrule=0pt,
    rightrule=0pt}



\newcounter{examplecounter}
\newtcolorbox[use counter=examplecounter]{example}
	{center,%
    colframe =red!0,%
    colback=cultured,
    title={Example},
    coltitle=richblack,
    attach title to upper={\ ---\ },
    sharpish corners,
    enlarge by=0.5pt}

    

        
    
% Highlighters
\newcommand{\hldl}[1]{%
	\sethlcolor{lightdarkliver}%
	\hl{#1}
}
\newcommand{\hldb}[1]{%
    \sethlcolor{lightdelectricblue}%
    \hl{#1}%
}


% Images
\newcounter{figurecounter}
\setcounter{figurecounter}{1}

\newcommand{\img}[3]{
    \begin{figure}[h!]
        \centering
        \captionsetup{justification=centering,margin=0cm,labelformat=empty}
        \includegraphics[width=#2\linewidth]{./img/#1}
        \label{figure}
        \caption{\small\textbf{fig-\thefigurecounter} -- \textcolor{darkliver}{#3}}
    \end{figure}
    \addtocounter{figurecounter}{1}}

\newcommand{\imgr}[3]{
    \begin{wrapfigure}{r}{#2\textwidth}
        \centering
        \captionsetup{justification=centering,margin=0cm,labelformat=empty}
        \includegraphics[width=\linewidth]{./img/#1}
        \label{figure}
        \caption{\small \textbf{fig: \thefigurecounter} -- \textcolor{darkliver}{#3}}
    \end{wrapfigure}
    \addtocounter{figurecounter}{1}}

\newcommand{\imgl}[3]{
    \begin{wrapfigure}{l}{#2\textwidth}
        \centering
        \captionsetup{justification=centering,margin=0cm,labelformat=empty}
        \includegraphics[width=\linewidth]{./img/#1}
        \label{figure}
        \caption{\small \textbf{fig: \thefigurecounter} -- \textcolor{darkliver}{#3}}
    \end{wrapfigure}
    \addtocounter{figurecounter}{1}}

% New commands
\newenvironment{callout}
	{\begin{calloutbox}\color{charcoal}\textbf\textit}
	{\end{calloutbox}}

% for this file
\newcommand{\newpoint}[1]{\indent$\blacktriangleright$ \textbf{#1}}

\title{Scientific Method \\ \large Understanding How Science Works}
\author{Amir H. Ebrahimnezhad}
\date{}

\begin{document}
        \maketitle
        \tableofcontents
        \section*{Abstract}
            While epistemology is a vast philosophy, we consider a more specific part of it. In this draft we are looking at the philosophy of science, scientific method, mathematical modeling and empiricism. By the end of this draft one would know the grounding mechanism that science has been working upon.
        \section{Introduction}
            \subsection{Considering Science}
                Apart from the more general concept of knowledge, as scientists we must have a sense of the term in our boundaries; to know what kind of \textit{knowledge} is considered scientific knowledge. The term knowledge was supposed to have a close relation with Justified True Belief though we mentioned that JTB can also be something that is not knowledge. For scientific knowledge let us begin with the same term.
                \\
                \\
                A Scientific Justified True Belief therefore can be considered a close resemblence of the Scientific Knowledge. Thus we would try to define that first, by which we would have a better understanding of the Scientific Knowledge itself. To define such a term it is clear that we have to define Justification in scientific manner.
                \\
                \\
                But before that what kind of beliefs are scientific? Do people who do a dance to make the sky rain have beliefs that can be considered scientific? A 
        \section{Plato's Theory of the Forms}
            \newpoint{A Realist Theory of Universals}
            We begin our deep dive with some historical approaches to the nature, of which it seems good to start with Plato's Forms. The Theory of forms in Plato's view starts with the distinction between the reality in the percieving manner and the reality of a higher fundamentality. Plato states that non-material abstract forms possess the highest and most fundamental kind of reality. 
            \\
            \newpoint{Sophism: } 

        \section{Aristotle's Philosophy of Science}
            \newpoint{Three Stages for Science: }Aristotle states sceintific inquiry in three stages. The first starts with having an absolute knowledge of the reality, having a knowledge for a certain event in the real world. For instance the knowledge that things fall to the earth (down) if they are not picked or thrown. By possessing such knowledge a scientific explanation begins, where one would try to explain the observed fact (the knowledge) in the principles and the logic he agrees with. Aristotle thought of scientific inquiry as a path from observations to explanatory prionciples and back to observations.\cite{Losee2001-cx}
            \begin{callout}
                Scientific explanation is a trnasition from knowledge of a fact to a knowledge of the reasons for the fact.
            \end{callout}
            \newpoint{Induction:} Aristotle believed that for each thing there is a matter that makes the particular a unique individual, and the form which makes it to be a member of a generalization. He then persuits to state that generalizations about forms are drawn from sense experience, using \textit{induction}. 
            \\
            \\
            \newpoint{Enumeration:} Aristotle considered two types of induction, enumeration and direct intuition. The first one is a simple generalization method. Where a property which seems to be aquiered by several matters, is therefore generalized to be a property of the group those matter belong to. For instance if several humen have five fingers in each hand then we can argue that the group of things, which are called humen has the property of five fingers in each hand. We can go further and look at other animals and since most of mammals would also have five fingers we can extend the group and have the following argument:
            \begin{enumerate}
                \item Several mammals have five fingers in each arms.
                \item $\therefore$ Mammals have five fingers in each arm.
            \end{enumerate}
            In a more general manner we can write Simple Enumeration as:
            \begin{define}
                \textit{Premisses:} What is obsereved to be true of several individuals $\longrightarrow$ \textit{conclusion:} What is presumed to be true of the species to which the individuals belong
            \end{define}
            \newpoint{Intuition:} In the second type one would argue not by the mass number of individuals properties, but by having an insight. In this sense we won't argue because there are many examples that supports our reasoning. But we argue since the logical or \textit{essential} thing is the conclusion that we have. For instance we might have the insight that the bright side of the moons seems to always face the sun. Thus concluding that the light of the moon is just the reflection of the light of the sun.\\
            \newpoint{Deduction:} After the induction, the generalization would be used as premisese for the deduction of statements about the initial observations. There he allowed for types of statements to be used as premisses of deduction.
            \begin{enumerate}
                \item \textbf{A}: All $S$ are $P$. 
                \item \textbf{E}: No $S$ is $P$.
                \item \textbf{I}: Some $S$ are $P$.
                \item \textbf{O}: Some $S$ are not $P$.
            \end{enumerate}
            We therefore have the relations that (1) $S$ wholly included in $P$. (2) $S$ wholly excluded from $P$. (3) $S$ partially included in $P$. (4) $S$ partially excluded from $P$. The first type was considered the most important one among them by Aristotle. Because he believed that certain properties are only for individuals of certain classes he maintained that a proper scientific explanation should be given in terms of statements of this type (type $A$). He then would use the $A$ type statements to form an argument:
            \begin{argue}
                All  $M$ are $P$\\
                All $S$ are $M$\\
                $\therefore$ All $S$ are $P$
            \end{argue}
            The important thing showed by Aristotle was that any arguments validity is depended on the relationship between the premisses and conclusion.\cite{Losee2001-cx}
            \\
            \newpoint{Aristotle and Empirical Requirements for Scientific Explanation:} There are mnay ways to form an argument for proving a statement, but not all of them are stisfying arguments if we deduct a true statement from a false premis the argument itself is not satisfactory because it contains contradictions, error and false statements. Therefore there needs to be some requirements for any argument namely,\textit{The premisses must be, true, indemonstrable, better known than the conclusion, and causes of the attribution made in the conclusion.}
            \\
            \\
            The first rule, is self-evident. If the premisses are false one cannot conclude the conclusion of his argument. The second one is more tricky, by indemonstrable, Aristotle doesn't mean statements, which their validity is unknown.
            \\
            \\
            By indemonstrable, Aristotle means that there has to be some principles at the core which we never quesiton, to avoid regress in explanation, this is still at the core of science where we try to have axioms that we build the theories upon. Consequently, not all knowledge within a science is susceptible to proof. There are statements which we accept as true. In the next drafts we will return to a more formal (mathematical) proof of this for mathematics itself, known as the Godel incompleteness theorem.
            \\
            \\
            The third requirement, states that the laws of science must be self-evident. The premisses, with which we derive the conclusion must be at least as evident as the conlusion itself.
            \\
            \\
            The last requirement is the most important one, where it is important that the premisses, be infact related to the conclusion we want to derive. It is possible to construct a valid syllogisms with true premisses in such a way that the premisses fail to state the cause of the attribution which is made in the conclusion.
            \begin{argue}
                \textit{Syllogism of the Reasoned Fact}\\
                All ruminants with four-chambered stomachs are animals with missing upper incisor teeth.
                \\
                All oxen are runminants with four-chambered stomachs.\\
                $\therefore$ All oxen are animals with missing upper incisor teeth.
            \end{argue}
            \begin{argue}
                \textit{Syllogism of the Fact}
                \\
                All ruminants with cloven hoofs are animals with missing upper incisor teeth.\\
                All oxen are ruminants with cloven hoofs.\\
                $\therefore$ All oxen are animals with missing upper incisor teeth.
            \end{argue}
            The syllogism of the reasoned fact, is stating the cause of the conclusion in it, which is the ability of ruminants to store partially chewed food in one stoamch chamber and to return it to the mouth for further mastication. But the latter syllogism (syllogism of the fact), do not state the cause of the concluded statement, which means it just happened to be an accident.
            \\
            \\
            This can be regarded again as we discussed in the epistemology section, as a True belief that happened to be mere luck of correct guessing. With that said an important step for any scientific inquiry would be to justify our premisses, by which we mean to show that the relation between the premisses and the conclusion is correlational and not accidental. Aristotle would then say that in a causal relation the attribute 
            \begin{enumerate}
                \item is true of every instance of the subject.
                \item is true of the subject preciselt and not as part of a larger whole.
                \item is "essential to" the subject.
            \end{enumerate}
            appart from (1), (2) the third one would be the most confusing. To determine which attributions are essential and which are not. Simple example may make it look clear but in a scientific inquiry and specially with what modern science has achieved this line had become more and more blury.\\
            \\
            \newpoint{The Structure of Science in Aristotle's view:} Although Aristotle didn't specify a criterion of the "essential" attribution of a predicate to a subject class, he did insist that each particular science has a distinctive subject genus and set or predicates. The first principles of a science are not subject to deduction. This can be thought of axioms in a logical systems, no one can question the basis of axioms since they are in fact the most basic statements in a field. Firstly this would stop a regressive questioning and let us build from a set of accepted statements and secondly it produces a field in which the science grows, By not having any basic accepted axioms, we lost the boundaries in which we do the science. Therefore before beginning a science it is best to set a boundary for what we look for, the boundaries are not strict and can change later but it is always a good preception on the boundaries we are in.
            \\
            \\
            Then Aristotle sets one more requirement on scientific interpretations, An adequate explanation of a correlation or process should specify all four aspects of causation, Formal, Material, Efficient, and Final causes.
            \\
            \\
            The formal cause is the genralized pattern of the process. It is to take into account in the most general scenario, what is happening. The material cause is the materialistic view of the process; what is the physical processes involved etc (this may be the concern of the science of physics itself but in Aristotle's view there was a need to separate that and show it in a process with other causes). The efficient cause is the result of the process in the specific materialistic sense. The final cause is the result that is caused by the process itself. For instance a lion possessing teeth has a general patter that it bites and eats with them, the matrial cause would be the structure of it's jaws then the efficient cause is that it can eat and hunt, at last the final cause is that the lion would be fed and let to reproduce and live.

            \section{Formal Philosophy of Science - Karl Popper}
            \newpoint{The Problem of Demarcation: }The first important subject to consider in philosophy of science is to make a distinction between what is \textit{Scientific} and what is \textit{Non-Scientific}.
            




        \newpage
        \bibliography{libscientificmethod}
        \bibliographystyle{plain}

\end{document}