\documentclass[9pt,a4paper,twocolumn]{article}
\usepackage[utf8]{inputenc}
\usepackage{amsmath}
\usepackage{amsfonts}
\usepackage{amssymb}
\usepackage{url}
\usepackage{makeidx}
\usepackage{graphicx}
\usepackage{graphicx, adjustbox}
\usepackage{lmodern}
\usepackage{fourier}
\usepackage{float}
\usepackage{caption}
\usepackage{wrapfig}
\usepackage{mhchem}
\usepackage[left=1.5cm,right=1.5cm,top=2cm,bottom=3cm]{geometry}
\usepackage{multicol}
\usepackage{soul}



%Colors
\usepackage[dvipsnames]{xcolor}


\definecolor{black}{RGB}{0, 0, 0}
\definecolor{richblack}{RGB}{7, 14, 13}
\definecolor{charcoal}{RGB}{45, 67, 77}
\definecolor{delectricblue}{RGB}{93, 117, 131}
\definecolor{cultured}{RGB}{245, 245, 245}
\definecolor{lightgray}{RGB}{211, 216, 218}
\definecolor{silversand}{RGB}{190, 194, 198}
\definecolor{spanishgray}{RGB}{148, 150, 157}
\definecolor{darkliver}{RGB}{64, 63, 76}

\colorlet{lightdelectricblue}{delectricblue!30}
\colorlet{lightdarkliver}{darkliver!30}


%ColorDefines
\newcommand{\trueblack}[1]{\textcolor{black}{#1}}
\newcommand{\rich}[1]{\textcolor{richblack}{#1}}
\newcommand{\lightblack}[1]{\textcolor{charcoal}{#1}}
\newcommand{\lightrich}[1]{\textcolor{delectricblue}{#1}}
\newcommand{\liver}[1]{\textcolor{darkliver}{#1}}

%Boxes
\usepackage{tcolorbox}
\newtcolorbox{calloutbox}{center,%
    colframe =red!0,%
    colback=cultured,
    title={Callout},
    coltitle=richblack,
    attach title to upper={\ ---\ },
    sharpish corners,
    enlarge by=0.5pt}

\newtcolorbox[use counter=equation]{eq}{center,
	colframe =red!0,
	colback=cultured,
	title={\thetcbcounter},
	coltitle=richblack,
	detach title,
	after upper={\par\hfill\tcbtitle},
	sharpish corners,
    enlarge by=0.5pt }
    
\newtcolorbox{qt}{center,
	colframe=delectricblue,
	colback=white!0,
	title={\large "},
	coltitle=delectricblue,
	attach title to upper,
	after upper ={\large "},
	sharp corners,
	enlarge by=0.5pt,
	boxrule=0pt,
	leftrule=2pt}
	
\newtcolorbox{exc}{center,%
    colframe =red!0,%
    colback=darkliver!15,
    title={Excercise},
    coltitle=richblack,
    attach title to upper={\ ---\ },
    sharpish corners,
    enlarge by=0.5pt}
    
\newcounter{theo}
\newtcolorbox[use counter=theo]{theobox}
	{center,%
    colframe =red!0,%
    colback=cultured,
    title={Theorem \thetcbcounter},
    coltitle=richblack,
    attach title to upper={\ ---\ },
    sharpish corners,
    enlarge by=0.5pt}

\newcounter{def}
\newtcolorbox{define}
{center,
	colframe=darkliver!50,
	colback=white!0,
	title={\large "},
	coltitle=darkliver!50,
	attach title to upper,
	after upper ={\large "},
	sharp corners,
	enlarge by=0.5pt,
	boxrule=0pt,
	leftrule=2pt}

\newcounter{examplecounter}
\newtcolorbox[use counter=examplecounter]{example}
	{center,%
    colframe =red!0,%
    colback=cultured,
    title={Example},
    coltitle=richblack,
    attach title to upper={\ ---\ },
    sharpish corners,
    enlarge by=0.5pt}

    

        
    
% Highlighters
\newcommand{\hldl}[1]{%
	\sethlcolor{lightdarkliver}%
	\hl{#1}
}
\newcommand{\hldb}[1]{%
    \sethlcolor{lightdelectricblue}%
    \hl{#1}%
}


% Images
\newcounter{figurecounter}
\setcounter{figurecounter}{1}

\newcommand{\img}[3]{
    \begin{figure}[h!]
        \centering
        \captionsetup{justification=centering,margin=0cm,labelformat=empty}
        \includegraphics[width=#2\linewidth]{./img/#1}
        \label{figure}
        \caption{\small\textbf{fig-\thefigurecounter} -- \textcolor{darkliver}{#3}}
    \end{figure}
    \addtocounter{figurecounter}{1}}

\newcommand{\imgr}[3]{
    \begin{wrapfigure}{r}{#2\textwidth}
        \centering
        \captionsetup{justification=centering,margin=0cm,labelformat=empty}
        \includegraphics[width=\linewidth]{./img/#1}
        \label{figure}
        \caption{\small \textbf{fig: \thefigurecounter} -- \textcolor{darkliver}{#3}}
    \end{wrapfigure}
    \addtocounter{figurecounter}{1}}

\newcommand{\imgl}[3]{
    \begin{wrapfigure}{l}{#2\textwidth}
        \centering
        \captionsetup{justification=centering,margin=0cm,labelformat=empty}
        \includegraphics[width=\linewidth]{./img/#1}
        \label{figure}
        \caption{\small \textbf{fig: \thefigurecounter} -- \textcolor{darkliver}{#3}}
    \end{wrapfigure}
    \addtocounter{figurecounter}{1}}

% New commands
\newenvironment{callout}
	{\begin{calloutbox}\color{charcoal}\textbf\textit}
	{\end{calloutbox}}

% for this file
\newcommand{\newpoint}[1]{\indent$\blacktriangleright$ \textbf{#1}}

\title{Epistemology \\ \large A Review on Knowledge}
\author{Amir H. Ebrahimnezhad}
\date{}

\begin{document}
        \maketitle
        \tableofcontents
        \section*{Abstract}
                In this draft we investigate \textit{Epistemology, How do we know things?, Evidence and Knowldge, Why science works,etc.}
        \section{Introduction}
            \indent At the core of the science there's always a simple question to be answered, the question that needs to be asked and properly investigated before any kind of scientific advancement is acheived. That is \textit{"Why are we sure about the knowledge we have, and what is it after all?"}. The boundaries of science are small to hold such a question within, since they infact are the product of it themselves.
            \\
            \\
            \indent In this draft, I will investigate the philosophy of knowledge, or as it is commonly called \textit{"Epistemology"}. It seems to be a good place to start, since the question of the whole research lies upon the statement that \textit{maybe} it is not possible to know everything about the universe, where we have to first define what we mean by everything. But knowledge itself is where we begin.
            \\
            \\
            \indent Epistemology, concerns itself about the problems and theories regarding knowldege. The word is derived from the Greek words \textit{epistéme} and \textit{logos}, which together means the study of knowledge. But to even begin with such philosophy one must try to define first hand:
            \begin{itemize}
                \item What is knowledge, and what do we mean when we say that we know something?
                \item What is the source of knowledge, how do we gain reliable information and consider them as knowledge?
                \item Is absolute knowledge possible? If not, what are the limitation?\cite{CW/E}
            \end{itemize}
            \indent The first question, seems to be a matter of definition, but an important roleis being played by asking about \textit{"What knowledge is?"}. The importance of the question arises from the fact that by defining knowledge carelessly we might include falsehood with the truth. which by any good considerations, is the last thing, which one in search of knowing would intend to do. Beside that if you define knowledge in a careless manner, you get in trouble to argue for good strategies and sources, and even not be able to find true limitations of knowledge.
            \\
            \\
            \indent The second question concerns us to think about methods, with which we gain information (false or true premises) about anything. What makes a method reliable and other don't. This qeustion incudes the old fashioned problem \textit{Why should we trust science?}, with this question I'll try to show that science, and specifically the process of experimenting is found to be the most reliable way to produce knowledge.
            \\
            \\
            \indent The last question is rather the aim of the project in front of you. This questions invites the careful study of the source of knowledge to be more specific, to show if it has any boundaries, or is it an endless tunnel of ever comming knowledge. We may what to argue, or more clearly, philosophize about the topic. But the important considerations of this questions comes in later drafts (or chapters depending on where you are reading thins). Where we investigate the logic of the world, Computation, and mathematical view of nature and experience.
        
        \section{Knowledge and Justification}
            \newpoint{Absilute Knowledge:} The idea starts with the question, is there an absolute knowledge, and if so is it possible to gain it. Parmenide wanted the idea to be true, and for being so, he describes that knowledge should not depend upon changing observations and experiences, because it has its sole origin in the logic of rational thought; A knowledge that is to assure the experiences but at the same time, a knowledge that is given a priori and is conditioned by nothing but itself. \textit{A Knowledge that can claim for itself absolute centainty and validity.}
            \\
            \\
            Parmendise could be the first to attain the concept that it might be possible to grasp the absolute knowledge of the world, with the vision that the phenomenal reality is merely the deceptive illusiveness of a true and unchangeable worl. This hidden world, was believed to be accessible only by pure reasoning.
            \begin{qt}
                Thus, the idea coagulated that true knowledge of the world could only be arrived at by following the path of rational thought,
            \end{qt}
            The weak point in the ontology of Parmenides was already pointed out by Aristotle. Instead Aristotle introduced a distinction between that which is \textit{"actuality"} and that which might be \textit{"potentiality"}. This he ultimately raised to a point of departure for his own metaphysics, in which he differs clearly from that of the Eleates.
            \\
            \\
            But as Parmendise intended the concept of \textit{true being, that is only accessible through rational thought} was sustained. We'll turn back again after some basic investigation.\cite{Kuppers2018-vv}

            \newpoint{Cognitive Success} is a term used to describe the ability of an individual to think, reason, learn, and solve probelms effectively. The ability to solve problems, and find the true values to things we seek, is a complex process, requiring one's mind to adjust, learn, be creative and manipulate information in a way that is actually useful to solve a problem. But despite that, one can easily argue that if you got the wrong information, false premises and false statements. No amount of intelligent process (without considering luck) would be able to make a useful prediction, or any effective progress toward one's goal. In fact this is known as a motto in data science \textit{Garbage in, Garbage out.}\cite{sep-epistemology}\cite{Clegg2017-ev}
            \\
            \\
            Therefore it is safe to say: \textbf{\textit{By any process, of which we receive information from, we seek statements that are true}}. From here we first have to define a true statement (knowledge) which is the first question posed in the introduction.
            \begin{callout}
                It is worth to note that since this is the study of science, we might not consider all the possible ways one might use \textit{knowledge}. One might know someone, know how to do something, etc... Although one can argue that these concepts are also a higher conceptions of just basic facts (one might know how to do something because he understands basic factual statements of the system and prepared a path to follow, which, because of the facts beneath, happens to reach the desiered goal). We would only talk about, things we consider to be facts, in it's scientific term. (i.e earth is orbiting sun.).
            \end{callout}
                \newpoint{Defining Knowldege:} We have different opinion in different areas of our lives and works, we might have an opinion about who is going to be the president later this year, or if the stocks are going to be bullish or bearish next week; Although we are able to hold any opinion and belief in our mind, we might like to be able to categorize them by some statements.\cite{CW/E}\cite{sep-epistemology}
                \\
                \\
                \newpoint{Validity:} The first way to characterize an statement is the validity. It is safe to assume that we desire statements that we believe to be true. Consider the following statement:
                \begin{align*}
                    \text{Gravity is described by Newton's law}
                \end{align*}
                The statement is true. Not always but if we have an accuracy of a 1700s' scientist, it is most certainly a true statement about the gravity. I would here propose that when we talk about the validity of an statement we might like to consider how accurate are we talking. For that the statement was considered to be a true statement for centuries; Now it is considered true but only if we change a little:
                \begin{align*}
                    \text{In the limit of small velocities (with small accuracies)...}
                \end{align*}
                The validity of the statemen changed over time, it might happen to any statement, for instance if you believe that it's raining outside, you might find it true or false. This is a problem, not only you might find contradiction with what you hold as a belief. But the worst is yet to come, there can be scenarios where you are evaluating an statement correctly (you might be right about the weather), but it just happens to be a lucky guess. 
                \\
                \\
                Certainly we would like true statements we hold, which are not evaluated true by mere luck, to be considered as knowledge. This would lead us to the second characteristic of knowledge.
                \\
                \\
                \newpoint{Justified:} When Alice and Bob say that it's raining outside, where Alice just guessed, and Bob have looked through the window and actually saw the raining. One must consider the two ways, upon which they stated the condition of weather, differently. The first is unable to answer, \textit{"Why do you believe that it's raining?"}, while the latter would.
                \\
                \\
                Assume that Alice and Bob always hold believes, by the way proposed, Alice only guesses, and Bob tries to justify what he considers true and if there's no justification, he would simply change his mind. Now if you are to use the information from one of them, who would you choose? A logical answer would be to always ask Bob, since there's atleast and arguement upon which he considers the belief to be true. 
                \\
                \\
                Knowledge should be justifiable, this is more than just having good excuses to believe something, because it also helps the process of finding truth working, believing without justification cannot be questioned properly (other than questioning the unquestionability itself.). Being justifiable helps us to use the socratic method, either we derive an unquestionable fact underneath, or we find another belief which can or cannot be justified. Therefore it seems that \textit{Knowledge is Justified True Belief.}
                \\
                \\
                But there are problems with such statement, since the justification condition was added to ensure that the belief is not true merely because of luck. For instance believing you have lung cancer, because an astrology magazine suggests, would be considered not justified from a scientist prespective but justified if you believe in astrology.
                \\
                \\
                Edmun Gettier, showed that there are cases of \textit{Justified True Belief} that are not cases of knowledge. JTB, therefore, is not sufficient for knowledge. Cases that this is the case are known as the Gettier cases, which arise because neither the possession of adequate evidence, nor origination in reliable faculties, nor the conjunction of these conditions, is sufficient for ensuring that a belief is not true merely because of luck. This suggests that we must add another element to JTB, so that it is sufficient to be considered Knowledge.\cite{sep-epistemology}
                \\
                \\
                \newpoint{Defining Justification:}
                Imagine a situation where a kid, despite having a birth certificate, and what he has been told his entire life, were to find that the parents he thought are not his actual parents. This situation shows that although a belief was justified, it ultimately came to be wrong. Debates concerning the nature of justification can be understood as debates concernin the nature of such non-knowledge-guaranteeing cognitive successes as the one this imaginary kid would enjoy.
                \\
                \\
                The term justification is used as to say \textit{under no obligation to refrain.} This definition of understanding is labeled as \textit{Dentological Justification} we can define:
                \begin{define}
                    $S$ is justified in doing $x$ if and only if $S$ is not obliged to refrain from doing $x$.
                \end{define}
                So for the term justification we would define:
                \begin{define}
                    $S$ is justified in believing that $p$ if and only if $S$ is not oblied to refrain from believing that that $p$.
                \end{define}
                The dentological understanding of the concept of justification is common among philosophers such as, Descartes, Locke, Moore and Chisholm.
                \\
                \\
                Dentological justification is commonly used, \textit{"Inocent until proven guily"} is an obvious example in law, where we are assumming the most common assertion (people are mostly inocent) until there's an evidence to support otherwise. But such generalization is not the case in science, or to be more specific, until no evidence is gathered, we cannot put our finger on where the logical place to stand is. Although it is important to get back to dentological justification in science. We'll cover the use of it later.
                \\
                \\
                But on the other hand we can define another type for justification:
                \begin{define}
                    $S$ is justified in believing that $p$ if and only if $S$ believes that $p$ in a way that makes it sufficiently likely that her belief is true.
                \end{define}
                Dentological justification, though promising, lacks an important concept, where the justification should be correlated with the evaluation of the belief, one can believe in a justified manner (dentologically), but nevertheless his/her belief is false. The problem arises since dentological justification asserts true until proven false (we are justified to believe that $p$ is true because there's no obligation to refrain us from doing so), This sort of implication puts facts, and unfalsifiable assertions into one basket. We are able to be justified in believing any sort of assertion even if it is not justified.\cite{sep-epistemology}
                \\
                \\
                As an easy example of how a dentological justification might result in a false belief let us review the Russell's teapot analogy, though the claim is to show that the philosophic burden of proof lies upon a person making empirically unfalsifiable, the example would also show how a dentological justification is weak. \cite{enwiki:1149010951}
                \\
                \\
                In his paper "Is There a God?" we have:
                \begin{qt}
                    Many orthodox people speak as though it were the business of sceptics to disprove received dogmas rather than of dogmatists to prove them. This is, of course, a mistake. If I were to suggest that between the Earth and Mars there is a china teapot revolving about the sun in an elliptical orbit, nobody would be able to disprove my assertion provided I were careful to add that the teapot is too small to be revealed even by our most powerful telescopes. But if I were to go on to say that, since my assertion cannot be disproved, it is intolerable presumption on the part of human reason to doubt it, I should rightly be thought to be talking nonsense. If, however, the existence of such a teapot were affirmed in ancient books, taught as the sacred truth every Sunday, and instilled into the minds of children at school, hesitation to believe in its existence would become a mark of eccentricity and entitle the doubter to the attentions of the psychiatrist in an 10 enlightened age or of the Inquisitor in an earlier time.
                \end{qt}
                Although the debate, that is there any evidence to prove gods existence remains, for what seems like forever, the burden of proof is always upon the one claiming it. There are several more cases that can easily fit under the label \textit{Knowledge}, that no-one would accept, most of the folklore sotries of beings such as Zeus, Thor, unicorns, etc... were widely regarded as a true belief, with the justification that you cannot disprove their existence. But any person in the 21st century would deny their existence. Therefore, the second type (Sufficiently Likely Justification), seems to work the best for most cases, and be the justification that Russell would mean.\cite{Russell1952}
                \\
                \\
                \newpoint{Evidentialism:} Whether a blief is truley justified or not, there's something that makes it so,


        
        
        
        
        
        
        
        
        
        
        
        \newpage
        \bibliography{library}
        \bibliographystyle{plain}

\end{document}