\documentclass[10pt,a4paper]{article}
\usepackage[utf8]{inputenc}
\usepackage[T1]{fontenc}
\usepackage{amsmath}
\usepackage{amsfonts}
\usepackage{amssymb}
\usepackage{url}
\usepackage{makeidx}
\usepackage{graphicx}
\usepackage{graphicx, adjustbox}
\usepackage{lmodern}
\usepackage{fourier}
\usepackage{float}
\usepackage{caption}
\usepackage{wrapfig}
\usepackage{mhchem}
\usepackage[left=1.5cm,right=1.5cm,top=1cm,bottom=3cm]{geometry}
\usepackage{multicol}
\usepackage{soul}



%Colors
\usepackage[dvipsnames]{xcolor}


\definecolor{black}{RGB}{0, 0, 0}
\definecolor{richblack}{RGB}{7, 14, 13}
\definecolor{charcoal}{RGB}{45, 67, 77}
\definecolor{delectricblue}{RGB}{93, 117, 131}
\definecolor{cultured}{RGB}{245, 245, 245}
\definecolor{lightgray}{RGB}{211, 216, 218}
\definecolor{silversand}{RGB}{190, 194, 198}
\definecolor{spanishgray}{RGB}{148, 150, 157}
\definecolor{darkliver}{RGB}{64, 63, 76}

\colorlet{lightdelectricblue}{delectricblue!30}
\colorlet{lightdarkliver}{darkliver!30}


%ColorDefines
\newcommand{\trueblack}[1]{\textcolor{black}{#1}}
\newcommand{\rich}[1]{\textcolor{richblack}{#1}}
\newcommand{\lightblack}[1]{\textcolor{charcoal}{#1}}
\newcommand{\lightrich}[1]{\textcolor{delectricblue}{#1}}


%Boxes
\usepackage{tcolorbox}
\newtcolorbox{calloutbox}{center,%
    colframe =red!0,%
    colback=cultured,
    title={Callout},
    coltitle=richblack,
    attach title to upper={\ ---\ },
    sharpish corners,
    enlarge by=0.5pt}

\newtcolorbox[use counter=equation]{eq}{center,
	colframe =red!0,
	colback=cultured,
	title={\thetcbcounter},
	coltitle=richblack,
	detach title,
	after upper={\par\hfill\tcbtitle},
	sharpish corners,
    enlarge by=0.5pt }
    
\newtcolorbox{qt}{center,
	colframe=delectricblue,
	colback=white!0,
	title={\large "},
	coltitle=delectricblue,
	attach title to upper,
	after upper ={\large "},
	sharp corners,
	enlarge by=0.5pt,
	boxrule=0pt,
	leftrule=2pt}
	
\newtcolorbox{exc}{center,%
    colframe =red!0,%
    colback=darkliver!15,
    title={Excercise},
    coltitle=richblack,
    attach title to upper={\ ---\ },
    sharpish corners,
    enlarge by=0.5pt}
    
\newcounter{theo}
\newtcolorbox[use counter=theo]{theorem}
	{center,%
    colframe =red!0,%
    colback=cultured,
    title={Theorem \thetcbcounter},
    coltitle=richblack,
    attach title to upper={\ ---\ },
    sharpish corners,
    enlarge by=0.5pt}

\newcounter{defcounting}
\newtcolorbox[use counter=defcounting]{define}
{center,%
	colframe=darkliver!50,%
	colback=white!0,
	title={\textcolor{black}{\textbf{\textit{Definition}} \  \thetcbcounter  \ --}},
	coltitle=darkliver!50,
	attach title to upper,
	after upper ={ },
	sharp corners,
	enlarge by=0.5pt,
	boxrule=0pt,
	leftrule=2pt,
    rightrule = 0pt}

\newcounter{lemmacount}
\newtcolorbox[use counter=lemmacount]{lemma}
{center,%
    colframe=charcoal!50,%
    colback=white!0,
    title={\textcolor{black}{\textbf{\textit{Lemma}} \  \thetcbcounter  \ --}},
    coltitle=darkliver!50,
    attach title to upper,
    after upper ={ },
    sharp corners,
    enlarge by=0.5pt,
    boxrule=2pt}
    

\newcounter{examplecounter}
\newtcolorbox[use counter=examplecounter]{example}
	{center,%
    colframe =red!0,%
    colback=cultured,
    title={Example},
    coltitle=richblack,
    attach title to upper={\ ---\ },
    sharpish corners,
    enlarge by=0.5pt}

    

        
    
% Highlighters
\newcommand{\hldl}[1]{%
	\sethlcolor{lightdarkliver}%
	\hl{#1}
}
\newcommand{\hldb}[1]{%
    \sethlcolor{lightdelectricblue}%
    \hl{#1}%
}


% Images
\newcounter{figurecounter}
\setcounter{figurecounter}{1}

\newcommand{\img}[3]{
    \begin{figure}[h!]
        \centering
        \captionsetup{justification=centering,margin=0cm,labelformat=empty}
        \includegraphics[width=#2\linewidth]{./img/#1}
        \label{figure}
        \caption{\small\textbf{fig-\thefigurecounter} -- \textcolor{darkliver}{#3}}
    \end{figure}
    \addtocounter{figurecounter}{1}}

\newcommand{\imgr}[3]{
    \begin{wrapfigure}{r}{#2\textwidth}
        \centering
        \captionsetup{justification=centering,margin=0cm,labelformat=empty}
        \includegraphics[width=\linewidth]{./img/#1}
        \label{figure}
        \caption{\small \textbf{fig: \thefigurecounter} -- \textcolor{darkliver}{#3}}
    \end{wrapfigure}
    \addtocounter{figurecounter}{1}}

\newcommand{\imgl}[3]{
    \begin{wrapfigure}{l}{#2\textwidth}
        \centering
        \captionsetup{justification=centering,margin=0cm,labelformat=empty}
        \includegraphics[width=\linewidth]{./img/#1}
        \label{figure}
        \caption{\small \textbf{fig: \thefigurecounter} -- \textcolor{darkliver}{#3}}
    \end{wrapfigure}
    \addtocounter{figurecounter}{1}}




% New commands
\newenvironment{callout}
	{\begin{calloutbox}\color{charcoal}\textbf\textit}
	{\end{calloutbox}}



% for this file
\newcommand{\newpoint}[1]{\indent$\mathsection$ \textbf{#1}}
\newcommand{\curveL}{\mathcal{L}}
\newcommand{\curveA}{\mathcal{A}}
\newcommand{\curveP}{\mathcal{P}}
\newcommand{\thm}{\text{Thm}}
\newcommand{\proof}{\\ \ \\ $\blacktriangleright$ \textit{proof: }}
\usepackage{xepersian}
\settextfont{Arabic Typesetting}
\title{حدود ریاضیاتی بر نظریات فیزیک
\\ \large بررسی مسئله اندازه گیری و ناتمامیت گودل}
\author{امیرحسین ابراهیم نژاد}
\date{\today}

\begin{document}
        \maketitle
        \tableofcontents
        \newpage
        \section{Ground Works}
        \subsection{Vorwort}
         در این مقاله، ما به سوی درک اپیستمولوژی پایه، تئوری ریاضی سیستم های فرمال، روش اثبات بیانیه ها و استنتاج به عنوان شرح ریاضی خالص، قضایای ناتمامی و محدودیت های محاسباتی حرکت خواهیم کرد. سپس در مورد اندازه گیری و رفتارهای مشابه طبیعت صحبت خواهیم کرد و آنچه را قضایای ناتمامی پیشنهاد می دهند.
    \section{What is Epistemology?}
            \indent در اصل علم، همیشه یک سوال ساده برای پاسخ دادن وجود دارد؛ سوالی که باید پرسیده شود و قبل از هرگونه پیشرفت علمی به درستی بررسی شود. این سوال \textit{«چرا مطمئن هستیم از دانشی که داریم و در نهایت آن چیست؟»} است. مرزهای علم نیز برای نگه داشتن چنین سوالی خیلی کوچک هستند، زیرا در واقع خودشان نتیجه آن هستند.
            \\
            \\
           این به نظر می‌رسد یک مکان مناسب برای شروع است، زیرا سؤال تحقیق کلی بر این حرف استوار است که \textit{شاید} امکان دانستن همه چیز درباره جهان وجود نداشته باشد و باید ابتدا تعریف کنیم که به وسیله «همه چیز» چه چیزی را می‌گوییم. اما خود دانش جایی است که ما شروع می‌کنیم.
            \\
            \\
            \indent اپیستمولوژی، با مسائل و نظریه‌های مربوط به دانش سروکار دارد. این واژه از کلمات یونانی "اپیستمه" و "لاگوس" برگرفته شده است که به همراه هم به معنای مطالعه دانش هستند. اما برای شروع به فلسفه چنینی، باید در ابتدا سعی کرد تا تعریفی به صورت مستقیم برای آن ارائه داد:
            \begin{itemize}
                \item دانش چیست و به چه معنا می‌گوییم که چیزی را می‌دانیم؟
                \item منبع دانش چیست و چگونه اطلاعات قابل اعتمادی را به دست می‌آوریم و آنها را به عنوان دانش در نظر می‌گیریم؟
                \item آیا دانش مطلق ممکن است؟ اگر نه، محدودیت های آن چیستند؟ \cite{CW/E}
            \end{itemize}
            \indent سؤال اول به نظر می‌رسد مسأله ای تعریفی است، اما از آنجایی که درباره "چیستی دانش" سؤال مطرح می‌کنیم، نقش مهمی را بازی می‌کند. اهمیت این سؤال از آن جاست که با تعریف نادرست دانش، ممکن است حقیقت را با دروغ در یک بستر قرار دهیم. که هرگز هدف هیچ کسی که در جستجوی دانش است نیست. به علاوه، اگر دانش را به صورت نادرستی تعریف کنیم، ممکن است برای استدلال به استراتژی‌های خوب و منابع مناسب مشکل داشته باشیم و حتی نتوانیم محدودیت‌های واقعی دانش را پیدا کنیم.
            \\
            \\
            \indent سؤال دوم ما باعث می‌شود که درباره روش‌هایی که با آن‌ها اطلاعات (پیش فرض‌های درست یا نادرست) درباره هر چیزی بدست می‌آوریم، فکر کنیم. این سؤال شامل مشکل قدیمی \textit{«چرا باید به علم اعتماد کنیم؟»} است، با این سؤال من سعی خواهم کرد نشان دهم که علم و به ویژه فرایند آزمایشگاهی به عنوان روش پایدارتری برای تولید دانش شناخته شده است.
            \\
            \\
            \indent سؤال آخر بیشتر هدف پروژه‌ای است که در پیش رو دارید. این سؤال دعوت به مطالعه دقیق منبع دانش دارد و به صورت خاص، نشان می‌دهد که آیا محدودیتی دارد یا آیا یک تونل بی‌پایان از دانش است که همواره دارای دانش جدید است. ممکن است خواهیم خواست بحث و تأمل در مورد این موضوع داشته باشیم، جایی که منطق جهان، محاسبات، دیدگاه ریاضی درباره طبیعت و تجربه را بررسی خواهیم کرد.سؤال آخر بیشتر هدف پروژه‌ای است که در پیش رو دارید. این سؤال دعوت به مطالعه دقیق منبع دانش دارد و به صورت خاص، نشان می‌دهد که آیا محدودیتی دارد یا آیا یک تونل بی‌پایان از دانش است که همواره دارای دانش جدید است. ممکن است خواهیم خواست بحث و تأمل در مورد این موضوع داشته باشیم، جایی که منطق جهان، محاسبات، دیدگاه ریاضی درباره طبیعت و تجربه را بررسی خواهیم کرد.
        
        \section{Knowledge and Justification}
            \subsection{Absilute Knowledge}ایده با سؤال آغاز می‌شود، آیا دانش مطلق وجود دارد و در صورت وجود، آیا ممکن است به آن دست یابیم؟ پارمنیدس خواست تا این ایده درست باشد و برای ثابت شدن آن، توضیح داد که دانش نباید به مشاهدات و تجربیاتی که به تغییر دچار می‌شوند، وابسته باشد، چرا که مبنای آن در منطق تفکر عقلانی قرار دارد؛ یک دانش که برای تأمین تجربیات ضروری است، اما در عین حال، یک دانشی است که پیشاپیش داده شده است و با هیچ چیز جز خودش شرطی ندارد. \textit{دانشی که می‌تواند برای خودش ادعای قطعیت و اعتبار مطلق را داشته باشد.}
            \\
            \\
           پارمنیدس می‌تواند نخستین شخصی باشد که به این مفهوم دست یافته است که ممکن است بتواند دانش مطلق جهان را درک کند، با دیدگاهی که واقعیت ظاهری تنها یک پوچ و فریبنده از جهان واقعی و ثابت است. جهان پنهان، فقط توسط استدلال خالص قابل دسترسی بود.
            \begin{qt}
               بنابراین، این ایده تثبیت شد که دانش واقعی جهان تنها با پیروی از مسیر تفکر منطقی قابل دستیابی است.
            \end{qt}
            نقطه ضعف در انتولوژی پارمنیدس قبلاً توسط آریستوتل مورد توجه قرار گرفته بود. به جای آن، آریستوتل تمایزی بین آن چیزی که در واقعیت وجود دارد («عملیات») و آن چیزی که ممکن است وجود داشته باشد («احتمالی») معرفی کرد. این تمایز را او در نهایت به عنوان نقطه شروعی برای فلسفه خود ارتقا داد؛ در فلسفه خود، که از فلسفه الیات‌ها به طور مشخصی متمایز است، این تمایز را معرفی کرد.
            \\
            \\
            اما همانطور که پارمنیدس می‌خواست، مفهوم "واقعی بودن" که فقط توسط تفکر منطقی قابل دسترسی است، حفظ شد. بعد از چندین تحقیقات پایه‌ای، به موضوع مورد نظر باز می‌گردیم..\cite{Kuppers2018-vv}
            \newpoint{Cognitive Success}ذهن شخصیتی است که برای تفکر، استدلال، یادگیری و حل مسائل به طور مؤثر نیاز دارید. امکان حل مسائل و یافتن ارزش‌های واقعی در چیزهایی که به دنبال آن هستیم، یک فرایند پیچیده است که نیازمند تنظیم، یادگیری، خلاقیت و اطلاعات مفید برای حل یک مسأله به یک نحو مفید است. اما با این حال، می‌توان به راحتی بحث کرد که اگر به اطلاعات اشتباه، پیش فرض‌های نادرست و عبارات غلط دسترسی داشته باشید، هیچ میزان فرآیند هوشمند (بدون در نظر گرفتن شانس) نمی‌تواند پیش‌بینی مفیدی تولید کند یا به سمت هدف خود پیشرفتی مؤثر داشته باشد. در واقع، این با عنوان شعاری در علوم داده شناختی شناخته می‌شود: "ورودی غیر مفید، خروجی غیر مفید".\cite{sep-epistemology}\cite{Clegg2017-ev}
            \\
            \\
            بنابراین، می‌توان گفت: \textbf{\textit{با هر فرآیندی که اطلاعات را از آن دریافت می‌کنیم، به دنبال عباراتی هستیم که صحیح باشد.}} از اینجا لازم است ابتدا یک عبارت درست (دانش) را تعریف کنیم که سؤال اول مطرح شده در مقدمه است.
            \begin{callout}
                Iشاید ارزش داشته باشد بیان کنم که از آنجا که این مطالعه در حوزه علم است، ممکن است همه روش‌های ممکنی که یک فرد ممکن است از \textit{دانش} استفاده کند را در نظر نگیریم. یک فرد می‌تواند شخصی را بشناسد، یک کاری را بلد باشد و غیره... با این حال، می‌توان بحث کرد که این مفاهیم نیز نمونه بالاتری از حقایق پایه هستند (یک فرد ممکن است چیزی را بداند چون او جملات پایه‌ای سیستم را درک کرده است و مسیری برای دنبال کردن آن آماده کرده است که به دلیل حقایقی که در پایین قرار دارد، به هدف مورد نظر می‌رسد). ما فقط درباره چیزهایی صحبت خواهیم کرد که به عنوان واقعیت‌ها در اصطلاح علمی در نظر می‌گیریم. (به عنوان مثال، زمین دور خورشید در حرکت است.)
            \end{callout}
                \subsection{Defining Knowldege} در زندگی و کار ما در حوزه‌های مختلف، نظرات متفاوتی داریم و ممکن است در مورد اینکه چه کسی در انتخابات ریاست جمهوری در سال جاری برنده خواهد شد یا آیا بازار سهام هفته آینده قوی یا ضعیف خواهد بود، نظر داشته باشیم؛ اگرچه ما قادر به داشتن هرگونه نظر و باوری در ذهن خود هستیم، ممکن است بخواهیم آنها را با برخی عبارات دسته‌بندی کنیم.\cite{CW/E}\cite{sep-epistemology}
                \\
                \\
                \subsubsection{Validity}راه اول برای توصیف یک عبارت، صحت آن است. می‌توان فرض کرد که ما به دنبال عباراتی هستیم که به آنها باور داریم و درست باشند. به بیان یک عبارت نمونه:
                \begin{align*}
                    \text{جاذبه بوسیله قانون نیوتن توصیف می‌شود}
                \end{align*}
                این عبارت صحیح است. البته این موضوع در همه موارد درست نیست، اما اگر به دقت یک دانشمند در سال 1700 بازگردیم، آنگاه این عبارت با اطمینان بسیاری در مورد جاذبه صحیح است. در اینجا پیشنهاد می‌دهم که وقتی درباره صحت یک عبارت صحبت می‌کنیم، شاید بخواهیم در نظر بگیریم که در حال حاضر چقدر دقیق صحبت می‌کنیم. برای این منظور، این عبارت برای قرن‌ها به عنوان یک عبارت درست در مورد جاذبه در نظر گرفته شده است؛ امروزه نیز می‌توان آن را صحیح دانست، اما تنها در صورتی که کمی آن را تغییر دهیم:
                \begin{align*}
                    \text{در حد کوچکی از سرعت‌ها (با دقت‌های کوچک)...}
                \end{align*}   

                    صحت این عبارت در طول زمان تغییر کرده است، این مشکل می‌تواند به هر عبارت دیگری هم اتفاق بیفتد، برای مثال اگر باور داشته باشید که در خارج باران می‌بارد، ممکن است این باور را صحیح یا نادرست بدانید. این یک مشکل است، نه تنها می‌توانید با تضاد با باورهای خود روبرو شوید، بلکه بدترین این است که ممکن است در شرایطی قرار داشته باشید که یک عبارت را به درستی ارزیابی کرده‌اید (ممکن است در مورد وضعیت آب و هوا درست بوده باشید)، اما فقط یک حدس خوش شانس بوده باشد.
                    \\
                    \\
                    بدون شک، ما به دنبال عبارات صحیحی هستیم که آنها را به عنوان دانش در نظر بگیریم و از تصادف در ارزیابی یک عبارت به عنوان دانش خودداری کنیم. این مسئله به ما منجر خواهد شد به مرحله‌ی دوم تعریف دانش، که خاصیت دوم آن است.
                    \\
                    \\
                    \subsubsection{Justified} وقتی آلیس و باب می‌گویند که باران می‌بارد، در حالی که آلیس فقط حدس زده است و باب از پنجره بیرون را نگاه کرده و به واقعیت بارش باران پی برده است، باید دو روشی که به کمک آن‌ها شرایط آب و هوا را بیان کرده‌اند، را به صورت متفاوت در نظر گرفت. در اینجا، آلیس قادر نیست به سؤال "چرا باور داری که باران می‌بارد؟" پاسخ دهد، در حالی که باب می‌تواند به این سؤال پاسخ دهد.
                    \\
                    \\
                    فرض کنید همیشه آلیس و باب با استفاده از روشی که پیشنهاد شد، باور دارند. آلیس فقط حدس می‌زند و باب سعی می‌کند به چیزی که به عنوان درست در نظر گرفته است توجیهی ارائه دهد و اگر توجیهی وجود نداشته باشد، ساده‌ترین راه را انتخاب کرده و عقیده خود را تغییر می‌دهد. اگر شما بخواهید از اطلاعات یکی از آن‌ها استفاده کنید، کدام یک را انتخاب می‌کنید؟ پاسخ منطقی این است که همیشه باید از باب بپرسید، زیرا حداقل یک دلیلی وجود دارد که او باور دارد که باور خود را درست است.
                    \\
                    \\
                   دانش باید توجیه‌پذیر باشد؛ این بیشتر از داشتن بهانه‌های خوب برای باور چیزی است، زیرا این کمک می‌کند فرایند یافتن حقیقت کار کند و باور درخصوصی که بدون توجیه باشد، نمی‌تواند به درستی سوال شود (علاوه بر پرسیدن از خود قابل پرسش بودن خود). بودن قابل توجیه بودن به ما کمک می‌کند تا از روش سقراطی استفاده کنیم، به این صورت که یا یک حقیقت قابل پرسش پایین‌تر پیدا می‌کنیم، یا باور دیگری را پیدا می‌کنیم که ممکن است قابل توجیه باشد یا نباشد. بنابراین، به نظر می‌رسد که «دانش باور درست توجیه‌شده است».
                    \\
                    \\
                    اما با این حال، مشکلاتی با چنین بیانیه‌ای وجود دارد؛ زیرا شرط توجیه، برای اطمینان از اینکه باور فقط به دلیل شانس درست نیست، اضافه شده است. به عنوان مثال، اعتقاد به این که شما سرطان ریه دارید، به دلیل اینکه یک مجله طالع‌بینی چنین گفته باشد، از دیدگاه یک دانشمند به عنوان قابل توجیه در نظر گرفته نمی‌شود، اما اگر به طالع‌بینی باور داشته باشید، قابل توجیه خواهد بود.
                    \\
                    \\
                   ایدموند گتیر نشان داد که برخی از موارد «باور درست توجیه‌شده» (JTB)، مواردی از دانش نیستند. به همین دلیل JTB، کافی برای داشتن دانش نیست. مواردی که این اتفاق رخ می‌دهند، با نام Gettier cases شناخته می‌شوند و به دلیل اینکه حضور شواهد کافی، یا برگرفته از ظرفیت‌های قابل اعتماد، یا اتصال این شرایط، کافی برای اطمینان از اینکه یک باور فقط به دلیل شانس درست نیست، نیستند. این نشان می‌دهد که باید عنصر دیگری را به JTB اضافه کنیم، تا کافی باشد برای آنکه به عنوان دانش در نظر گرفته شود.\cite{sep-epistemology}
                    \\
                    \\
                    \subsection{Defining Justification}
                   تصور کنید یک کودک، علیرغم داشتن گواهینامه تولد و هر آنچه در طول عمر خود به او گفته شده است، متوجه شود که پدر و مادری که فکر می‌کرده از والدین واقعی او نیستند. این وضعیت نشان می‌دهد که علیرغم این که باور به توجیه رسیده بود، در نهایت نادرست شد. بحث‌های مربوط به ماهیت توجیه، می‌تواند به عنوان بحث‌های مربوط به ماهیت موفقیت‌های شناختی، که قطعیت در دانش را تضمین نمی‌کنند، مانند موفقیتی که این بچه تصوری به دست می‌آورد، درک شوند..
                    \\
                    \\
                    اصطلاح توجیه به عنوان یک روش برای گفتن "زیر هیچ التزامی برای خودداری نیست" استفاده می‌شود. این تعریف از درک با عنوان توجیه دانتولوژیک شناخته می‌شود، و ما می‌توانیم آن را به شرح زیر تعریف کنیم:

«توجیه دانتولوژیک»: درکی که بیان می‌کند که فردی هیچ التزامی برای خودداری از باور خود ندارد.
                    \begin{define}
                        فرد $S$ در انجام عمل $x$، تنها در صورتی دارای توجیه است که فرد $S$ مجبور به خودداری از انجام $x$ نباشد. به بیان دیگر، فرد $S$ توجیه شده در انجام $x$ است اگر و تنها اگر وظیفه‌ای برای از سر گیری از انجام $x$ نداشته باشد..
                    \end{define}
                    برای تعریف اصطلاح توجیه، می‌توانیم به شرح زیر عمل کنیم:

«توجیه»: وضعیتی که فردی در آن نباید از باور یا عمل خودداری کند، چون هیچ التزامی برای این خودداری وجود ندارد.
                    \begin{define}
                        فرد $S$ در باور داشتن ادعای $p$، تنها در صورتی توجیه شده است که فرد $S$ مجبور به خودداری از باور داشتن $p$ نباشد. به عبارت دیگر، فرد $S$ توجیه شده در باور داشتن $p$ است اگر و تنها اگر وظیفه‌ای برای از سر گیری از باور داشتن $p$ نداشته باشد.
                    \end{define}
                   درک دانتولوژیک از مفهوم توجیه، در میان فلاسفه‌ای چون دکارت، لاک، مور و چیشولم رایج است. به طور کلی، این فلاسفه باور داشتند که توجیه باید به معنای عدم وجود هرگونه الزام یا وظیفه برای خودداری از باور یا انجام چیزی باشد. به عبارت دیگر، این فلاسفه باور داشتند که به جای تلاش برای پیدا کردن دلایل و شواهد قطعی برای باور، کافی است که هیچ وظیفه‌ای برای خودداری از باور نباشد.
                    \\
                    \\
                   دانتولوژی توجیه به طور رایجی مورد استفاده قرار می‌گیرد؛ به عنوان مثال در قانون، "برای ناتوانی داریم تا اثبات گردد" یک مثال آشکار از آن است که در آن فرض می‌شود که بیشترین تأیید درستی (معمولاً) به معنای بی‌گناهی فرد وجود دارد تا زمانی که شواهدی برای اثبات خلاف آن وجود داشته باشد. اما این تعمیم در علم صدق نمی‌کند، به عبارت دیگر، تا زمانی که شواهدی جمع‌آوری نشده باشند، نمی‌توانیم به طور منطقی درباره محل قرار گرفتن صحیح اطلاعات حساسیت داشته باشیم. با این حال، بازگشت به توجیه دانتولوژیک در علم نیز حائز اهمیت است. به محض اینکه شواهدی جمع‌آوری شد، می‌توان به این توجیه بازگشت کرد.
                    \\
                    \\
                    اما به طرف دیگر، می‌توانیم نوع دیگری از توجیه را تعریف کنیم
                    \begin{define}
                        فرد $S$ در باور داشتن ادعای $p$ توجیه شده است، به شرطی که فرد $S$ به گونه‌ای به باورش دست یافته باشد که احتمال صحت باور او کافی باشد. به عبارت دیگر، فرد $S$ تنها در صورتی توجیه شده در باور داشتن $p$ است که باور او به گونه‌ای باشد که به خوبی می‌تواند نشان دهد باور او به $p$ با احتمال قابل قبولی درست باشد.
                    \end{define}
                    توجیه دانتولوژیک، اگرچه قابل اعتماد است، در مفهوم مهمی کمبود دارد که باید توجه شود: ارتباط توجیه با ارزیابی باور. ممکن است کسی به گونه‌ای که با توجیه دانتولوژیک باور دارد، باور خود را داشته باشد، اما باز هم باور او نادرست باشد. مشکل اینجاست که توجیه دانتولوژیک تأکید دارد که باور درست است تا زمانی که خلاف آن ثابت شود (ما توجیه شده‌ایم که به $p$ باور داشته باشیم زیرا هیچ التزامی برای خودداری از این باور وجود ندارد)، در نتیجه ممکن است واقعیت‌ها و دعاوی غیرقابل اثبات را در یک سبد قرار دهد. به عبارت دیگر، ممکن است ما در باور ادعایی باشیم که در واقع توجیه شده نباشد، اما باز هم توجیه دانتولوژیک به عنوان معیاری برای قبول این باور، صحیح در نظر گرفته شود.\cite{sep-epistemology}
                    \\
                    \\
                    به عنوان یک مثال ساده از اینکه چگونه توجیه دانتولوژیک می‌تواند به باور نادرستی منجر شود، بیایید به مثال مشهور "چای‌سرویس راسل" بپردازیم. این مثال هدف دارد که نشان دهد بار فلسفی اثبات بر عهده شخصی است که ادعایی غیرقابل اثبات تجربی دارد، اما همچنین مثال نشان می‌دهد که توجیه دانتولوژیک چقدر ضعیف است.\cite{enwiki:1149010951}
                    \\
                    \\
                    در مقاله‌اش با عنوان "آیا خدا وجود دارد؟"، او مطرح کرد:
                    \begin{qt}
                       در این مقاله، او می‌گوید: "بسیاری از افراد مذهبی، به گونه‌ای صحبت می‌کنند که نشان دادن عدم درستی گزاره‌های پذیرفته شده را بر عهده شکاکان قرار داده‌اند و نه برعکس، بر عهده دغل‌بازان است که برای ثابت کردن خودشان تلاش کنند. البته، این یک اشتباه است. اگر من اینگونه پیشنهاد کنم که بین زمین و مریخ، یک چای سرویس چینی در مدار بیضی‌شکلی حول خورشید می‌گردد، هیچ کس نمی‌تواند اظهارات من را رد کند، به شرطی که دقت کنم که این چای سرویس به اندازه کافی کوچک است که حتی با قوی‌ترین تلسکوپ‌های ما قابل مشاهده نیست. اما اگر من به علاوه از این بگویم که با توجه به اینکه اظهارات من رد نشده‌اند، شک کردن به آن‌ها نشانه تعلق یک روزگار باشد، در زمان معاصر باید به عنوان حرف بی‌معنی تلقی شوم. اما اگر وجود چنین چای سرویسی در کتاب‌های باستانی تأیید شده باشد و هر یکشنبه به عنوان حقیقت مقدس تدریس شود و به ذهن کودکان در مدارس تزریق شود، تردید از وجود آن یک علامت از غرایت خواهد بود و موجب می‌شود که شخص متشکک به جایزه پزشکی نیاز داشته باشد، در صورتی که در دوران تاریکی موجب شد که او به عنوان مرتد محاکمه شود."
                    \end{qt}
                    اگرچه بحث در مورد وجود خدا به نظر می‌آید همواره ادامه دارد، با این حال بار اثبات همیشه بر عهده کسی است که آن را ادعا می‌کند. چندین مورد دیگر وجود دارند که به راحتی قابل تطبیق با برچسب "دانش" هستند و هیچ کس آن‌ها را باور نمی‌کند. بسیاری از داستان‌های محلی مانند زئوس، تور، تک شاخ و ... به عنوان باور صحیح واقعی در نظر گرفته می‌شدند، با توجیهی که می‌توانید وجود آن‌ها را رد نکنید. اما هر کسی در قرن بیست و یکم از وجود آن‌ها انکار می‌کند. به همین دلیل، نوع دوم (توجیه کافی) به نظر می‌آید برای اکثر موارد بهترین توجیه باشد که راسل نیز به آن اشاره کرده بود.\cite{Russell1952}
                    \\
                    \\
                    \subsection{Kant, Schelling, Fichte and Evidence} همانطور که اشاره کردیم، ایده پارمنیدس بر این اساس است که دانش واقعی فقط بر اساس تفکر منطقی باید مبتنی باشد که به بازسازی واقعیت با روش استنتاجی منجر می‌شود. این ایده توسط ریاضیدان اقلیدس در بنیادهای اصولی هندسه خود به کار گرفته شده است که به رشد منطق بنیادگذاری شده است. ایده این بود که دانش واقعی باید به طور کامل از نخستین و بالاترین حقیقت و قابل تکذیب باشد و همچنین باید توانایی ارائه توجیهی برای ادعای آن داشته باشد که دانشی که از آن حاصل شده است، سازگار و درست باشد. این اصل هنوز هم توسط فیزیکدانانی که درگیر فرمول‌بندی نهایی برای توصیف طبیعت در کلیت خود هستند، به کار می‌رود.
                    \\
                    \\
                    این یک قدم بسیار پایه‌ای اما منطقی به سوی دانش است، زیرا دنیای اطراف به نظر مرتب و منظم می‌آید و مرتبط بودن با منطق شروع می‌شود (به طوری که ممکن است بتوانید یک سیستم منظم را با هرج و مرج در بنیاد آن ایجاد کنید، اما ما به دنبال قدمی متداولتر هستیم)..\cite{Kuppers2018-vv}
                    \\
                    \\
فلسفه کانت بر توانایی و محدودیت های عقل تمرکز دارد. کانت دو سؤال مطرح می کند: آیا استدلال می تواند به ما دانش متافیزیکی بدهد، همانطور که راشیدگان ادعا می کنند؟ و آیا عقل می تواند راهنمای عملیاتی باشد و اصول اخلاقی را توجیه کند؟ کانت در برابر راشیدگان بر این باور است که اگر مرزهایی مانند دانش از خدا یا جهانی فراتر از حواس مورد توجه قرار نگیرند، استدلال به تناقض منجر خواهد شد. و در برابر تجربویان، که مدعی هستند عواطف و نه عقل ما را به سوی عمل هدایت می کنند، او ادعا می کند که عقل می تواند ما را به سمت اصولی هدایت کند که می تواند بین افراد عقلانی به اشتراک گذاشته شود.
                    \\
                    \\
                    کانت می‌گوید که دانش را به دو روش حسیت و درک کسب می‌کنیم و قضاوت تجربی بر هر دوی این روش‌ها بستگی دارد. پس از آن، در کتاب خود به بحث \textit{پیرامونی گفت‌وگو} می‌پردازد. او در برابر تلاش فیلسوفانی همچون پارمنیدس که سعی در خارج کردن دانش واقعی از اشیاء جهان دارند، استدلال می‌کند. وی افزود که \textit{پیرامونی گفت‌وگو} برای اشیاءی که توسط حس ها نمایش داده نمی‌شوند، \textit{منطق اغترار} است (این بخش در فصول بعدی کتاب که وارد منطق ریاضی می‌شویم، بسیار مهم می‌شود و نشان می‌دهد که می‌تواند چندین سیستم منطقی وجود داشته باشد، با این وجود نامطلوب هستند.)\cite{sep-kant-reason}
                    \begin{qt}
                        قانون عقل برای جستجوی وحدت الزامی است، زیرا بدون این قانون ما عقلی نخواهیم داشت و بدون آن، هیچ استفاده مترابطی از درک نخواهیم داشت و اگر این گونه باشد، نشانه کافی برای حقیقت تجربی وجود نخواهد داشت.
                    \end{qt}
                    تجزیه و تحلیل شرایطی که زیرساخت دانش را فراهم می‌کند، او را به مفهوم \textit{موضوع پرتوان} به عنوان منبع دانش قبل از هر تجربه رساند. وی بر این باور است که درک موضوعات جهان خارجی توسط موارد موجود در خود آن‌ها، یعنی آنچه که او آن را \textit{اشیاء نسبت به خود} نامید، تحت تأثیر قرار می‌گیرد. به عبارت دیگر، این مفهوم به صورت درونی و بدون وابستگی به تجربه ما وجود دارد که می‌توان آن را به عنوان "انتزاعی" تصور کرد، که به طرز ساده‌تری توسط دموکریتوس بیان شده است، ایده‌ای که در بحث دموکریتوس در مورد خدایان به کار رفته است، که در آن مشخص می‌شود که دانش ما از خدایان از ذرات بزرگ اتمی، که مبتنی بر دیدگاه اتمی است، تشکیل شده است. گزارشی ذکر می‌کند که دموکریتوس و لوکیپوس ادعا کردند که هم فکر و هم احساس به علت تصاویری هستند که از خارج بر روی بدن تأثیر می‌گذارند و فکر و درک نیز به صورتی که حساب می‌شود بر تصاویر وابسته است.\cite{Cartledge1998-CARDTG-3}\cite{sep-democritus}\cite{Graham2010-th}
                    \\
                    \\
این مفهوم در ابتدا توسط فیشته در سال ۱۷۹۴ در کتابش \textit{"عقاید در علم"} رد شد. او به این استدلال می‌کند که وظیفه تولید دانش بر عهده "اشیاء نسبت به خود" قرار گرفته، باعث می‌شود که دانش همچنان وابسته به جهان خارجی باشد و به همین دلیل خصوصیت بی‌قید و شرط برای آن وجود ندارد. با این حال، به نظر فیشته، بدون وجود بی قید و شرط، دانش به عنوان پیش نیاز ضروری برای کسب دانش توسط موضوع پرتوان، به صورت بدون قید و شرط نمی‌تواند مطلق باشد و دیگر به تغییرات تجربه خارجی وابسته نیست. بنابراین، فیشته از ایده آن شروع به کرد که عملکرد فعالیت‌های موضوع پرتوان باید به طور کامل بی قید و شرط باشد، به طوری که تنها به خودش بستگی داشته باشد.v
                    \\
                    \\
                   افراطی گرایی زیادتی که در اینجا می‌بینیم، در زمان فیشته هدف انتقادات بود. رویکرد فلسفی فیشته که موضوع پرتوان را به عنوان تنها و بی قید و شرط منبع دانش ترویج می‌دهد، منجر به تعارض با واقعیت تجربی می‌شود. در قابلیت درک واقعیت خود، ایده‌آلیسم ذهنی فیشته به روشنی نشان می‌دهد که با ضعف‌های مشابه، از جمله از دیدگاه پارمنیدس درباره واقعیت وجود واقعی آن است.
                    \\
                    \\
                    در کتابش \textit{"افکاری درباره فلسفه طبیعت"} که در سال ۱۷۹۷ منتشر شد، شلینگ با ابتکاری به تصحیح این نقص پرداخت. او در ابتدا هویت موضوع-فرضیه را در قالبی موضوعی قرار داد و نه همانند فیشته، در نظر گرفتن آن به عنوان یک هویت که به صورت انحصاری از موضوع برخاسته است. علاوه بر این، بنا به گفته شلینگ، باید هویت موضوع-فرضیه را به عنوان یک مقوله مطلق در نظر گرفت. این بدان معناست که کل موضوعی که در وجود دارد همزمان کل فرضیه‌ای است که در وجود دارد و کل فرضیه‌ای که در وجود دارد، همزمان کل موضوعی است که در وجود دارد.
                    \\
                    \\
برخلاف فیشته، در فلسفه شلینگ، جهان واقع بیش از یک تصویر از جهان ایده‌آل است. او اظهار داشت که ظاهرهای مفهومی و مادی به عنوان دو نمایشی از همان موجود باید در نظر گرفته شوند، که هویت مطلق موضوع-فرضیه آن است.
                \begin{qt}
                    "طبیعت ذهن قابل دیدن است، و ذهن طبیعت غیرقابل دیدن است"، باید به عنوان این معنا گرفته شود که موضوع پرتوان می‌تواند خود را در طبیعت به عنوان در یک آینه ببیند. به بیان دیگر، طبیعت ذهن قابل مشاهده است. به طور معکوس، ذهن طبیعت غیر قابل دیدن است، به این معنی که ذهن بالاترین سطح وجود خود را با بازتاب طبیعت تصویر می‌کند. بنابراین، ذهن در طبیعت و طبیعت در ذهن می‌توانند یکدیگر را مشاهده کنند.
                \end{qt}
در سامانه شلینگ، وظیفه علم تجربی بهترین حالت آن است که اصولی را که توسط فلسفه طبیعت به آن دیکته شده‌اند، تأیید کند. در هیچ حالی نمی‌توانند رد شوند؛ زیرا رد این اصول بلافاصله با منطق و اصول دلیل‌گونه در تضاد خواهد بود و به معنایی جستن از پیشامدهای شناختی خواهد بود. در واقع، اصول فلسفه طبیعت به عنوان اصولی بی‌چالش قابل تأیید تلقی می‌شوند. اگر نتایج تجربی با این اصول مطابقت ندارند، در این صورت اصول بدون چالش باقی می‌مانند، در حالی که مشاهدات تجربی به وضوح ناکارآمد، ناقص یا فریبنده در نظر گرفته می‌شوند. این نزدیک‌تر به مفاهیمی است که توسط پارمنیدس و فیشته معرفی شده‌اند تا به چیزی که علم دارای ماهیت علمی است نزدیک باشد. 

اگر چه درست است که اظهارات منطقی بدون چالش رد نمی‌شوند، این امر به دلیل ساختار منطقی آن‌هاست و نه به دلیل ارزش بیشتر آن‌ها در ارتباط با تجربه. اظهارات منطقی بدون چالش مانده زیرا بر اساس سیستم‌های منطقی ساخته شده‌اند که از قبل توافق بر آن صورت گرفته است. به این معنا که هر سیستم منطقی در حد خود به خوبی کار می‌کند، و اگر این گونه باشد، طبیعت به هزاران روش به طور همزمان عمل خواهد کرد تا منطق برقرار باشد. با این حال، اگر فرض کنیم منطق جهان یکتا باقی می‌ماند، در آن صورت کار با منطق و استفاده از اطلاعات تجربی فقط برای ثابت کردن نقاط نظر خود به یک نکته مهم پوشیده خواهد شد، که این است که باید بین بسیاری از روش‌های ممکن منطقی انتخاب کنیم و همچنان انتظار داریم که این انتخاب درست باشد و جهان را به درستی توصیف کند.
                \\
                \\
                یک جنبه دیگر از علم شناختی شلینگ باید تأکید شود. با توجه به اصل هویت، ایده آل و واقعی در کنار یکدیگر یک کل را تشکیل می‌دهند که نمی‌توان آن را فراتر از آن تجاوز کرد. کل در عین حال یک استعاره برای مطلق است، که با این حال فقط به شکل دوگانه- یعنی در جوهر ایده آل و واقعی- به موضوع نشان داده می‌شود. با این حال، مطلق هرگز نباید به ایده آل و واقعی "توسعه" پیدا کرده و خارج از مطلق سوق داده شود. به عنوان مطلق، همیشه باید با خودش در تمام مطلقیتش هم‌تطابق داشته باشد.
                \begin{qt}
                   فلسفه طبیعت و تحقیقات تجربی در طبیعت بنابراین با دو مفهوم بنیادین مختلف از دانش سروکار دارند. یکی به "طبیعت به عنوان موضوع" و دیگری به "طبیعت به عنوان شیء" اختصاص دارد. "طبیعت به عنوان موضوع" استعاره ای برای پویایی نامحدود طبیعت ("natura naturans") است. این به معنای کاملاً پویایی طبیعی است و نیروهای محرک آن، اصول طبیعی خلاق، هستند. کشف این اصول، وظیفه فلسفه طبیعت است. "طبیعت به عنوان شیء"، در مقابل، به عنوان تولید پذیرفته شده طبیعت در محصولات خود ("natura naturata") است. این محصولات به ذات خود محدود هستند و به شبکه ای از اعمال ختم شده تبدیل می‌شوند، که روشن سازی آن وظیفه تحقیقات تجربی در طبیعت است. با این حال، برای جلوگیری از تجزیه نظری طبیعت به دو شکل، شلینگ از یک دستکاری استفاده کرد.
                \end{qt}
               به طبق این نظریه، پویایی طبیعت در محصولات خود واقعاً خاموش نمی شود؛ بلکه هنوز با قدرت تولیدی فعال است که با این حال، به صورت بی‌نهایت تأخیر دارد. همانطور که در فلسفه الهه‌گرایی هم روی می‌دهد، برای نجات پایداری مدل شناختی، مفهوم بی‌نهایت مجدداً باید به کار گرفته شود.\cite{Kuppers2018-vv}
    
                به طور خلاصه، می‌توان گفت که فلسفه طبیعت شلینگ در دو جنبه مهم با روش علمی امروزی در تضاد بود:
                \begin{itemize}
                    \item در فلسفه شلینگ، نظریه به مراتب مهمتر از تجربه‌گرایی است و ادعاهای صحت شناختی نیاز به بررسی تجربی ندارند؛ بلکه این ادعاها کاملاً از استدلال منطقی مشتق می‌شوند. به طور خلاصه: دانش پیشین اولویت دارد نسبت به دانش پسین.
                    \item روش تحقیقاتی که توسط دکارت، نیوتن و دیگران پیشنهاد شده است و بر این اساس است که باید از ساده به پیچیده، از قسمت به کل و از علت به معلول حرکت کرد، توسط شلینگ به روش برعکس تبدیل شده است. روش تحلیلی، که بر مبنای تجزیه، انتزاع و ساده‌سازی استوار است، کنار گذاشته شده و یا حداقل در اولویت کمتری قرار می‌گیرد، و جای خود را به روش هلیستیک می‌دهد.
                \end{itemize}
                افزودنی به مقدمه "ایده‌های یک فلسفه طبیعت" شلینگ، که در آن بارها تلاش می‌کند برای بیان غیر قابل انعکاس است، با ابداعات واژه‌شناسانه و مقایسه‌های تصویری پر از جذابیت است که در انتهای راه خود به انتزاع نامفهوم منتهی می‌شوند. به عنوان مثال، می‌خوانیم که مطلق "درون خودش محصور و پوشیده شده است"، یا اینکه مطلق "از شب وجود خود به روز زاییده می‌شود". شلینگ در آنجا در مورد "تنفس کامل بطور مطلق" و "رمز طبیعت" صحبت می‌کند.
                \\
                \\
حتی فیلسوفان نزدیک به دایره رمانتیک یئنا مانند فریدریش شلگن و یوهان ویلهلم ریتر، از مفهوم اینکه تخمین خالص بدون هیچ تجربه‌ای، می‌تواند اساس دانش عمیقی درباره جهان فراهم کند، انتقاد کردند. به گفتۀ ریتر: "ما به طور نامحسوس به نظریه حقیقی نزدیک می‌شویم، بدون آنکه به دنبال آن باشیم - ما آن را با مشاهده آنچه در واقع رخ می‌دهد پیدا خواهیم کرد، زیرا چه چیزی بیشتر از این نیاز داریم که نظریه به ما بگوید چه چیزی در واقع رخ می‌دهد؟"\cite{Kuppers2018-vv} 
    
            \section{اثبات‌گرایی } 
                آیا باور واقعاً به‌طور صحیح توجیه شده است یا خیر، چیزی وجود دارد که آن را توجیه می‌کند. اما قبل از شروع به بررسی مفهوم شواهد، کاربردی است که برخی از ایده‌های پارمنیدس، شلینگ و دیگران را بررسی کنیم.\cite{sep-epistemology}
                \\
                \\
                قوی‌ترین مورد برای توجیه دقت، به نظر می‌رسد مورد شواهد باشد.
                \begin{define}
به گفته اثبات‌گرایی، فرد تنها زمانی مشروع به باور شیئی است که در صورت داشتن شواهد محکم، آن باور را پشتیبانی کند.. \cite{enwiki:1149588226}
                \end{define}
                یک باور بر پایه منطقی و دلایل معقول قائم می‌شود. بنابراین، استدلال باید به عنوان یک مبنای صحیح برای باور در نظر گرفته شود. داشتن دلایل مناسب، ضروری است برای توجیه هر باور. ریچارد فِلدمن و ارل کان، دو دفاع‌کننده برجسته اثبات‌گرایی بودند که با استفاده از تعریف خود، این مکتب فلسفی را دفاع می‌کردند.
                \begin{define}
                    اثبات‌گرایی یک پایان‌نامه درباره وضعیت توجیهی تمامی نگرش‌های ذهنی است که شامل باور، عدم باور و تعلیق قضاوت می‌شود. به بیان دیگر، این مفهوم فلسفی بیان می‌کند که چگونگی توجیه هریک از نگرش‌های ذهنی نظیر باور، عدم باور و تعلیق قضاوت بر چه اساس اتفاق می‌افتد.
                \end{define}
                با توجه به اثبات‌گرایی، تعریف مجدد توجیه به این صورت است:
                \begin{define}
نگرش ذهنی یا همان دیدگاه شخص d نسبت به پیام pدر زمان t تنها در صورتی می‌تواند مشروع باشد که شواهد و مدارک فرد در آن زمان، پشتیبانی از دیدگاه وی نسبت به پیام p باشند
                \end{define}                

\newpoint{Evidence:}شواهد حمایت یا مخالفت با $p$ هر اطلاعاتی است که می‌تواند اعتبار $p$ را ارزیابی کند. به عبارت دیگر، هر جمله‌ای که می‌تواند صحت یا نادرستی $p$ را اثبات کند..\cite{Mittag2015-qc}
                \\
                \\
با چنین تعریف گسترده‌ای که داریم، می‌توانیم شلینگ، فیشته و پارمنیدس را به‌عنوان شواهد در نظر بگیریم. اما همانطور که اشاره کردیم، طبیعت چنین توجیه‌هایی از علمی بودن خیلی دور است. در اینجا روش‌هایی که فرد ممکن است به عنوان راه‌های کسب شواهد در نظر بگیرد، بررسی می‌شود.
                \\
                \\
                \subsection{Rationalism and Reliabilism}ما قبلا بررسی کرده‌ایم که شلینگ چگونه تفکر منطقی و فرایند منطقی را به عنوان مدرک می‌پذیرد. او اهمیت تجربه‌گرایی را نیز می‌پذیرد، اما به عنوان معادلی برای تفکر منطقی آن را در نظر نمی‌گیرد، لذا می‌توانیم او را یک رشدی تلقی کنیم. رشدی ها اعتقاد دارند که منطق منبع دانش است و سیلوگیسم‌ها، یعنی استدلال منطقی که از استدلال استنتاجی برای رسیدن به نتیجه‌ای استفاده می‌کند، می‌توانند -- و اگر به درستی استفاده شوند -- به دست آوردن دانش کمک کنند. \cite{enwiki:1145200450}
                \begin{align*}
                    \text{If} A, \text{then} B;
                    \\ A;
                    \\ \therefore B
                \end{align*}
مشکل این روش در این است که محدودیتی برای پیش‌فرض‌هایی که استفاده می‌شوند وجود ندارد. یک فرد می‌تواند با استفاده از این منطق، استدلال‌هایی به شکل زیر را تشکیل دهد:
                \begin{align*}
                    \text{If Alice is awake, it is morning.}
                    \\ \text {Alice is awake.}
                    \\ \therefore \text{It's morning.}
                \end{align*}
               اگرچه بسیاری از مردم این جملات را به عنوان درست می‌پذیرند و از این رو تمام استدلال را به عنوان درست در نظر می‌گیرند، با این حال، می‌توان به سادگی مشاهده کرد که پیش‌فرض، به طور ضروری نتیجه را دنبال نخواهد کرد (آلیس می‌تواند به بی‌خوابی دچار شود و هیچ‌گاه خواب نکند). علاوه بر استفاده از سیستم‌های منطقی مختلف که می‌تواند با واقعیت در تعارض باشد، استفاده از پیش‌فرض‌هایی که به نتیجه نمی‌انجامد، یکی از مشکلاتی است که ممکن است در بسیاری از استدلال‌های منطقی وجود داشته باشد. این مشکل بهتر است به عنوان یکی از ابزارهایی در جعبه ابزار شواهد در نظر گرفته شود، نه به عنوان شواهد به طور کلی.
                \\
                \\
                صرفاً تفکر منطقی به خودی خود نباید به عنوان شواهد در نظر گرفته شود. پیش‌فرض‌ها باید به نتیجه دنبال کنند و منطق باید منطق واقعی باشد که در طبیعت (واقعیت) عمل می‌کند. نتیجه‌گیری از این نکته این است که تنها تفکر منطقی کافی نیست تا به عنوان شواهد در نظر گرفته شود. \cite{CW/E}
                \\
                \\
                توسعه خارج از استدلال انسانی، که به عنوان خطاگیر ثابت شده است، اگر از یک رویکرد ریاضی استفاده شود، چه اتفاقی می افتد؟ بررسی اینکه آیا ریاضیات می تواند به عنوان یک پایه سنگی قوی عمل کند، به بخش های بعدی منتقل می شود، اما در حال حاضر، مانند آنچه که برای خطاهای منطقی ذکر کردیم، ممکن است به دلیل داشتن این استدلال، به موارد زیر توجه کنید:
                \begin{align*}
                    x &=c \\
                    x^2 &= cx\\
                    x^2 - c^2 &= cx - c^2\\
                    (x+c)(x-c) &= c(x-c)\\
                    x + c &= c \\
                    2c &= c \\ 
                    2 &=1 \\ 
                \end{align*}
اینجا یک دلیلی را پیدا کرده ایم که با قوانین ریاضی کار می کند، اما با این حال منجر به خطاها می شود. برای دقیق تر بودن، این مشکل به خاطر ریاضیات نیست، زیرا در خودش، وقتی صفر است، اجازه نمی دهد $(x-c)$ از طرفین معادله حذف شود. در اینجا ما با عدم استفاده صحیح از سیستم منطقی، اشتباه کرده ایم.. \cite{CW/E}
                \\
                \\
                \subsection{Coherentism}
ما می‌توانیم به دنبال یافتن یک حکم همواره درست باشیم که به هر حالتی نمی‌تواند غلط باشد، سپس تلاش کنیم از طریق سایر حکم‌ها بررسی کنیم که آیا می‌توان آن‌ها را از آن حکم استخراج کرد یا نه؟ یا آیا هیچ حکم دیگری وجود دارد که اعتبار آن توسط حکم اصلی ثابت شده باشد؟ با استفاده از این روش، به دنبال حکم‌های منسجم در کنار حکم اصلی درست می‌گردیم. 
                \\
                \\
                روش منطقی یا "هماهنگ‌گرایی" به یک روش دیگر برای شناخت می‌انجامد که ممکن است در همان اندازه نادرست باشد و آن روش "قابلیت اعتماد کامل" است. \cite{CW/E}, Wenning argues:
                \begin{qt}
در ذهن افراد قرون وسطی، منطقی بود که زمین در مرکز جهان، پایین ترین نقطه زیر آسمان ها باشد. برای فکرکردگان قرون وسطی، انسانیت در مرکز جهان قرار داشت؛ نه به دلیل وضعیت نجیب و بالایی که به عنوان قله آفرینش داشتیم، بلکه به خاطر طبیعت فاسد ما. نزدیک­ترین نقطه به مرکز جهان، جایی است که برای بدترین همه، به خصوص جهنم، رزرو شده است. کسانی که خیلی بد نیستند، پس از مرگ به دنیای زیرزمینی یا هادز منتقل می‌شوند، اما نه به جهنم.
                \end{qt}
این همان دلیلی است که در نگاه قرون وسطی، بهشت به عنوان "بالا" و جهنم به عنوان "پایین" بازنمایی شده بود. موقعیت انسان نزدیک به یا در مرکز جهان، علتی برای فخر و تکبر نبود؛ بلکه مسئله‌ای بود که در رابطه انسان با خدایان کاملاً منطقی بود. این باور کاملاً قابل قبول بود. تفسیر مسائل به هر نحو دیگری، در نظر گرفتن فهم الهی آن زمان، معنی نمی‌داد. با این حال، همه این نتایج اشتباه بودند؛ این مشکل به دلیل این است که با انتخاب یک اظهارنظر منسجم با حکم همه درست، فرض می‌شود که این حکم همیشه درست است، در نتیجه سیستم به محض اثبات اشتباه بودن یکی از حکم‌های همه درست، منهدم می‌شود. استفاده از چنین روشی می‌تواند برای جمع آوری حکم‌های منطقی مفید باشد، اما به خطرناکی می‌انجامد اگر ما فقط به این روش اکتفا کنیم. ما می‌خواهیم یک قانون پیدا کنیم که پیامدهای مثبتی داشته باشد و سپس قوانین منطقی دیگری را پیدا کنیم که نتایج بیشتری را القا کنند؛ اما ما باید همیشه در جستجوی رد کردن چنین قانونی باشیم. اما چگونه؟
                
            \section{Empiricism}
                تجربی‌گرایان همچنین همانند راشناسند، اصول دستور/استنتاج را قبول می کنند، اما با تفاوت مهمی؛ استفاده از تفکر منطقی، ریاضیات و منطق برای ثابت کردن خود به تنهایی نیست، بلکه برای هدایت ما در بررسی چیزهایی که بررسی می کنیم، استفاده می شود. با عکس العمل، تجربی گرایان تصورات و مفاهیم فطری را رد می کنند. تا جایی که ما در یک موضوع دانش داریم، دانش ما به وسیله تجربه های ما به دست می آید، به صورت حسی یا تأملی. به عبارت دیگر، تجربه تنها منبع ایده های ماست. علاوه بر این، آنها نسخه متناظر اصالت دانش و اصالت مفهوم را رد می کنند. از آنجا که تنها تفکر به تنهایی به ما هیچ دانشی نمی دهد، قطعاً به ما دانش برتری نمی دهد. تجربی گرایان نیازی به رد اصالت یافتنی استدلال ندارند، اما بیشتر آنها در این زمینه برخورداری را نپذیرفته‌اند.\cite{sep-rationalism-empiricism}
                \begin{define}
                    \textit{The EMpiricism Thesis: }ما هیچ منبع دانشی در مورد S یا برای مفاهیمی که در S استفاده می‌کنیم جز تجربه نداریم.
                \end{define}
                \begin{callout}
                    برای روشنی بیشتر، تئوری تجربگرایی به معنای داشتن دانش تجربی نیست، بلکه به معنای این است که دانش تنها با تجربه ممکن است به دست آید، در صورتی که اصلاً ممکن باشد.
                \end{callout}
تجربی گرایی به دلیل داده های تجربی بهتر نیست، بلکه به این خاطر است که از بهترین روش های دیگر نیز استفاده می کند. داده ها برای جلوگیری از اشتباهات استفاده می شوند.
                \\
                \\
ما توسط بسیاری از چیزها کور شده‌ایم؛ اشتباهات منطقی ما، حداقل تفکر ما در جهان، توانایی ما برای ایجاد اشتباهات در سیستم های منطقی و غیره. بنابراین، کسی می‌تواند دلیل کند که به دنبال یافتن منبعی نباشد که در خودمان است، بلکه در هر چیز دیگری که در منطق عمل می کند، اما توسط نقص های ما تهدید نمی شود. علتی که فلسفه طبیعی به سمت تجربگرایی پیچیده است، ممکن است از همین دلیل باشد؛ از آنجا که وظیفه خود توصیف طبیعت است، چرا از طبیعت برای دریافت دانش استفاده نکنیم؟
                \\
                \\
در اینجا ممکن است سوالی در مورد نقش تفکر منطقی، ریاضیات و منطق پیش بیاید. استفاده از آنها برای اختراع طبیعت نیست؛ همانطور که قبلاً دیدیم، ممکن است به مشکلاتی برخوردهایم. اما مشکلات با تنها مشاهده طبیعت هم دو گانه هستند؛ اولاً برای نتیجه گیری بسیار سخت است، زیرا سیستم های طبیعی پیچیده هستند، بنابراین نمی توانیم به راحتی شاهد باشیم و قانون پشت آن را استخراج کنیم، اگرچه سعی می کنیم سیستم های ساده را در آزمایشگاه جدا کنیم تا بتوانیم بیشتر با آنها آشنا شویم. دوم، روش های متعددی برای توصیف فرایند طبیعی در اصطلاحات و تأملات وجود دارد. آریستوتل می گفت که اشیاء به پایین می افتند چون هر یک از چهار عنصر (زمین، هوا، آتش و آب) محل طبیعی خود را دارد و این عناصر تمایلی برای بازگشت به محل طبیعی خود دارند. بنابراین، اشیاء ساخته شده از زمین می خواهند به زمین بازگردند، در حالی که آتش برای مثال به سوی بهشت می رود.\cite{greg-grav}.این باعث می‌شود که به شکل سخت‌گیرانه‌تری در باره استدلال‌های خود فکر کنیم و از ریاضیات که از کلمات بیشتری استفاده می‌کند و ارتباطات را توصیف می‌کند، استفاده کنیم.
                \begin{equation}
                    F = \frac{d}{dt} p
                \end{equation}
قانون نیوتن رابطه بین نیروی خالص و جرمی یک جسم را که با استفاده از ریاضیات تعریف شده‌اند، توصیف می‌کند. با این معادله، نیازی به گفتن این نیست که یک جسم به سمت چه قرار است حرکت کند، ما فقط باید بپرسیم آیا نیرویی بر روی آن عمل می کند یا خیر؟ به همین دلیل، استفاده از یک فرمت سخت‌گیرانه از منطق برای توصیف آنچه به نظر می‌رسد یک طبیعت منطقی است ضروری است.
        
            \section{Formal Language Theory}
بررسی قضیه ناتمامیت نیاز به دانشی از مفاهیم پایه مانند زبان‌ها، گرامرها، اتوماتاها و سایر موارد دارد. بنابراین، در این بخش با صحبت درباره این موضوعات به شکل کلی، شروع می‌کنیم. نسخه‌ای با جزئیات بیشتر در ادامه منتشر خواهد شد.\cite{Leary2019-ip}
                \subsection{Languages}
ما با تعریف یک زبان شروع می‌کنیم. به طور غیررسمی، زبان آن است که ما صحبت می‌کنیم یا در برگه‌ای برای ارائه اطلاعات به فرد دیگری نوشته می‌کنیم. اما به صورت ریاضی و یک ایده بیشتر انتزاعی از یک زبان، یک دسته از نمادها است که برای ساخت کلمات از آنها استفاده می‌کنیم (چسباندن نمادها به هم به منظور شباهت با یکدیگر). برای ارائه تعریف ریاضی دقیقی از زبان، ابتدا الفبا را تعریف می‌کنیم به عنوان:
                    \begin{define}
                        \textit{An alphabet $\Sigma$, is a set of symbols.}
                    \end{define}
                    \begin{define}
                        \textit{A word $M$, is a combination of symbols from alphabet $\Sigma$.}
                    \end{define}
به عنوان مثال، می‌توان یک مجموعه مانند {a،b،c،d،… }{a،b،c،d،…} را به عنوان الفبای خود انتخاب کرد. در این صورت، "catcat" یک کلمه در الفبای آن است. همانطور که مشاهده می کنید برای کلمه‌ای که به عنوان مثال آورده شد یک معنی وجود دارد که یک شیء (به دقت تر یک موجود زنده) را تعریف می کند. با توجه به اینکه می‌توان برای هر الفبایی به صورت نامحدود کلمات ایجاد کرد (بجز مجموعه‌ای خالی که هیچ کلمه‌ای ندارد)، با تعریف یک دستور زبان برای زبان خود، بین ترکیب های قابل قبول و غیرقابل قبول تمایز قائل می‌شویم. یک دستور زبان یا گرامر راهی برای مشخص کردن یک زبان است، یک راه برای لیست کردن رشته های قابل قبول ΣΣ است. ما می‌توانیم به سادگی رشته ها را لیست کنیم یا مجموعه ای از قواعد (یا یک الگوریتم) برای تعیین اینکه آیا یک ترکیب داده شده برای یک زبان قابل قبول است یا نه، داشته باشیم. بنابراین، زبان را به عنوان زیر تعریف می‌کنیم:
                    \begin{define}
                        \textit{Given an alphabet $\Sigma$, $\Sigma^\infty$ is the set of all possible words in the alphabet.}
                    \end{define}
                    \begin{define}
                        \textit{A subset $S$ of a set $X$ is decidable if and only if there exists a function that given $x\in X$ decides if $x\in S$ is true or false.}
                    \end{define}
                    \begin{define}
                        \textit{A Language $L$, is a subset of the alphabet $\Sigma^\infty$ ($L\subset \Sigma^\infty$) where there exists a function $\eta(\sigma\in\Sigma^\infty)$ called grammar that decides $L$.}
                    \end{define}
                    به شکل رسمی، ما یک دستور زبان را به شرح زیر تعریف می‌کنیم:
                    \begin{define}
                        \textit{A Grammar is a set $\{V_T,V_N,S, R\}$ where $V_T$ is the set of terminal elements, $V_N$ is the set of non-terminal elements, $S$ is a memeber of $V_N$, and $R$ is a finite set of rules.}
                    \end{define}
ما در بخش‌های بعدی از این مقاله از این تعاریف استفاده خواهیم کرد. اما برای الان، ما به شکل رسمی دیگری از $R$ نیز اشاره خواهیم کرد:\cite{Leary2019-ip}
                    \begin{define}
                        \textit{$R$ is a finite set of ordered pairs from $\Sigma^\infty V_N \Sigma^\infty\times \Sigma^\infty$, where $\Sigma = V_T\cup V_N$.}
                    \end{define}
                    \newpoint{Induction:} Induction is a method of proving that a statemens $P(n)$ is true for every natural number $n$, that is, that the infinitely many cases, $P(0), P(1),\dots$ all hold. \cite{enwiki:1157726892}
                    \begin{qt}
                        اصل استقرای ریاضی اثبات می‌کند که ما می‌توانیم هر قدر بخواهیم بالای یک نردبان برویم، با اثبات اینکه می‌توانیم به پایین‌ترین درجه‌ی آن صعود کنیم (پایه) و از هر درجه به درجه‌ی بعدی بتوانیم صعود کنیم (گام).
                    \end{qt}
                    \begin{theorem}
                        For every natural number $n$,
                        \begin{equation}
                            1+2+\dots + n = \frac{n(n+1)}{2}
                        \end{equation}
                    \
                    \proof If $n=1$, the equality holds. For the inductive case, fix $k\geq 1$ and assume that:
                    \begin{equation}
                        1+2+\dots+k =\frac{k(k+1)}{2}
                    \end{equation}
                    Now adding $k+1$ to each side we have:
                    \begin{equation}
                        1+2+\dots+(k+1) = \frac{k(k+1)}{2}+(k+1)
                    \end{equation}
                    Since the right hand side simplifies to:
                    \begin{equation}
                        \frac{(k+1)((k+1) + 1)}{2}
                    \end{equation}
                    با پایان دادن به مرحله استقرایی و در نتیجه اثبات، همانطور که در مرحله استقرایی مشاهده می‌کنید، آنچه را که ما اثبات می‌کنیم عبارت است از:
                    \begin{qt}
                        \textit{If the formula holds for $k$, then the formula holds for $k+1$.}
                    \end{qt}
                    \end{theorem}
                    از یک دیدگاه کمی متفاوت، آنچه که ما انجام داده‌ایم، ساخت یک مجموعه از اعداد با ویژگی خاص است. اگر $S$ را برای مجموعه اعدادی که قضیه ما برای آنها صدق می‌کند به کار ببریم، در اثبات استقرایی خود نشان دادیم که مجموعه $S$ با مجموعه اعداد طبیعی همانی است، بدین ترتیب قضیه برای هر عدد طبیعی $n$ برقرار است، همانطور که لازم است.
                    \\
                    \\
                    آنچه که باعث می‌شود اثبات استقرایی به خوبی کار کند، حقیقتی است که اعداد طبیعی می‌توانند به صورت بازگشتی تعریف شوند. یک مورد پایه وجود دارد که شامل کوچکترین عدد طبیعی است، و یک مورد بازگشتی وجود دارد که نشان می‌دهد چگونه می‌توان اعداد طبیعی بزرگتر را از اعداد کوچکتر ساخت.
                    \\
                    \\
                    \newpoint{Terms and Formulas:} همانطور که قبلاً اشاره شد، همه‌ی کلمات مجموعه $\Sigma^\infty$ معنایی ندارند. با توجه به اینکه هر ترکیبی از الفبا یک کلمه است، باید تفاوت‌هایی بین آنچه کلمات معنادار هستند و آنچه نیستند، وجود داشته باشد. ما دو نوع کلمه را به عنوان "عبارت‌ها و فرمول‌ها" در نظر خواهیم گرفت، به شرح زیر:
                    \begin{define}
                        \textit{If $\curveL$ is a language, a \textbf{term of $\curveL$} is a nonempty finite string $t$ of symbols from $\curveL$ such that either:}
                        \begin{enumerate}
                            \item $t$ is a variable, or
                            \item $t$ is a constant symbol, or 
                            \item $t:\equiv ft_1t_2t_3\dots t_n$, where $f$ is an $n$-ary function symbol of $\curveL$ and each of the $t_i$ is a term of $\curveL$.
                        \end{enumerate}
                    \end{define}
                    \begin{define}
                        \textit{If $\curveL$ is a language, a formula of $\curveL$ is a nonempty finite string of $\phi$ of symbols from $\curveL$ such that either:}
                        \begin{enumerate}
                            \item $\phi :\equiv = t_1t_2,$ where $t_1, t_2$ are terms of $\curveL$, or 
                            \item $\phi :\equiv R t_1t_2\dots t_n$ where $R$ is an $n$-ary relation symbol of $\curveL$ and $t_1, t_2, \dots , t_n$ are all terms of $\curveL$, or 
                            \item $\phi :\equiv (\neg \alpha)$ where $\alpha$ is a formula of $\curveL$, or
                            \item $\phi:\equiv (\alpha\lor \beta)$, where $\alpha$ and $\beta$ are formulas of $\curveL$, or 
                            \item $\phi :\equiv (\forall v)(\alpha)$, where $v$ is a variable and $\alpha$ is a formula of $\curveL$
                        \end{enumerate}
                    \end{define}
توجه کنید که پنج شرط تعریف، به دو گروه قابل تفکیک تقسیم می‌شوند. دو شرط اول ، یعنی فرمول‌های اتمی، به صورت صریح تعریف شده‌اند. سه شرط آخر، حالت بازگشتی هستند، نشان می‌دهد که اگر αα و ββ فرمول باشند، می‌توانند برای ساخت فرمول‌های پیچیده‌تری مانند (α∨β)(α∨β) یا (∀v)(α)(∀v)(α) استفاده شوند..
                    اکنون که مجموعه‌ی فرمول‌ها به صورت بازگشتی تعریف شده است، وقت آن است که در هنگام اثبات برای اثبات درستی یک پیشند در مورد هر فرمول از روش استقرایی استفاده کنیم. اثبات استقرایی از دو بخش تشکیل شده است: یک مورد پایه و یک مورد بازگشتی. اولین گام تأیید این عبارت برای هر فرمول اتمی است و سپس با استفاده از روش استقرایی، برای فرمول‌های بازگشتی از فرمول‌های اتمی آن را اثبات می‌کنیم. این روش به نام استقرایی بر روی پیچیدگی فرمول یا استقرایی بر روی ساختار فرمول شناخته می‌شود.
                    \\
                    \\
                    \newpoint{A First-order Language:} قبل از ورود به جملات، باید تعریف زبان درجه یک را بدانید. زبان درجه یک $\curveL$ به عنوان یک مجموعه نامتناهی از نمادها تعریف می‌شود که به دسته‌های زیر تقسیم می‌شوند:

                    \begin{itemize}
                        \item \textit{Parentheses:} $(,)$.
                        \item \textit{Connectives:} $\land, \lor, \neg$.
                        \item \textit{Quantifier:} $\forall, \exists$.
                        \item \textit{Variables:} one for each positive integer $n$ denoted: $v_n$ for $n$th number.
                        \item \textit{Equality:} $=$.
                        \item \textit{Constant:}ما می‌توانیم برای هر عدد مثبت یک نماد جدید داشته باشیم یا هر روش دیگری که در آن بین دو عدد تفکیک می‌کنیم (مانند استفاده از | برای 1، || برای 2 و غیره).
                        \item \textit{Functions:} برای هر عدد صحیح مثبت $n$، یک مجموعه‌ای از صفر یا چند نماد تابع $n$-آری وجود دارد.
                        \item \textit{Relation:} برای هر عدد صحیح مثبت $n$، یک مجموعه‌ای از صفر یا چند نماد رابطه $n$-آری وجود دارد.
                    \end{itemize}
                    \begin{callout}
                        داشتن آری $n$ به این معنی است که قصد نمایش تابعی از $n$ متغیر را داریم.
                    \end{callout}
                    توجه کنید که با تعریف چنین زبانی، می‌توانیم از فرآیند یافتن الگوریتم گرامر (الگوریتمی که بین عبارات بی‌معنی و عبارات پرمعنا تمایز قائل می‌شود) صرف نظر کنیم، زیرا تمام توابع ممکن و غیره را تعریف کرده‌ایم. به این شکل، تنها باید 

                    \newpoint{Sentences:} PM

در زبان $\curveL$، فرمول‌هایی وجود دارند که در آن‌ها بسیار علاقه‌مندیم. این فرمول‌ها جملات 4 هستند. فرمول‌هایی که می‌توانند در یک مدل ریاضی خاص، درست یا نادرست باشند.

در منطق ریاضی، جمله نوعی فرمول در زبان رسمی است که می‌تواند در یک مدل ریاضی خاص، درست یا نادرست باشد. \cite{Leary2019-ip}
                    \begin{define}
                        \textit{A sentence in a language $\curveL$ is a formula of $\curveL$ that contains no free variable.}
                    \end{define}
                    \newpoint{Structures: } PM

در تعریف هر زبانی با ثابت‌ها، توابع و غیره، ممکن است چندین روش برای تعریف ساختار وجود داشته باشد که با یکدیگر متضاد باشند. اما نکته این است که هیچ اولویتی وجود ندارد. بنابراین، بدون تعیین ساختار مورد نظر، بدون تصمیم‌گیری در مورد نحوه تفسیر نمادهای زبان، نمی‌توانیم درباره صحت یا نادرستی یک جمله صحبت کنیم. بنابراین، ما داریم
                    \begin{define}
                        \textit{Fix a language $\curveL$. An $\curveL$-Structure $\curveA$ is a nonempty set $A$m caled the \textbf{Universe of $\curveA$}, together with:}
                        \begin{enumerate}
                            \item \textit{For each constant symbol $c$ of $\curveL$, an element $c^\curveA$ of $A$}
                            \item \textit{For each $n$-ary function symbol $f$ of $\curveL$, a function $f^\curveA L A^n\rightarrow A$, and }
                            \item \textit{For each $n$-ary relation symbol $R$ of $\curveL$, and $n$-ary relation $R^\curveA$ on $A$.}
                        \end{enumerate}
                    \end{define}
                    \newpoint{Truth in a Structure:}با توجه به اینکه ما قوانین رسمی درباره چه چیزهایی یک زبان را تشکیل می‌دهند را می‌دانیم، می‌خواهیم نحوه ادغام دستور زبان و مفاهیم را بررسی کنیم. می‌خواهیم به سوال پاسخ دهیم که به چه معناست که یک فرمول $\curveL$ در یک ساختار $\curveA$ $\curveL$ صحیح است.
                    \\
                    \\
برای شروع فرآیند اتصال نمادها با ساختارها، ما تابع‌های تخصیص را معرفی می‌کنیم. این تابع‌های تخصیص مفاهیم تفسیر یک عبارت و یا یک فرمول در یک ساختار را شکلی‌سازی می‌کنند.
                    \begin{define}
                        \textit{if $\curveA$ us an $\curveL$-structure, a \textbf{variable assignment function into $\curveA$} is a function $s$ that assigns to each variable an element of the universe $A$. So a variable assignment function into $\curveA$ is any function with domain $V$ and codomain $A$.}
                    \end{define}
                    We will have occasion to want to fix the value of the assignment function $s$ for certain variables, then:
                    \begin{define}
                        \textit{If $s$ is a variable assignment function into $\curveA$ and $x$ is a variable and $a\in A$, then $s[x|a]$ is the variable assignment function into $\curveA$ defined as follows:}
                        \begin{equation}
                            s[x|a](v) = \left\{
                                \begin{matrix}
                                    s(v) & \text{if} \ v \ \text{is a variable other than }x\\
                                    a & \text{if} \ v \ \text{is the variable} x 
                                \end{matrix}\right.
                        \end{equation}
                    \end{define} 
                 ما تابع $s[x|a]$ را "تابع تغییر x" از تابع نسبت $s$ نامگذاری می‌کنیم. این دو تابع در واقع بسیار شبیه به هم هستند به جز این که متغیر $x$ به یک عنصر خاص از دانشگاه اختصاص داده شده است.
                    \\
                    \\
                   آنچه که بعدا انجام خواهیم داد، توسعه یک تابع نسبت متغیر به یک تابع نسبت عبارت $\bar{s}$ است. این تابع برای هر یک از عبارات زبان $\curveL$، یک عنصر از دانشگاه را اختصاص خواهد داد.
                    \begin{define}
                        \textit{فرض کنید $\curveA$ یک ساختار $\curveL$ باشد و $s$ یک تابع نسبت متغیر به $\curveA$ باشد. تابع $\bar{s}$ که تابع تخصیص عبارت تولید شده توسط $s$ نامیده می‌شود، تابعی است که دامنه آن شامل مجموعه عبارت‌های $\curveL$ و برد آن $A$ است که به صورت بازگشتی تعریف می‌شود به شرح زیر:}
                        \begin{enumerate}
                            \item \textit{If $t$ is a variable, $\bar s(t) = s(t)$}.
                            \item \textit{If $t$ is a constant symbol $c$, then $\bar s(t)= c^\curveA$}.
                            \item \textit{If $t:\equiv ft_1t_2\dots t_n$, then $\bar s(t) = f^\curveA (\bar s(t_1),\bar s(t_2),\dots,\bar s(t_n))$}.
                        \end{enumerate}
                    \end{define}
                    اگرچه عمدتا به حقیقت جملات علاقه‌مند خواهیم بود، ابتدا شرح حقیقت (یا رضایت‌بخشی) برای فرمول‌های دلخواه را نسبت به یک تابع نسبت، شرح خواهیم داد.
                    \begin{define}
                        \textit{Suppose that $\curveA$ us an $\curveL$-structure, $\phi$ is an $\curveL$-formula, and $s$: $V\rightarrow A$ is an assignment function. We will say that $\curveA$ satisfies $\phi$ with assignment $s$, and write $\curveA\vDash \phi[s]$, in the following circumstances:}
                        \begin{enumerate}
                            \item \textit{If $\phi :\equiv =t_1t_2$ and $\bar s(t_1)$ is the same element of the universe $A$ as $\bar s(t_2)$, or}
                            \item \textit{If $\phi :\equiv Rt_1\dots t_n$ and $(\bar s(t_1),\dots,\bar s(t_n))\in R^\curveA$, or}
                            \item \textit{If $\phi:\equiv (\neg \alpha)$ and $\curveA \not\vDash \alpha[s]$ (where $\not\vdash$ means "does not satisfy") or}
                            \item \textit{If $\phi:\equiv (\alpha\lor\beta)$ and $\curveA\vDash\alpha[s]$, or $\curveA \vDash\beta[s]$ (or both), or}
                            \item \textit{$\phi:\equiv(\forall x)(\alpha)$ and, for each element $a$ of $A$, $\curveA\vDash \alpha[s(x|a)]$}
                        \end{enumerate}
                    \end{define}
                    \begin{callout}
                        If $\Gamma$ is a set of $\curveL$-formulas, we say that $\curveA$ satisfies $\Gamma$ with assignment $s$, and write $\curveA\vDash \Gamma[s]$ if for each $\gamma\in\Gamma, \curveA\vDash \gamma[s]$
                    \end{callout}
                    \newpoint{Substitutions and Substitutability:} Suppose that you knew the sentence $\forall x \phi(x)$ was tru in particular structure $\curveA$. Then, if $c$ is a constant symbol in the language, you would certainly expect $\phi(c)$ to be true in $\curveA$ as well.
                    \\
                    \\
                    حال فرض کنید $\curveA\vDash \forall x\exists y\neg(x=y)$. در واقع، این جمله در هر ساختار $\curveA$ که $A$ حداقل دو عضو دارد، درست است. قوانین جایگزینی که در این بخش به آن‌ها پرداخته خواهیم کرد، برای کمک به جلوگیری از این مشکل طراحی شده‌اند، یعنی مشکل تلاش برای جایگذاری عبارت درون یک کوانتور که متغیر مربوط به آن در عبارت دخیل است.
                    \begin{define}
                        \textit{Suppose that $u$ is a term, $x$ is a variable, and $t$ is a term. We define the term $u_t^x$, and read $u$ with $x$ replaced by $t$ as:}
                        \begin{enumerate}
                            \item \textit{If $u$ is a variable not equal to x, then $u_t^x$ is $u$}
                            \item \textit{If $u$ is $x$, then $u_t^x$ is $t$}
                            \item \textit{If $u$ is a constant symbol then $u_t^x$ is $u$}
                            \item \textit{If $u:\equiv fu_1u_2\dots u_n$, where $f$ is a $n$-ary function symbol and the $u_i$ are terms, then:}
                            \begin{equation}
                                u_t^x \ \text{is} \ f(u_1)_t^x\dots(u_n)_t^x
                            \end{equation}
                        \end{enumerate}
                    \end{define}
بعد از تعریف اینکه جایگذاری یک عبارت به جای یک متغیر چه راه‌حلی است، حالا تعریف مفهوم جایگزین‌پذیری را بیان می‌کنیم:
                    \begin{define}
                        \textit{Suppose that $\phi$ is a $\curveL$-formula, $t$ is a term, and $x$ is a variable. We say that $t$ is substitutable for $x$ in $\phi$ if:}
                        \begin{enumerate}
                            \item \textit{$\phi$ is atomic, or}
                            \item \textit{$\phi :\equiv \neg(\alpha)$ and $t$ is substitutable for $x$ in $\alpha$, or}
                            \item \textit{$\phi:\equiv (\alpha\lor\beta)$ and $t$ is substitutable for $x$ in both $\alpha$ and $\beta$}
                            \item \textit{$\phi:\equiv (\forall y)(\alpha)$ and either}
                            \begin{enumerate}
                                \item \textit{$x$ is not free in $\phi$, or}
                                \item \textit{$y$ does not occur in $t$ and$t$ is substitable or $x$ in $\alpha$.}
                            \end{enumerate}
                        \end{enumerate}
                    \end{define}
                    \newpoint{Logical Implication: }در این بخش، سؤال "اگر من می‌دانم که این عبارت درست است، آیا لزوماً عبارت دیگری هم درست است؟" را به شکل رسمی شده بیان می‌کنیم.
                    \begin{define}
                        \textit{Suppose that $\Delta$ and $\Gamma$ are sets of $\curveL$-formulas. We will say that $\Delta$ logically implies $\Gamma$ and write $\Delta\vDash \Gamma$ if for every $\curveL$-structure $\curveA$, if $\curveA\vDash\Delta$, then $\curveA\vDash\Gamma$.}
                    \end{define}
                    This definition says that if $\Delta$ is true in $\curveA$, then $\Gamma$ is true in $\curveA$. Remember, for $\Delta$ to be true in $\curveA$, it must be the case that $\curveA\vDash\Delta[s]$ for every assignment function $s$.
                    \begin{define}
                        An $\curveL$-formula $\phi$ is said to be valid if $\emptyset\vDash\phi$, in other words, if $\phi$ is true in every $\curveL$-structure with every assignment function $s$. In this case we will write $\vDash\phi$.
                    \end{define}
                    \begin{callout}
                        For the double turnstyle symbol $\vDash$, if there is a structure on the left, $\curveA\vDash \sigma$, we are discussing truth in a single structure. On the other hand if there is a set of sentences on the left $\Gamma\vDash\sigma$, then we are discussing logical implication.
                    \end{callout}
                \subsection{Deductions}
در این بخش، سعی خواهیم کرد روش استنتاجی را به شکل رسمی بیان کنیم. در ریاضیات معمولاً برای هر عبارت درست (حداقل امید داریم) بر پایه وجود یک حکم استوار هستیم. یک حکم، یک دنباله از عبارات است که هریک از آن‌ها با ارجاع به عبارات قبلی توجیه می‌شود. این نقطه شروع کاملاً منطقی است و ما را به سختی اصلی خواهد کشاند که باید با آن برخورد کنیم، یعنی انتقال از فهم غیررسمی اینکه چه شروطی برای یک حکم وجود دارد به تعریف رسمی استنتاج.
                    \\
                    \\
                    اثبات‌هایی که در دوره ریاضی خود دیده‌اید، چند ویژگی خوب داشته‌اند. اولین این ویژگی‌ها این است که آسان برای پیروی هستند. این بدان معنا نیست که کشف یک اثبات آسان است، بلکه تنها به این معناست که اگر کسی یک اثبات را به شما نشان دهد، باید راهبرد آن را برای شما آسان باشد و بتوانید بفهمید چرا اثبات درست است. ویژگی دوم قابل تقدیر اثبات‌ها این است که وقتی شما چیزی را اثبات می‌کنید، می‌دانید که درست است! تعریف ما از استنتاج طراحی شده است تا مطمئن شود که استنتاج نیز به راحتی قابل بررسی باشد و حقیقت را حفظ کند.
                    \\
                    \\
سپس، محدودیت‌های زیر را بر روی اصول منطقی و قواعد استنتاج خود تحمیل خواهیم کرد:
                    \begin{enumerate}
                        \item یک الگوریتم وجود خواهد داشت که با دادن یک فرمول، تصمیم می‌گیرد که آیا آن فرمول یک اصل منطقی است یا نه.
                        \item خواهد بود الگوریتمی که با دادن یک مجموعه متناهی از فرمول‌های $\Gamma$ و یک فرمول $\theta$، تصمیم خواهد گرفت که آیا $(\Gamma,\theta)$ یک قانون استنباطی است یا نه.
                        \item برای هر قانون استنباطی $(\Gamma,\theta)$، $\Gamma$ یک مجموعه متناهی از فرمول‌ها خواهد بود.
                        \item هر اصل منطقی معتبر خواهد بود.
                        \item قوانین استنباطی ما، رابطه حقیقت را حفظ می‌کنند. به عبارت دیگر، برای هر قانون استنباطی $(\Gamma, \theta)\rightarrow \Gamma\vDash\theta$.
                    \end{enumerate}
                    ایده این است که برای بررسی اینکه آیا یک استنباطات خودخوانده شده به $\alpha$، واقعاً یک استنباط به $\alpha$ است، نباید هیچ دانش و بصیرتی لازم باشد. بررسی صحت یک استنباط به حدی ساده خواهد بود که می‌توان آن را در قالب یک برنامه رایانه‌ای پیاده‌سازی کرد.
                    \\
                    \\
ما با تعیین یک زبان $\curveL$ شروع می‌کنیم. همچنین فرض می‌کنیم که مجموعه‌ی ثابتی از $\curveL$-فرمول‌ها به نام $\Lambda$ به ما داده شده است، که به عنوان مجموعه‌ی اصول منطقی شناخته می‌شود، و همچنین یک مجموعه از جفت‌های مرتب $(\Gamma, \phi)$ به نام قوانین استنباط داده شده است. یک استنتاج باید یک دنباله یا لیست مرتب از فرمول‌های $\curveL$ با ویژگی‌های خاص باشد.
                    \begin{define}
                        \textit{فرض کنید $\Sigma$ مجموعه‌ای از فرمول‌های $\curveL$ و $D$ یک دنباله متناهی $(\phi_1, \dots, \phi_n)$ از فرمول‌های $\curveL$ باشد. می‌گوییم که $D$ یک استنتاج از $\Sigma$ است اگر برای هر عضو در $D$:}
                        \begin{enumerate}
                            \item $\phi_i\in\Lambda$ ($\phi_i$ is a logical axiom), or
                            \item $\phi_i\in\Sigma$ ($\phi_i$ is a nonlogical axiom), or
                            \item There is a rule of inference $(\Gamma,\phi_i)$ such that $\Gamma\subseteq\{\phi_1,\dots,\phi_{i-1}\}$
                        \end{enumerate}
                    \end{define}
                        اگر یک استنتاج از $\Sigma$ وجود داشته باشد که خط آخر آن فرمول $\phi$ باشد، آنگاه این را یک استنتاج از $\Sigma$ به $\phi$ می‌نامیم. و به شکل زیر نوشته می‌شود: $\Sigma\vdash\phi$.
                        \begin{callout}
حالا معنای کلمه مبرر را مشخص کرده‌ایم. در یک استنتاج، ما مجاز هستیم هر فرمول $\curveL$ ای را که دوست داریم بنویسیم، تا زمانی که آن فرمول یا یک اصل منطقی باشد و یا به صراحت در مجموعه‌ای از اصول غیرمنطقی $\Sigma$ لیست شود. هر فرمولی که در یک استنتاج بنویسیم که یک اصل نباشد، باید از فرمول‌های قبلی در استنتاج از طریق یک قانون استنباطی بوجود آید.
                        \end{callout}
                        \subsubsection{Logical Axioms} 
در این بخش، مجموعه‌ای از اصول منطقی $\Lambda$ برای $\curveL$ جمع‌آوری خواهیم کرد. این مجموعه از اصول، با اینکه نامحدود است، قابل تصمیم‌گیری است. به این معنی که یک الگوریتم وجود دارد که با دادن ورودی $x$، می‌تواند بگوید که $x\in\Lambda$ درست است یا نادرست.
                            \\
                            \\
                            \newpoint{Equality Axioms:} ما مسیری را طی کرده‌ایم که فرض می‌کنیم علامت برابری $=$ بخشی از خود زبان است. سه گروه از اصول برای این نماد طراحی شده‌اند. اولین گروه فقط می‌گوید هر شیء با خودش برابر است:
                            \begin{equation}
                                x = x
                            \end{equation}  
برای گروه دوم از اصول، فرض کنید $x_i$ و $y_i$ متغیرها باشند و $f$ یک نماد تابع $n$ آری باشد.
                            \begin{equation}
                                \left[(x_1 = y_1)\land (x_2 = y_2) \land\dots\land(x_n = y_n) \right] \rightarrow \left(f(x_1,x_2,\dots,x_n) = f(y_1,y_2,\dots,y_n)\right)
                            \end{equation}
فرض برای گروه سوم از اصول، مانند گروه دوم است، با این تفاوت که فرض می‌کنیم $R$ یک نماد رابطه $n$ آری باشد.
                            \begin{equation}
                                \left[(x_1 = y_1)\land (x_2 = y_2) \land\dots\land(x_n = y_n) \right] \rightarrow \left(R(x_1,x_2,\dots,x_n) = R(y_1,y_2,\dots,y_n)\right)
                            \end{equation}
                            \newpoint{Quantifier Axioms:} اصول کوانتورها به منظور اجازه دادن ورودی منطقی بسیار معقول در یک استنتاج طراحی شده‌اند. فرض کنید $\forall xP(x)$ را می‌دانیم. سپس، اگر $t$ هر عبارتی از زبان باشد، باید بتوانیم اعلام کنیم که $P(t)$ درست است. برای جلوگیری از برخی مشکلات، ما باید این الزام را داشته باشیم که عبارت $t$ برای متغیر $x$ قابل جایگذاری باشد..
                            \begin{align}
                                (\forall x\phi)\rightarrow \phi_t^x, \text{if } t \text{ is substitutable for } x \text{ in } \ \phi \\
                                \phi_t^x\rightarrow (\exists x\phi), \text{if } t \text{ is substitutable for } x \text{ in } \ \phi
                            \end{align}
در بسیاری از کتاب‌های منطق، اصل نخست به نام فرض همونی، و اصل دوم با نام عمومی سازی شناخته می‌شود.
                        \subsubsection{Rules of Inference:}
دو نوع قانون استنباطی وجود خواهد داشت، یکی مربوط به نتیجه‌گیری‌های پروپوزیشنال و دیگری مربوط به کوانتورها خواهد بود.
                            \\
                            \\
                            \newpoint{Propositional Consequence:}ما با یک زبان محدود $curveP$ کار می‌کنیم که فقط شامل مجموعه‌ای از متغیرهای پروپوزیشنال $A,B,C,\dots$ و اتصالی $\lor$ و $\neg$ است. توجه کنید که هیچ کوانتور، نماد رابطه، نماد تابع و نماد ثابتی وجود ندارد. هر متغیر پروپوزیشنال می‌تواند یکی از دو مقدار درستی T، F بگیرد. سپس ما می‌توانیم یک تابع $v$ برای اختصاص دادن مقدار درستی تعریف کرده و برای توسعه آن، می‌توانیم تعریف کنیم:
                            \begin{equation}
                                \bar v(\phi) = \left\{
                                \begin{matrix}
                                    v(\phi) &\text{if } \phi \text{ is a propositional variable}
                                    \\
                                    F & \text{if } \phi:\equiv(\neg\alpha) \text{ and } \bar v{\alpha} = T
                                    \\
                                    F & \text{if } \phi:\equiv(\alpha\lor\beta) \text{and } \bar v(\alpha) = \bar (\beta) = F
                                    \\
                                    T & \text{otherwise}
                                \end{matrix}
                                \right.
                            \end{equation}
برای بحث در مورد نتیجه‌گیری اصولی در منطق چندمرتبه، فرمول‌های خود را به دنیای منطق اصولی منتقل می‌کنیم و از ایده تراکیب در آن حوزه استفاده می‌کنیم. به طور خاص، با توجه به $\beta$، فرمول $\curveL$- مرتبه اول، یک روش وجود دارد که $\beta$ را به فرمول $\beta_P$ منطق اصولی منتقل می‌کند که با $\beta$ متناظر است.
                            \begin{enumerate}
                                \item همه زیرفرمول‌های $\beta$ به شکل $\forall x\alpha$ را که در دامنه یک کوانتور دیگر نیستند، پیدا کرده و با یک روش سیستماتیک با متغیرهای اصولی جایگزین کنید. این بدان معنی است که اگر $\forall yQ(y,c)$ دو بار در $\beta$ ظاهر شود، هر دو بار با یک حرف یکسان جایگزین می‌شود و زیرفرمول‌های متمایز با حروف متمایز جایگزین می‌شوند.
                                \item به تمامی فرمول‌های اتمی که باقی مانده‌اند پی ببرید و با متغیرهای اصولی جدید به صورت سیستماتیک جایگزین کنید.
                                \item در این نقطه، $\beta$ با یک فرمول اصولی $\beta_P$ جایگزین شده است.
                            \end{enumerate}
                            در حال حاضر تقریبا به نقطه‌ای رسیده‌ایم که می‌توانیم قاعده استنتاج اصولی خود را بیان کنیم. به یاد دارید که یک قاعده استنتاج، یک جفت مرتب $(\Gamma,\phi)$ است، که در آن $\Gamma$ مجموعه‌ای از فرمول‌های $\curveL$ و $\phi$ یک فرمول $\curveL$ است..
                            \begin{define}
                                \textit{فرض کنید $\Gamma_P$ مجموعه‌ای از فرمول‌های اصولی باشد و $\phi_P$ یک فرمول اصولی باشد. می‌گوییم که $\phi_P$ نتیجه‌گیری اصولی $\Gamma_P$ است، اگر هر تخصیص حقیقتی که باعث برقراری حقیقت فرمول‌های اصولی در $\Gamma_P$ شود، فرمول $\phi_P$ را نیز درست کند. توجه کنید که $\phi_P$ درستی‌ابی است اگر و تنها اگر $\phi_P$ نتیجه‌گیری اصولی $\emptyset$ باشد.}
                            \end{define}
                            Notice that if $\Gamma_P=\{\gamma_{1P},\dots,\gamma_{nP}\}$ is a nonempty finite set of propositional formulas and $\phi_P$ is a propositional formula, then $\phi_P$ is a propositional consequence of $\Gamma_P$ if and only if:
                            \begin{equation}
                                \left[
                                    \gamma_{1P}\land\dots\land \gamma_{nP} 
                                \right]\rightarrow \phi_P 
                            \end{equation}
                            is a tautology.
                            \begin{callout}
یک درستی‌ابی یا تحلیلی فرمولی است که در هر ساختار ممکن $\curveA$، درست است.
                            \end{callout}
                            حالا تعریف خود را گسترش می‌دهیم:
                            \begin{define}
                                فرض کنید $\Gamma$ یک مجموعه متناهی از فرمول‌های $\curveL$ و $\phi$ یک فرمول $\curveL$ باشد. می‌گوییم که $\phi$ نتیجه‌گیری اصولی $\Gamma$ است، اگر $\phi_P$ نتیجه‌گیری اصولی $\Gamma_P$ باشد، جایی که دو مجموعه آخر به دست آمده از روش تبدیل فرمول‌های منطق چندمرتبه به فرمول‌های اصولی هستند.
                            \end{define}
                            \begin{define}
                                اگر $\Gamma$ یک مجموعه متناهی از فرمول‌های $\curveL$ باشد، $\phi$ یک فرمول $\curveL$ باشد و $\phi$ نتیجه‌گیری اصولی $\Gamma$ باشد، آنگاه $(\Gamma,\phi)$ یک قاعده استنتاج از نوع (PC) است.
                            \end{define}
                            \newpoint{Quantifier Rules} فرض کنید، بدون هیچ فرض خاصی در مورد $x$، موفق به اثبات این شده‌اید که $x$ برابر با $a$ است. در این صورت، همچنین ثابت کرده‌اید $(\forall x)x$ برابر با $a$ است. اگر به آن را از پشت به جلو نگاه کنیم، داریم:

\begin{define}
فرض کنید متغیر $x$ در فرمول $\psi$ آزاد نباشد. در این صورت، هر دوی قواعد استنتاج زیر از نوع (QR) هستند:
                                \begin{align}
                                    \left(\left\{
                                        \psi\rightarrow\phi
                                    \right\}, \psi \rightarrow (\forall x \phi)\right)\\
                                    \left(\left\{
                                        \phi\rightarrow\psi
                                    \right\}, \exists x\phi \rightarrow \psi\right)
                                \end{align}
                            \end{define}
                \subsection{Soundness}
                    بطور کلی در ریاضیات، می‌خواهیم اطمینان حاصل کنیم که زمانی که چیزی ثابت شده است، به طور قطعی درست است. در این بخش، یک قضیه را ثابت خواهیم کرد که نشان می‌دهد سیستم منطقی که توسعه داده‌ایم این ویژگی بسیار مطلوب را داراست. این نتیجه با نام قضیه صحت شناسی شناخته می‌شود. همانطور که به یاد می‌آورید، الزاماتی که برای قواعد استنتاج خود تعیین کرده‌ایم عبارتند از:
                    \begin{enumerate}
                        \item یک الگوریتم وجود خواهد داشت که با در اختیار داشتن یک فرمول $\theta$، تصمیم بگیرد که آیا $\theta$ یک اصل منطقی است یا خیر.
                        \item یک الگوریتم وجود خواهد داشت که با در اختیار داشتن یک مجموعه متناهی فرمول‌ها $\Gamma$ و یک فرمول $\theta$، تصمیم بگیرد که آیا $(\Gamma, \theta)$ یک قاعده استنتاج است یا خیر.
                        \item برای هر قاعده استنتاج $(\Gamma, \theta)$، $\Gamma$ یک مجموعه متناهی از فرمول‌هاست.
                        \item هر اصل منطقی معتبر خواهد بود.
                        \item قواعد استنتاج ما درستی را حفظ خواهند کرد، به عبارت دیگر، برای هر قاعده استنتاج: $(\Gamma,\theta)$, $\Gamma\vDash \theta$.
                    \end{enumerate}
                    این الزامات دو هدف دارند: اولا، به ما امکان می‌دهد به صورت خودکار تصدیق کنیم که یک استنتاج پذیرفته شده در واقع یک استنتاج است، و دوما، پایه‌ای برای قضیه صحت شناسی فراهم می‌کنند. ایده پشت قضیه صحت شناسی بسیار ساده است. فرض کنید $\Sigma$ یک مجموعه از فرمول‌های $\curveL$ باشد و فرض کنید یک استنتاج برای $\phi$ از $\Sigma$ وجود داشته باشد. آنچه قضیه صحت شناسی به ما می‌گوید این است که در هر ساختار $\curveA$ که تمامی فرمول‌های $\Sigma$ را درست می‌کند، $\phi$ نیز درست است.
                    \begin{theorem}
                        If $\Sigma\vdash\phi$, then $\Sigma\vDash\phi$
                        \\
                        \\
                        \textit{proof.} Let $\thm_\Sigma = \{\phi | \Sigma \vdash\phi\}$ and $C =\{\phi | \Sigma \vDash \phi\}$. By showing that $\thm\subseteq C$ we prove the theorem. Notice that $C$ has the following characteristics:
                        \begin{enumerate}
                            \item $\Sigma \subseteq C$. If $\sigma\in\Sigma$ then certainly $\Sigma\vDash\sigma$.
                            \item $\Lambda\subseteq C$. As the logical axioms are valid, they are true in any structure. Thus $\Sigma\vDash\lambda$ for any logical axiom.
                            \item If $(\Gamma,\theta)$ is a rule of inference and $\Gamma\subseteq C$, then $\theta\in C$. Then because: $\curveA\vDash\Gamma$ and $\Gamma\vDash\theta$ we know that $\curveA\vDash\theta$.
                        \end{enumerate}
                        Since we have the following proposition and $C$ has these characteritics the prove is completed:
                        \textit{Fix sets of $\curveL$-formulas $\Sigma$ and $\Lambda$ and a collection of rules of inference. The set $\thm_\Sigma$ is the smallest set $C$ such:}
                        \begin{itemize}
                            \item $\Sigma\subseteq C$
                            \item $\Lambda\subseteq C$ 
                            \item If $(\Gamma,\theta)$ is a rule of inference and $\Gamma\subseteq C$, then $\theta\in C$.
                        \end{itemize}
                    \end{theorem}
            \section{Completeness}
قبل از شروع به بحث در مورد قضیه ناتمامی، ممکن است بهتر باشد که ابتدا مفهوم صحت کامل را درک کنیم. ما یک سیستم استنتاجی شامل اصول منطقی و قواعد استنتاج تعریف کرده‌ایم. قضیه صحت شناسی نشان داد که سیستم استنتاجی ما درستی را حفظ می‌کند؛ به عبارت دیگر، اگر یک استنتاج برای یک فرمول $\varphi$ از یک مجموعه فرمول‌های $\Sigma$ وجود داشته باشد، آنگاه $\varphi$ در هر مدلی از $\Sigma$ درست است. به صورت رسمی، این را به صورت زیر می‌نویسیم:
                $$\Sigma \vdash \varphi \implies \Sigma \models \varphi$$
                where $\vdash$ denotes provability in the deductive system, and $\models$ denotes semantic entailment.
                
                قضیه کاملیت، نتیجه‌ی اولین و مهم‌ترین قضیه‌ی این فصل است و به ما در مورد برعکس شدن قضیه صحت شناسی اطلاعات می‌دهد. به طور خاص، این قضیه بیان می‌کند که هر فرمول معنایی معتبر در سیستم استنتاجی ما قابل اثبات است. به عبارت دیگر، اگر یک فرمول در هر مدلی درست باشد، آنگاه با استفاده تنها از اصول منطقی و قواعد استنتاجی که تعریف کرده‌ایم، می‌توان آن را اثبات کرد. به صورت رسمی، این را به صورت زیر می‌نویسیم:

                \begin{equation} 
                    \Sigma \models \varphi \implies \Sigma \vdash \varphi
                \end{equation}
                
ترکیب شده با قضیه صحت شناسی، قضیه کاملیت به ما معادل زیر را می‌دهد:
                
                \begin{define}
                    A deductive system consisting of a collection of logical axioms $\Lambda$ and a collection of rules of inference is said to be complete if for every set of nonlogical axioms $\Sigma$ and every $\curveL$-formula $\phi$,
                    \begin{equation}
                        \text{If } \Sigma\vDash\phi \text{, then } \Sigma\vdash\phi
                    \end{equation}
                \end{define}

این معادله به ما تضمین می‌کند که اگر سیستم استنتاجی ما به ما اجازه می‌دهد یک بیانیه را اثبات کنیم، آنگاه آن بیانیه درست است، و برعکس، اگر یک بیانیه در هر مدلی درست باشد، آنگاه می‌توان با استفاده از سیستم استنتاجی ما آن را اثبات کرد.
                
با این حال، مهم است به یاد داشت که قضیه کاملیت فقط برای فرمول‌هایی که در هر مدل ممکن درست هستند، صدق می‌کند. این قضیه به فرمول‌هایی که در برخی از مدل‌ها درست هستند و در برخی دیگر نیستند، نمی‌تواند تعمیم داده شود. علاوه بر این، این قضیه به طور خاص برای منطق چندجمله‌ای است و ممکن است برای سایر سیستم‌های منطقی صدق نکند.
            \section{Incompleteness Theorems}
                \subsection{Mathematics}
                    \subsubsection{Prespective}
در بخش قبلی، در مورد قضیه کاملیت صحبت کردیم که به بررسی این موضوع پرداخت که سیستم استنتاجی ما صحیح و کامل است. برای مجموعه ای از اصول منطقی که تعریف کرده‌ایم، هر فرمول $\phi$ که از یک مجموعه اصول غیرمنطقی $\Sigma$ نتیجه می‌شود، در تمامی مدل‌های $\Sigma$ با هر تابع تخصیص متغیر، درست است (صحیح بودن)، و علاوه بر این هر فرمول $\phi$ که در تمامی مدل‌های $\Sigma$ با هر تابع تخصیص متغیر، درست باشد، از $\Sigma$ نتیجه می‌شود (کامل بودن)-- به عبارت دیگر، ما می‌توانیم هر بیانیه درست را در سیستم اثبات کنیم. بنابراین، سیستم استنتاجی ما بهترین حالت را دارد.\cite{Hajek2007-fq}
                        \\
                        \\
در این بخش، قصد داریم نشان دهیم چگونه می‌توان قضیه ناتمامی را اثبات کرد. از آنجایی که قبلاً نشان داده شده است منطق چندجمله‌ای صحیح و کامل است، بنابراین ناتمامی و کاملیت باید برای زبان‌های مختلف اثبات شوند و باید جداگانه بررسی شود که آیا می‌توان آنها را در فیزیک استفاده کرد یا خیر. اما در بخش‌های قبلی، مفاهیم لازم را برای کسی که می‌خواهد در مورد کاملیت (یا ناتمامی) یک تئوری فیزیکی صحبت کند، توضیح دادیم.
                        \\
                        \\
بعداً، ما در مورد این که چرا به نظر ما این ممکن است صحبت خواهیم کرد.
                    \subsubsection{Basic Proof from the Theory of Algorithm\cite{Uspensky1995-sm}}
ما فرض می‌کنیم یک زیرمجموعه $T$ از زبان $L$ به ما داده شده است که با نام مجموعه \textit{بیانیات درست} شناخته می‌شود. این مجموعه فقط باید حاوی بیانیه‌هایی باشد که در زبان به عنوان درست تفسیر می‌شوند. با فرض این زیرمجموعه، بخشی را که باید در مورد درست یا نادرست بودن بیانیه‌ها در نظر بگیریم، رد می‌کنیم. برای ما، یک زبان به طور کامل تعریف شده است اگر جفت بنیادین زیر داده شود:
                        \begin{define}
                            \textit{Given a language $L$ and the subset of true statements $T$, we call $\left<L,T\right>$ a fundamental pair.}
                        \end{define}
                        \begin{define}
                            \textit{\textbf{Unprovable} means not provable, and provable means there exist a proof for such statement.}
                        \end{define}
در بخش‌های قبلی، مفاهیم اثبات، استنتاج و استدلال تعریف شده است. به منظور ساده‌سازی، یک سیستم استنتاجی را در نظر بگیرید که شامل یک الفبایی به نام \textit{الفبای اثبات} باشد و با $P$ نشان داده می‌شود. سپس مجموعه $P^\infty$ شامل همه کلماتی است که الفبا می‌تواند داشته باشد، اما محدودیت‌های گرامر، قواعد استنتاج، استدلال و غیره، ما را با مجموعه ای از اثبات‌ها به نام $\bar{P}$ فراهم می‌کنند. به عنوان یادآوری، سیستم استنتاجی ما یک بیانیه را اثبات می‌کند و فقط یک بیانیه برای هر اثبات وجود دارد. بنابراین، با در نظر گرفتن یک زبان $\curveL_x$ و یک زبان اثبات $P$، ما به وجود الگوریتمی نیاز داریم که یک اثبات را از $P$ به یک بیانیه درست از $\curveL_x$ تبدیل کند. بنابراین، یک سیستم استنتاجی (به ساده‌ترین شکل ممکن، اما همچنان می‌توانید به استدلال معرفی شده در بخش‌های قبلی فکر کنید) شامل $P$، $\bar{P}$ و $\delta$ است، که $\delta$ یک تابع است که اثبات‌ها را از $P$ به بیانیه‌های درست از $\curveL_x$ تبدیل می‌کند.
                        \begin{define}
                            \textit{We would call $\left<P,\bar P, \delta\right> a deductive system.$}
                        \end{define}
بنابراین تعریف نهایی ما به شکل زیر است:
                        \begin{enumerate}
                            \item We have a language $\curveL$ and an alphabet $P$ for the proofs. 
                            \item In the set $P^\infty$ we are given a subset $\bar P$, whose elements are called proofs. We futher assume:
                            \begin{enumerate}
                                \item We shall allow different concepts of proofs, different proof subsets of $P^\infty$. We shall also allow the alphabet $P$ to vary.
                                \item There has to be an effective method or algorithm to check if a given word in $P$ is a proof and to show what stetement it proves.
                            \end{enumerate}
                            \item We have a function $\delta$ (to determine what is being proved) whose domain of definition $\Delta$ satisfies $\bar P\subseteq \Delta\subseteq P^\infty$, and whose range of valurs is in $L^\infty$. We assue that we have an algorithm which computes this function. Therefore we might say that the proof $p$ in $\bar P$ is the proof of $\delta(p)$ in the language $\curveL$.
                        \end{enumerate}
بنابراین، ما می‌توانیم حالا به آسانی کاملیت و سازگاری را تعریف کنیم:
                        \begin{define}
                            \textit{We say the deductive system $\left<P,\bar P, \delta\right>$ is consistent relative to the fundamental pair $\left<\curveL,T\right>$ if we have $\delta(\bar P) \subseteq T$. In other words our deductive system would not prove contradictory statements.}
                        \end{define}
                        \begin{define}
                            \textit{We say the deductive system $\left<P,\bar P, \delta\right>$ is complete relative to the fundamental pair $\left<\curveL,T\right>$ if we have $T \subseteq \delta(\bar P)$. In other words our deductive system would prove every true statements within the language.}
                        \end{define}
شرایط عدم وجود یک سیستم استنتاجی کامل و سازگار می‌توانند به راحتی با استفاده از نظریه الگوریتم‌ها تعریف شوند. الگوریتم، مجموعه‌ای از دستورالعمل‌هاست که با دریافت یک ورودی، اگر چنین خروجی وجود دارد، ما را قادر می‌سازد تا یک خروجی را بدست آوریم و در صورت عدم وجود خروجی برای ورودی خاص ما هیچ چیز بدست نیاوریم. برای هدف ما، در نظر بگیرید که هر ورودی و خروجی یک کلمه باشد، بنابراین الفبای $I$ برای ورودی‌ها و الفبای $O$ برای خروجی‌ها وجود دارد.
                        \\
                        \\
به عبارت دیگر، برای کار با الگوریتم‌هایی که جفت یا دنباله‌ای از کلمات را پردازش می‌کنند، ابتدا باید این جفت‌ها یا دنباله‌ها را با استفاده از الفبای جدیدی به صورت یک کلمه در الگوریتم فشرده کنیم.
                        \\
                        \\
فرض کنید $\star$ نمایانگر یک حرف جدید وجود ندارد در الفبای $\curveL$ باشد. بنابراین، $\curveL_\star$ یک الفبای جدید است. ما از این حرف برای نشان دادن جداسازی بین کلمات در الفبایمان استفاده خواهیم کرد.
                        \\
                        \\
                        Now we must define some terms:
                        \begin{define}
                            \textit{\textbf{Domain of Applicability:} مجموعه همه ورودی‌هایی که توسط یک الگوریتم خاص قابل پردازش است، دامنه اعمال الگوریتم نامیده می‌شود.}
                        \end{define}
                        \begin{define}
                            \textit{The algorithm $A$ computes the function $f$, if we have $A(x)\sim f(x)$ for all $x$}
                        \end{define}
                        \begin{callout}
توجه کنید که $\sim$ علامت برابری شرطی است، به این معنی که هر دو طرف آن زمانی برابرند که تعریف شده باشند و هر دو طرف آن در صورتی که $x$ یکسان باشد، تعریف نشده باشند.
                        \end{callout}
                        \begin{define}
                            \textit{تابعی که توسط یک الگوریتم خاص قابل محاسبه است، تابع محاسب‌پذیر (computable function) نامیده می‌شود.}
                        \end{define}
                        \begin{define}
                            \textit{\textbf{Decidable:} A subset $S$ of a set $A$ is said to be decidable if there exsits and algorithm which determines whether or not an element of $A$ is in $S$.}
                        \end{define}
ما به یک مجموعه از تمامی اثبات‌ها نیاز داریم که یک مجموعه قابل تصمیم از تمامی کلمات موجود در الفبای اثبات باشد
                        \begin{lemma}
                            For any subset $X$, the set $\emptyset$ and $X$ are decidable relative to $X$.
                            \proof We provide an algorithm for the subset $X$ of $X$ that always returnes true. And for the $\emptyset$ always false.
                        \end{lemma}
                        \begin{theorem}
                            If $T$ is an enumerable set, then one can find a complete and consistent deductive system for the fundamental pair $\left<\curveL, T\right>$
                            \proof if $T=\emptyset$ the deductive system $\left<P, \bar P, \delta\right>$ where $\bar P = \emptyset$ and $delta(p)=\emptyset$ is consistent and complete relative to $T$.
                            \\
                            Otherwise if $T \not = \emptyset$ since it is enumerable (there exists an algorithm that maps the natural numbers to the set.), we would have the algoritm $\tau$ that enumerates $T$. We would later prove that the set of all proofs is enumerable therefore by choosing $\delta = \tau$ we would have a consistent and complete deductive system.
                        \end{theorem}
                        \begin{lemma}
                            A decidable subset of an enumerable set is enumerable.
                            \proof if $f$ is the enumerating function of a set $A$. If $S\subseteq A$ is empty set. Then it is enumerable by definition. If not there exists $s\in S$ thus we enumerate the set by the following function:
                            \begin{equation}
                                g(n)=\left\{\begin{matrix}
                                    f(n) & \text{if } f(n) \in S\\
                                    s & \text{otherwise}\end{matrix}\right.
                            \end{equation}
                        \end{lemma}
                        \begin{lemma}
                            A subset $S$ of an enumerable set $X$ is decidable relative to $X$ is decidable relative to $X$ if and only if both $S$ and its complement $X\backslash S$ are enumerable.
                            \proof If $S$ is decidable, then so is $X\backslash S$, and so both $S$ and $X\backslash S$ are enumerable by Lemma 2. Conversely, suppose that both $S$ and $X\backslash S$ are enumerable. If either one is empty then, $S$ is decidable, by Lemma 1. Suppose both $S$ and its complement are nonempty sets, in which case they are enumerated by some computable functions $f$ and $g$, respectively. Then, for deciding $x$ we need only compute $f(0),g(0),f(1),g(1),\dots$ until we encounter $x$.
                        \end{lemma}
                        \begin{theorem}
                            The set of all proofs is enumerable.
                            \proof Fir any enumerable alphabet $L$ we have:
                            \begin{equation}
                                L^\infty = \left\{x | x\in L\right\} \cup \left\{xy|x,y\in L\right\}\dots
                            \end{equation}
بنابراین، چون الفبای $P$ شمارا است، الگوریتمی وجود دارد که تمام کلمات را پیمایش می‌کند، در نتیجه $P^\infty$ و زیرمجموعه‌های آن شمارا هستند.
                        \end{theorem}
                        \begin{lemma}
                            Suppose that $R$ is an enumerable set and $f$ is a computable function which is defined on all elements of $R$. Then $f(R)$ is an enumerable set.
                            \proof If $R$ is empty, then so is $f(R)$ therefore enumerable by definition. If $R$ is enumerated by a function $\rho(x)$ then $f(R)$ is enumerated by $\eta(x) = f(\rho(x))$.
                        \end{lemma}
                        \begin{theorem}
مجموعه تمامی کلمات قابل تأیید (برای یک سیستم استنتاجی) شماراست. 

اثبات: فرض کنید $Q$ مجموعه‌ای از تمامی کلمات قابل تأیید برای سیستم استنتاجی $\left<P, \bar P, \delta\right>$ باشد، به طور واضح $Q = \delta(\bar P)$ است. اما طبق قضیه ۴، $P$ شماراست لذا به لم ۴، $Q$ نیز شماراست.
                        \end{theorem}
بنابراین، اگر $T$ یک مجموعه غیرشمارا باشد، پیدا کردن یک سیستم استنتاجی کامل و سازگار برای جفت $\left< \curveL,T\right>$ غیرممکن است. زیرا مجموعه $Q$ از کلمات قابل تأیید برای هر سیستم استنتاجی سازگار به عنوان یک زیرمجموعه مناسب از $T$ به دست می‌آید، و بنابراین یک عنصر در تکمیل $T\backslash Q$ وجود خواهد داشت. چنین عنصری یک بیانیه درست ولی ناتوان‌پذیر است!
                        \\
                        \\
قضیه‌های ۳ و ۵ در کنار هم شرطی برای جفت بنیادی ارائه می‌دهند که شرط لازم و کافی برای وجود یک سیستم استنتاجی کامل و سازگار برای آن جفت است. این شرط، شمارپذیر بودن مجموعه تمامی حقایق است.
                        \\
                        \\
از یک زبان "غنی" یا "با قابلیت ابراز بالا" انتظار می‌رود که مجموعه تمامی حقایق در آن بسیار پیچیده باشد و به همین دلیل برای آن نمی‌توان مجموعه‌ای شمارا وجود داشته باشد. به همین دلیل، برای چنین زبانی سیستم استنتاجی کامل و سازگاری وجود ندارد. با این حال، شرط قضایای ۳ و ۵ خیلی محبوب نیست زیرا بررسی تمامی مجموعه $T$ بسیار دشوار است.


                \subsection{Philosophy}
در این بخش، سعی خواهیم کرد قضایای ناتمامیت را در خارج از زبان ریاضیات رسمی توضیح دهیم. سپس به بحث در مورد یک جهان ریاضی پرداخته خواهیم شد.
                        \subsection{Introduction}
قضایای ناتمامیت گودل، نتایجی نوآورانه در منطق امروزی بودند که باعث تحول در درک ریاضیات و منطق شدند و دارای پیامدهای مهمی برای فلسفه ریاضیات هستند. در حالی که برخی از فیلسوفان سعی کرده‌اند از این نتایج در سایر حوزه‌های فلسفه استفاده کنند، صحت بسیاری از این کاربردها مشکل است و مورد بحث قرار می‌گیرد.
                            \\
                            \\
برای درک قضایای گودل، لازم است ایده‌های حیاتی مرتبط با آن‌ها را شفاف کنیم، شامل "سیستم فرمال"، "سازگاری" و "تکمیل". به طور اساسی، یک سیستم فرمال نشان دهنده مجموعه‌ای از اصول مزود به قواعد استنتاج است که امکان تولید قضایای جدید برای آن‌ها را فراهم می‌آورد. اصول باید متناهی یا حداقل تصمیم‌پذیر باشند، به این معنی که الگوریتمی می‌تواند تشخیص دهد که یک عبارت داده شده پیش‌فرض است یا خیر. اگر این معیار برآورده شود، نظریه به عنوان "قابل اختصار با الگوریتم" شناخته می‌شود. قواعد استنتاج یک سیستم فرمال نیز عملیات قابل اجرا هستند که امکان اطمینان از اینکه یک قاعده استنتاج به درستی بکار گرفته شده است را فراهم می‌آورند. به عبارت دیگر، همیشه امکان تعیین این وجود دارد که آیا هر توالی متناهی داده شده از فرمول‌ها، با در نظر گرفتن اصول و قواعد استنتاج، یک استنتاج واقعی در سیستم است یا خیر.
                            \\
                            \\
یک سیستم فرمال در صورتی تکمیل است که هر عبارت زبان سیستم می‌تواند در سیستم اثبات شود یا منفی آن قابل ثابت باشد. یک سیستم فرمال سازگار است زمانی که هیچ عبارتی وجود ندارد که خود عبارت و منفی آن هر دو در سیستم اثبات شوند. در این زمینه، تنها سیستم‌های سازگار مورد علاقه هستند زیرا یک سیستم فرمال ناسازگار، هر عبارتی را به راحتی قابل اثبات می‌کند و در نتیجه سیستمی کاملاً تکمیل شده بی‌معنی خواهد بود.
                            \\
                            \\
گودل دو قضیه ناتمامیت مجزا ولی به هم پیوسته کشف کرد، که به طور معمول به عنوان قضایای ناتمامیت اول و دوم شناخته می‌شوند. اگرچه "قضیه گودل" ممکن است به هر دو به صورت همزمان اشاره کند، اما به طور جداگانه به قضیه ناتمامیت اول اشاره می‌کند. با تغییری که در سال ۱۹۳۶ توسط جی بارکلی راسر وارد شد، قضیه اول به طور خلاصه به شرح زیر قابل بیان است:
                            \begin{qt}
هر سیستم فرمال منطقی $F$ که در آن می‌توان یک حداقل مقدار از حساب دبیرستانی را انجام داد، ناتمام است به این معنی که تنها برای برخی از عبارات زبان $F$ می‌توان در $F$ ثابت کرد یا رد کرد. با دیگر کلام، همواره عباراتی در زبان $F$ وجود دارند که نه قابل اثبات در $F$ هستند و نه قابل تکذیب در $F$.
                            \end{qt}
قضیه گودل، نه تنها وجود چنین عباراتی را ادعا می‌کند، بلکه با استفاده از روش اثبات گودل، یک جمله خاص را به طور صریح ارائه می‌دهد که در سیستم فرمال $F$ نه قابل اثبات است و نه قابل تکذیب. این جمله "تصمیم‌ناپذیر" از یک مشخصه $F$ به طور خودکار تولید می‌شود و یک عبارت نسبتاً ساده از نظریه اعداد است، یعنی یک جمله تماماً عددی بدون تعریف‌های مرتبط با زبان $F$.
                            \\
                            \\
 یک اشتباه شایع در مورد قضیه ناتمامیت اول گودل، تفسیر آن به عنوان نشان دادن وجود حقایقی است که قابل اثبات نیستند. با این حال، این اشتباه است چرا که قضیه ناتمامیت با پذیرش اثبات‌پذیری در یک سیستم فرمال خاص سر و کار دارد، نه اثبات‌پذیری به معنای مطلق. برای هر عبارت $A$ که در یک سیستم فرمال داده شده قابل اثبات نیست، سیستم‌های فرمال دیگری وجود دارند که $A$ در آن‌ها قابل اثبات است (با در نظر گرفتن $A$ به عنوان اصل). از سوی دیگر، سیستم اصول استاندارد نظریه مجموعه زرملو-فرانکل (نشان داده شده با $ZF$ یا $ZFC$ با اصل انتخاب) قدرتمند برای استنتاج تمام ریاضیات معمولی است. با این حال، به دلیل قضیه ناتمامیت اول گودل، مسائل ریاضی معناداری وجود دارند که حتی در $ZFC$ قابل اثبات نیستند. بنابراین ثابت کردن آن‌ها نیازمند یک سیستم فرمال است که از روش‌های $ZFC$ فراتر رفته باشد. به همین دلیل، این حقایق با استفاده از روش‌ها و اصول ریاضی معمولی امروز قابل اثبات نیستند و نمی‌توان به صورتی تأیید شده و بدون مشکل به عنوان حقایقی محسوب شوند.\cite{Hajek2007-fq}
                            \\
                            \\
قضیه ناتمامیت دوم گودل، مربوط به محدودیت‌های اثبات‌های سازگاری است و به طور خلاصه می‌توان آن را به شرح زیر بیان کرد:
                            \begin{qt}
برای هر سیستم فرمال منطقیF که در آن می‌توان یک حداقل مقدار از حساب دبیرستانی را انجام داد و خودش را سازگار می‌داند، سازگاری F نمی‌تواند با استفاده از ابزارهای همان سیستم فرمال، یعنی با استفاده از قواعد استنتاج و اصولی که در آن فراهم شده است، در F اثبات شود. 
                            \end{qt}
قضیه ناتمامیت دوم، نسخه‌ی کمی پیچیده‌تری از سیستم فرمال $F$ را مورد نیاز دارد که حساب بیشتری را شامل می‌شود نسبت به آن چیزی که برای قضیه ناتمامیت اول لازم است و تحت فرض‌های ضعیف‌تری برقرار است. مهم است بدانیم که همانند قضیه ناتمامیت اول، قضیه ناتمامیت دوم نیز تنها در مورد اثبات فرمالی یا قابل استنتاج در یک سیستم فرمال خاص (در این حالت $F$ خود) صدق می‌کند. این قضیه هیچ چیز درباره اینکه آیا عبارت "T سازگار است" می‌تواند با استدلالی که توسط ریاضی‌دانان به عنوان قطعی و قابل پذیرش در نظر گرفته شده باشد در برابر یک تئوری خاص $T$ که شرایط قضیه را برآورده می‌کند، ثابت شود. برای بسیاری از تئوری‌ها، چنین اثباتی ممکن است وجود داشته باشد.
                            
                        \subsection{Incompleteness Theorems and The Church-Turing Thesis}
اولین‌بار که گودل، عدم قطعیت یک تئوری را نشان داد، در مورد یک تئوری خاص به نام $P$ بود که یک تغییراتی از سیستم راسل با نام $PM$ (پرینسیپیا ماتماتیکا) است. این سیستم شامل تمام گسترش‌هایی از $P$ با همان زبانی است که مجموعه اصول آن محدود به مجموعه‌ای محاسباتی بازگشتی است. گرچه گودل اثبات نکرد، اما اشاره کرد که اثبات می‌تواند به سیستم‌های اصولی معمول نظریه مجموعه مانند $ZFC$ نیز اعمال شود. در آن زمان، نامشخص بود که نتیجه چقدر کلی است، با این حال، گودل نتیجه بسیار کلی‌ای را داشت.
                            \\
                            \\
آنچه هنوز نامشخص بود، تحلیل مفهوم ذهنی تصمیم‌پذیری بود که در شخصیت‌بخشیدن به مفهوم یک سیستم فرمال دلخواه لازم است. ریاضی‌دانان و منطق‌دانان از مفهوم ذهنی یک روش تصمیم‌گیری از دوران باستان استفاده کرده‌اند. با این حال، برای نتایج کلی مانند قضیه‌های عدم قطعیت کلی یا نتایج ناتصمیم‌پذیری، تشریح دقیق ریاضی این مفهوم لازم است. به جای مجموعه‌ها یا خصوصیات صورت پذیر، اغلب توابع یا عملیات مؤثر یا محاسباتی مدنظر هستند، اما این دو طرف یک سکه هستند.
                            \\
                            \\
گودل، چرچ و تورینگ به‌طور مستقل، تعاریف دقیق مختلف ریاضی برای توابع محاسباتی و مجموعه‌های قابل تصمیم را ارائه دادند که بعداً معادل یکدیگر بودند. تحلیل مفهومی دقیق تورینگ که با استفاده از ماشین‌های محاسباتی خیالی و انتزاعی به نام "ماشین‌های تورینگ" انجام شده بود، به عنوان یکی از مهمترین روش‌ها تأکید شد، همانطور که خود گودل نیز آن را برجسته کرد. مفهوم ذهنی و برخی از این تشریح‌های ریاضی به عنوان "فرضیه چرچ-تورینگ" شناخته می‌شوند. به دلیل دلایل تاریخی، مفهوم "تابع بازگشتی" در ادبیات منطقی حاکم بوده است، بنابراین مجموعه‌های قابل تصمیم به عنوان "مجموعه‌های بازگشتی" شناخته می‌شوند.
                            \\
                            \\
برای درک درست نتایج عدم قطعیت و ناتصمیم‌پذیری، تفاوت بین دو مفهوم کلیدی درباره مجموعه‌ها بسیار مهم است. اول، ممکن است یک روش مکانیکی در دسترس باشد که تصمیم بگیرد که آیا هر عدد داده شده در مجموعه قرار دارد یا نه، در این صورت، مجموعه به عنوان "قابل تصمیم" یا "بازگشتی" شناخته می‌شود. دوم، ممکن است یک روش مکانیکی وجود داشته باشد که عناصر مجموعه را، یکی یکی و به ترتیب اعداد، بیابد یا لیست کند. در این حالت، مجموعه به عنوان "قابل شمارش بازگشتی" (R.E.) یا به عبارتی قابلیت تولید مؤثر دارد یا "نیمه-قابل تصمیم" است. یک نتیجه بنیادی از نظریه محاسبه‌پذیری این است که مجموعه های نیمه قابل تصمیم، می‌توانند به صورت مؤثر تولید شوند، اما قابل تصمیم نیستند. در حقیقت این جانشین اصلی برای قضیه ناتمامیت اول در سطح انتزاعی است. با این حال، اگر یک مجموعه و همراه آن تکمیلش، قابل شمارش بازگشتی باشد، آنگاه مجموعه قابل تصمیم است یعنی دارای تصمیم‌پذیری است.\cite{sep-church-turing}
            \section{Nature aa a Mathematical Entity}
قضیه عدم کاملیت به همراه قضیه محاسبه‌پذیری (فرضیه چرچ-تورینگ) حد روی بُعد ریاضی است. با توجه به وابستگی به اصول انتخابی، این دو به موجودات ناتائیده یا ناتوان در ریاضیات و محاسبات اشاره دارند. مسئله‌هایی مانند تبعیض الگوریتمی و حریم خصوصی تنها چند نمونه از مواردی هستند که نیازمند بررسی دقیق هستند. بسیاری از چیزهایی را می‌توان از این دو قضیه آموخت و به درک بهتری از جهان به طور کلی دست پیدا کرد. اما تفاوت جزئی بین فیزیکدانان و ریاضی‌دانان وجود دارد.
                \\
                \\
فرضیه‌ای که بر اساس آن جهان واقعیت یک انتزاع است، اصلاً جدید نیست. در واقع، این فرضیه به همان قدمتی که پلاتو دارد برمی‌گردد!
                \subsection{Platonism in Mathematics}
                    \subsubsection{What is Mathematical Platonism}
فرگشت ریاضی می‌تواند به شرح زیر تعریف شود که شامل سه پایه زیر است:
                        \begin{enumerate}
                            \item  \textbf{Existence: }There are mathematical objects.
                            \item  \textbf{Abstractness:} Mathematical objects are abstract.
                            \item  \textbf{Independence:} Mathematical objects are independent of intelligent agents and their language, thought, and practices.
                        \end{enumerate}
                        فرگشت، نه تنها مختص به ریاضیات است، بلکه هر سیستم باوری را شامل می‌شود که بر پایه سه حقیقت فوق الذکر مبتنی است. دو حقیقت اولیه نسبتاً مستقیم هستند: وجود می‌تواند به عبارت "$\exists xMx$" بیان شود، جایی که "$Mx$" گزاره "$x$ یک شیء ریاضی" را نشان می‌دهد، که به اشیاء مطالعه شده در ریاضیات خالص مانند اعداد، مجموعه‌ها و توابع اشاره دارد. انتزاعیت بیانگر آن است که تمام اشیاء ریاضیاتی، غیرفضایی-زمانی هستند و به همین دلیل عاملیت ندارند. \cite{sep-platonism-mathematics}
                        \\
                        \\
استقلال، با این حال، کمی پیچیده‌تر است. درک معمولی آن به مفهومی است که حتی اگر عامل هوشمندی وجود نداشته باشد و یا زبان، فکر یا شیوه‌های آن‌ها متفاوت باشد، اشیاء ریاضیاتی همچنان وجود دارند. با این حال، این تفسیر ممکن است برای به دست آوردن کامل مفهوم استقلال کافی نباشد. در حال حاضر، مفهوم استقلال به تا حدودی مبهم باقی مانده است.
                    \subsubsection{Historical Remarks}
فرگشت باید از دیدگاه تاریخی پلاتو متمایز شود. در بحث‌های معاصر درباره فرگشت، تقریباً کسی برای تفسیر قوی و اصیل نظریه پلاتو دفاع نمی‌کند. هر چند که عبارت "فرگشت" از نظریه پلاتو درباره فرم‌های انتزاعی و جاودانه الهام گرفته شده است، اما تعریف فعلی فرگشت به طور مستقل از الهام تاریخی اولیه است..\cite{sep-platonism-mathematics}
                        \\
                        \\
مهم است به این نکته توجه کنیم که فرگشت به طور کامل یک دیدگاه متافیزیکی است و از دیدگاه‌های دیگری با محتوای اپیستمولوژی قابل تمایز است. توصیفات گذشته از فرگشت شامل بیانیه‌های اپیستمولوژی قوی بودند که الحاق می‌کردند درکی فوری یا فهمی فوری از حوزه اشیاء انتزاعی را. با این حال، اکنون رایج‌تر است که عبارت "فرگشت" را برای دیدگاه متافیزیکی خالص استفاده کنیم که قبلاً شرح داده شد. بسیاری از فیلسوفانی که این گونه از فرگشت را دفاع می‌کنند، ادعاهای اپیستمولوژیکی اضافی را که شامل کوئین و آنانی هستند که به دلیل برجستگی برخی قضایا، به دفاع از فرگشت ریاضیاتی کمک می‌کند، رد می‌کنند.
                        \\
                        \\
در نهایت، تعریف "فرگشت ریاضی" ادعایی را که تمام حقایق ریاضیات خالص ضروری هستند، به شکل سنتی توسط بیشتر فرگشت‌باوران مطرح می‌شود، از بین می‌برد. این استثنا با توجه به حقیقت استدلال برخی فیلسوفان فرگشت‌باور، مانند کوئین و دفاع‌کنندگان از استدلال اپیستمولوژیکی گفته شده، استوار است که ادعای قابلیت ازادی اضافه‌ای را رد می‌کنند.
                    \subsubsection{The Philosophical Significance of Mathematical Platonism}
فرگشت ریاضیاتی دارای پیامدهای فلسفی قابل توجهی است که با چالش به ایده فیزیک‌باوری مبتنی بر اینکه واقعیت تنها محدود به جهان فیزیکی است، همراه است. طبق نظریه فرگشت، واقعیت فراتر از فیزیکی است و شامل اشیاء انتزاعی است که بخشی از ترتیب فضایی-زمانی و عاملیت مطالعه‌شده در علوم فیزیکی نیستند. به علاوه، اگر فرگشت ریاضی صحیح باشد، این بر نظریه‌های طبیعی‌گرایی درباره دانش فشار می‌آورد، زیرا ما بدون شک دانش ریاضیاتی را داریم که شامل دانش اشیاء عاملیت‌ناپذیر است.
                        \\
                        \\
اگر چه این پیامدهای فلسفی به صرفه‌جویی محدود به فرگشت ریاضی نیستند، اما این شکل از فرگشت به دلیل موفقیت چشمگیر ریاضیات به عنوان یک رشته بنیادی و ابزاری برای سایر علوم، بخصوص به عنوان یک زبان علمی جهانی، برای حمایت از آن‌ها آماده است. فیلسوفان تحلیلی معاصر در مقابل تناقض با هر یک از ادعای اصلی یک رشته با مدارک علمی ریاضی، نیازمندی ندارند. رد کردن ریاضیات به طور کلی جذابیتی ندارد اگر تحلیل فلسفی دلایل قابل توجهی از این رشته نشان دهد. به علاوه، یک شکل از فرگشت که بر پایه یک رشته با مدارک علمی کمتر از ریاضیات استوار است، شانس کمتری برای حمایت از پیامدهای فلسفی دارد. به عنوان مثال، زمانی که صحبت از الهیات منجر به پیامدهای فلسفی غیرمنتظره می‌شود، بسیاری از فیلسوفان تردید ندارند تا بخش‌های مربوط به الهیات را رد کنند.
                    \subsubsection{Object Realism}
صرفیت اشیاء، وجود اشیاء ریاضی انتزاعی را ادعا می‌کند و ترکیبی از وجود و انتزاعیت است. این دیدگاه با نومینالیسم در تضاد است که به دلیل عدم وجود اشیاء انتزاعی، هر چند اصطلاح "نومینالیسم" ممکن است به باور عدم وجود جهانی در قالبی فلسفی سنتی اشاره کند.
                        \\
                        \\
استقلال در فرگشت ریاضی، عنصر دیگری است که از نظر منطقی به معنای ضعیف‌تری در مقایسه با صرفیت اشیاء قرار می‌گیرد. بنابراین، پیامدهای فلسفی استنتاجی از صرفیت اشیاء قوی‌تر و شفاف‌تر از آنچه در صورت استقلال در فرگشت ریاضی به دست می‌آید، است. فیزیک‌باوران ممکن است شیء‌های غیرفیزیکی را بپذیرند اگر آن‌ها به شیء‌های فیزیکی وابسته باشند یا به آن‌ها کاهش‌پذیر باشند، مانند شرکت‌ها، قوانین و شعرها. علاوه بر این، در مورد دانش اشیاء غیرفیزیکی که خودمان ساخته یا تشکیل داده‌ایم، هیچ رمزی وجود ندارد.
                        \\
                        \\
بعضی دیدگاه‌های در فلسفه ریاضیات، با صرفیت اشیاء همخوانی دارند، اما فرگشت‌باور نیستند. به عنوان مثال، دیدگاه‌های سنتی اینچوئیتیسم، وجود اشیاء ریاضی را تأیید می‌کنند، اما این اشیاء را به شیء‌هایی که بستگی به ریاضی‌دانان و فعالیت‌های آن‌ها دارند یا توسط آن‌ها تشکیل می‌شوند، وابسته می‌دانند.
                    \subsubsection{Truth-Value Realism}
صرف‌واقعیت ارزش حقیقت، فرض می‌کند که هر بیان ریاضیاتی درست‌شده دارای یک ارزش حقیقت منحصر به فرد و موضوعی است که رابطه‌ای با توانایی ما برای دانستن آن یا اینکه آیا از نظر منطقی به نظریات ریاضیاتی فعلی منطقی است، ندارد. علاوه بر این، این دیدگاه ادعا می‌کند که بسیاری از بیانیه‌های تلقی شده به عنوان حقیقی در ریاضیات، در واقع حقیقت دارند. این در حالی است که این دیدگاه متافیزیکی است، اما با فرگشت تفاوت دارد زیرا صرف‌واقعیت ارزش حقیقت، به جزییات اشیاء ریاضی خود را متعهد نمی‌کند تا ارزش‌های حقیقت بیانیه‌های ریاضی را توضیح دهد.
                        \\
                        \\
فرگشت می‌تواند با توضیح اینکه چگونه بیانیه‌های ریاضی ارزش‌های حقیقت خود را پیدا می‌کنند، الهام‌بخش صرف‌واقعیت ارزش حقیقت باشد. با این حال، صرف‌واقعیت ارزش حقیقت، مگر اینکه فرضیات اضافی اضافه شود، به فرگشت یا صرفیت اشیاء انتزاعی وابسته نیست. نومینالیست‌ها که به وجود اشیاء انتزاعی اعتقاد ندارند، همچنان می‌توانند صرف‌واقعیت ارزش حقیقت را حداقل برای شاخه‌های پایه ریاضیات مانند حساب، تأیید کنند. در این صورت، آن‌ها بیانیه‌های ریاضی را به یک زبان ترجمه می‌کنند که به وجود اشیاء انتزاعی وابسته نیست.
                        \\
                        \\
بعضی از فیلسوفان معتقدند که بحث در مورد فرگشت باید به بحث در مورد صرف‌واقعیت ارزش حقیقت تبدیل شود زیرا برای هر دو فلسفه و ریاضیات بیشتر و روشن‌تر است. آن‌ها اعتقاد دارند که بحث در مورد صرف‌واقعیت ارزش حقیقت قابلیت پیگیری بیشتری نسبت به بحث در مورد فرگشت دارد که اغلب نامفهوم است.
                    \subsubsection{The Mathematical Significance of Platonism}
صرف‌واقعیت کاری، یک دیدگاه روش‌شناسانه است که برای عمل به ریاضیات به عنوان گویی درست براساس فرگشت، طبق نظر برنایز (1935) و شاپیرو (1997، صفحات 21-27 و 38-44) از آن حمایت می‌کند. این نیاز به توضیحات بیشتری دارد. فرگشت در گذشته برای دفاع از روش‌های خاص ریاضی در بحث‌هایی درباره پایه‌های ریاضیات مورد استفاده قرار گرفته است، از جمله:
                        \begin{enumerate}
                            \item زبان‌های کلاسیک اولین نوع (یا قدرتمندتر)، که عبارت‌های تنها و مقداردهنده‌های آن به نظر می‌رسد به اشیاء ریاضی اشاره و بر روی آن‌ها گسترش پیدا می‌کنند. (در برخی قرن‌های قبل از تاریخ ریاضیات، از زبان‌هایی استفاده می‌شد که بیشتر به واژگان سازنده و حالتی وابسته بودند.)
                            \item منطق کلاسیک به جای منطق انتزاعی.
                            \item روش‌های غیر سازنده (مانند اثبات وجودی غیر سازنده) و مصوبات غیر سازنده (مانند اصل انتخاب).
                            \item تعریف ‌های غیر پایدار (به اصطلاح، تعریف‌های تمام‌گیر) که در آن شمارش بر روی کلیت‌ای صورت می‌گیرد و جسمی که تعریف می‌شود، عضو چنین کلیت‌ای است.
                            \item «بی‌نگرانی هیلبرتی»، به معنای اعتقاد به اینکه هر مسئله ریاضی در اصل قابل حل است.
                        \end{enumerate}
صرف‌واقعیت کاری و فرگشت دو دیدگاه متمایز درباره ریاضیات هستند. در حالی که صرف‌واقعیت کاری یک دیدگاه داخل ریاضیات است که به روش مناسب این حوزه توجه می‌کند، فرگشت یک دیدگاه صریح فلسفی است. صرف‌واقعیت کاری نظریه‌ای درباره اینکه آیا روش‌های کلاسیک ریاضی نیاز به دفاع فلسفی دارند یا خیر، و آیا چنین دفاعی باید بر پایه فرگشت باشد یا نه، اخذ نمی‌کند.
                        \\
                        \\
با این حال، بین دو دیدگاه رابطه‌های منطقی وجود دارد. اگر فرگشت ریاضی درست باشد، این می‌تواند استفاده از روش‌های کلاسیک مرتبط با صرف‌واقعیت کاری را توجیه کند. فرگشت ریاضی همچنین به دنبال پیدا کردن مصوبات جدید برای حل سؤالات بازگشت نشده توسط نظریه‌های ریاضی فعلی، مانند فرضیه پیوستگی، تشویق می‌کند.
                        \\
                        \\
                        از سوی دیگر، صرف‌واقعیت کاری به طور ضروری فرگشت را نیازمند نمی‌کند. در حالی که صرف‌واقعیت کاری می‌گوید که استفاده از زبان فرگشتی ریاضیات معاصر ما توجیه شده است، این کافی برای فرگشت حداقل در دو جهت کافی نیست. اولاً، زبان فرگشتی ریاضیات می‌تواند به گونه‌ای تحلیل شود که از ارجاع به و سرشماری از اشیاء ریاضی خودداری کند. دوماً، حتی اگر تحلیل صریح زبان ریاضیات قابل توجیه باشد، این تنها پشتیبانی از صرف‌واقعیت اشیاء است و نه فرگشت. برای سومین عنصر فرگشت، یعنی استقلال، باید دلیل دیگری مطرح شود.\cite{sep-platonism-mathematics}\cite{Benson2006-jm}\cite{Tegmark2008-qv}
                \subsection{The Fregean Argument for Existence}
در اینجا، یک الگوی کلی برای استدلالی که پشتیبانی از وجود اشیاء ریاضیاتی می‌کند، ارائه می‌دهیم، که به آن استدلال فرگه‌ای گفته می‌شود. در حالی که فرگه نخستین فیلسوفی بود که یک استدلال به این شکل توسعه داد، این الگو به طور گسترده و انتزاعی می‌تواند به کار برده شود، بدون اینکه به جزئیات خاص دفاع فرگه از وجود اشیاء ریاضیاتی وابسته باشد.
                    \\
                    \\
مهم است که توجه شود که لوگیسیسم فرگه‌ای تنها یک روش برای توسعه این استدلال نیست. دیگر فیلسوفان نیز نسخه‌های متفاوتی از این استدلال ارائه داده‌اند، که در زیر بحث خواهیم کرد.
                    \subsubsection{Structure of the Argument}
استدلال فرگه‌ای بر دو فرضیه تکیه می‌کند، اولی که بازتاب زبان ریاضیات را دربرمی‌گیرد:
                        \begin{qt}
                            \textbf{Classical Semantics}
                            \\
عبارات تنها در زبان ریاضیات به اشیاء ریاضیاتی اشاره دارند و مقداردهندهای نخستین آن به چنین اشیاءی می‌رسند.
                        \end{qt}
کلمه "purport" باید توضیح داده شود. وقتی جمله ای $S$ به یک روش خاص به ارجاع یا سرشماری می‌پردازد، این بدان معناست که درستی $S$ نیازمند موفقیت در ارجاع یا سرشماری به این روش است. فرضیه دوم نیاز به توضیحات زیادی ندارد:
                        \begin{qt}
                            \textbf{Truth}
                            \\
اغلب جملاتی که به عنوان قضایای ریاضی پذیرفته می‌شوند، درست هستند (بدون توجه به ساختار نحوی و معنایی آن‌ها).
                        \end{qt}
با توجه به فرضیات گفته شده، جملاتی را درنظر بگیرید که به عنوان قضایای ریاضی پذیرفته شده‌اند و شامل یک یا چند عبارت تنها ریاضی هستند. با توجه به درستی، بیشتر این جملات درست هستند. فرض کنید $S$ یکی از این جملات باشد. با توجه به علم نحوی، برای حقیقت $S$ لازم است که عبارات تنهای ریاضی آن به موفقیت در ارجاع به اشیاء ریاضی برسند. بنابراین باید اشیاء ریاضی وجود داشته باشند، همانطور که فرضیه وجود ادعا می‌کند.
                    \subsubsection{Defending Classical Semantics}
علم نحوی کلاسیک یک ادعای تفسیری درباره زبان ریاضیات است که می‌گوید که زبان به همان شیوه نحوی به کار می‌رود که زبان عادی استفاده می‌شود. این ادعا توصیفی بوده و نه قابل ارزیابی، و با دیدگاه‌های سنتی درباره نحو و زبان شامل تطابق گسترده‌ای است. اعتبار علم نحوی کلاسیک از شباهت‌های ظاهری نحوی بین زبان ریاضیاتی و غیر ریاضیاتی به وجود می‌آید، که توسط بررسی های نحوی و دانشمندان نامتعارف پشتیبانی می‌شود.
                        \\
                        \\
با این حال، برخی از فیلسوفان، از جمله نامی‌گرایان مانند هلمن و هافوبر، علم نحوی کلاسیک را در چالش قرار داده‌اند. آن‌ها می‌گویند که ساختار نحوی زبان ریاضیات اساساً با ساختار نحوی زبان عادی متفاوت است و این تفاوت اعتبار اولیه علم نحوی کلاسیک را ضعیف می‌کند. برای اثبات این چالش‌ها، آن‌ها باید نشان دهند که شباهت‌های ظاهری نحوی بین زبان ریاضیات و غیر ریاضیاتی فریبنده هستند و دلایل قانع کننده‌ای ارائه دهند که زبان‌شناسان و دانشمندان نامتعارف به عنوان مهم تلقی کنند.
                    \subsubsection{Defending Truth}
روش‌های مختلفی برای دفاع از درستی جملات ریاضی وجود دارد. به طور کلی، این دفاع‌ها در ابتدا یک استاندارد شناسایی می‌کنند که با آن می‌توان قابلیت ارزیابی درستی جملات ریاضی را تأیید کرد و سپس برای هر قضیه ریاضی، دلایلی ارائه می‌دهند که نشان می‌دهد آن قضیه به این استاندارد مطابقت دارد.
                        \\
                        \\
یکی از گزینه‌ها، تکیه بر استاندارد بنیادی خارج از ریاضیات مانند لوگیسیسم است. در این رویکرد، ادعا شده که هر قضیه منطق خالص درست است و سعی می‌شود نشان داده شود که قضایای بخشی از ریاضیات می‌توانند از منطق خالص و تعاریف به دست آید.
                        \\
                        \\
گزینه دیگر، تکیه بر استانداردهای علم تجربی مانند در استدلال قابلیت نیازپذیری کواین-پاتنام است. در این رویکرد، ادعا شده است که هر قسمت ضروری علم تجربی احتمالاً درست است و از اینرو، باور درستی آن مبرر است. سپس بر این اساس ادعا شده است که حجم گسترده‌ای از ریاضیات برای علم تجربی ضروری است و از اینرو، باور ترویجی به درستی مبرر است.
                        \\
                        \\
گزینه سوم که به آن طبیعت‌گرائی یا طبیعت‌گرایی ریاضیاتی می‌گویند، تکیه بر استانداردهای خود ریاضیات است. مبرریت ادعاهای ریاضی بر اساس استانداردهای منحصر به فرد ریاضیات، و نه تکیه بر استانداردهای غیر ریاضی مانند منطق یا علم تجربی، است. این روش در دوران اخیر توجه بسیاری از فیلسوفان را به خود جلب نموده است. به عنوان مثال، ریاضیدانان در کل از درستی قضایا ریاضی قانع هستند و این پذیرش می‌تواند موقعیت معقولی برای دفاع از درستی ریاضیات فراهم کند.
                        \begin{enumerate}
                            \item ریاضیدانان به اندازه کافی مبرر هستند تا در قبول قضایای ریاضیاتی باور و قانع باشند
                            \item قبول کردن یک جمله ریاضی $S$ شامل درستی آن جمله است.
                            \item وقتی یک ریاضیدان جمله ریاضی $S$ را قبول می‌کند، محتوای این نگرش به طور کلی معنای لحنی $S$ است.
                        \end{enumerate}
با توجه به این سه گزاره، نتیجه می‌شود که متخصصان ریاضیات در قبول قضایا ریاضی به لحاظ حرفی واقعیت‌های حرفی استدلال‌پذیر هستند. با ارتباط این نتیجه به سایر افراد، ما نیز باور درستی را ترویجی می‌توانیم داشته باشیم. باید توجه داشت که متخصصین که در (1) مورد علاقه قرار دارند، نیازی به باور به (2) و (3) ندارند، بلکه مهم آن است که سه گزاره درست باشند. وظیفه بررسی درستی (2) و (3) ممکن است به عهده زبان‌شناسان، روان‌شناسان، جامعه‌شناسان یا فیلسوفان باشد، امّا حتماً بر عهده خود ریاضیدانان نیست.
                        \\
                        \\
                     بدون شک، فیلسوفان داستان‌گوی ریاضی به سختی سعی می‌کنند در برابر (2) و (3) مقاومت کنند.\cite{sep-platonism-mathematics}
                    \subsubsection{The Notion of Ontological Commitment}
نسخه‌هایی از استدلال فرگه گاهی به صورت مفهوم تعهد موجودیت بیان می‌شوند. فرض کنید با معیار Quinean عادی تعهد موجودیت عمل کنیم:
                        \begin{qt}
                            \textbf{Quine’s Criterion.}
یک جمله درجه اول (یا مجموعه‌ای از چنین جملاتی) به موجودیت‌هایی تعهد دارد که برای درستی آن جمله (یا مجموعه جملات)، باید فرض شود که در دامنه متغیرهای آن جمله (یا مجموعه جملات) وجود دارند.
                        \end{qt}
دیدگاه معمول در نحوه تفسیر ریاضیات، نشان می‌دهد که بسیاری از جملات ریاضی به موجودیت‌های ریاضی تعهد دارند. این به این دلیل است که عبارات تکیه‌گاه و نماینده جمع، که برای مراجعه به و یا گسترش بر روی اشیاء ریاضی کاربرد دارند، باید برای قضیه ریاضی صحیح باشند. بنابراین، براساس معیار Quine، برای قضیه ریاضی می‌بایست به موجودیت‌های ریاضی تعهد شود. با این حال، چالش‌هایی برای این معیار وجود دارد و برخی فیلسوفان تقاضا دارند که به منظور درستی جملات، نه لزوماً تمام اشیاء در دامنه متغیرها باید وجود داشته باشند، و نه باید پیوند بین مؤیدِ وجود جمع اول درجه و تعهد موجودیت قطع شود.
                        \\
                        \\
یک ضد استدلال به این چالش‌ها، این است که استدلال فرگه ارائه شده در بالا بر مفهوم تعهد موجودیت وابسته نیست و بنابراین، چالش برای تعریف آن بی‌ربط است. با این حال، مخالفان می‌توانند دلیل کنند که استنتاج نهایی آن، یعنی وجود، برای داشتن تأثیر دلخواه خود ضعیف است. نتیجه در زبان فلسفی $LP$ به صورت "$\exists xMx$" فرموله شده است که ممکن است در صورتی که مخالفان مفهوم Quinean را از تعهد موجودیت پذیرفته نکنند، تعهد موجودیتی ایجاد نکند. در نهایت، مخالفان باید توضیح دهند که چرا مفهوم غیر استاندارد خود از تعهد موجودیت، قابلیت شرح دادن بیشتر و جذابیت نظری بیشتری نسبت به مفهوم Quinean استاندارد دارد.
                    \subsubsection{From Existence to Mathematical Platonism?}
فرض کنید ما وجود را قبول کرده‌ایم، شاید بر اساس استدلال فرگه. همانطور که دیدیم، این هنوز به معنویت ریاضیاتی قابل قبول نیست که نتیجه اضافه کردن دو گام دیگر، یعنی انتزاع و استقلال، است. آیا این دو گام بیشتر دفاع پذیر هستند؟
                        \\
                        \\
بر اساس استانداردهای فلسفی، انتزاع به نسبت مورد بحث کمتر دارای جنجال بوده است. در بین کمترین فیلسوفانی که آن را به چالش کشیده‌اند، می‌توان به Maddy (1990) (در مورد مجموعه‌های غیر خالص) و Bigelow (1988) (در مورد مجموعه‌ها و انواع مختلفی از اعداد) اشاره کرد. این عدم جنجال نسبی منجر به توسعه چندانی از دفاع رسمی از انتزاع نشده است. با این حال، نامساعد نیست که چگونگی دفاع از انتزاع ممکن است باشد. یکی از ایده‌ها این است که هر تفسیر فلسفی از تمرین ریاضی باید از نظر مبدأ پذیرفتنی باشد که ویژگی‌هایی را که باعث می‌شوند تمرین ریاضی درست نباشد یا ناکارآمد باشد، به ریاضیات نسبت دهد. این محدودیت باعث می‌شود که سخت باشد بر خلاف این حقیقت ادعا شود که اشیاء ریاضیات خالص دارای موقعیت فضایی-زمانی هستند، زیرا تمرین ریاضی واقعی در این صورت نادرست و ناکارآمد خواهد بود، چرا که ریاضیدانان خالص در این صورت باید به موقعیت اشیائشان علاقه مند شوند، به طریقی که زیست شناسان به موقعیت حیوانات علاقه مند هستند. حقیقت این است که عدم علاقه ریاضیدانان خالص به این پرسش نشان می‌دهد که اشیائشان انتزاعی هستند.
                        \\
                        \\
مفهوم استقلال در معنویت ریاضیاتی، به این معناست که اگر اشیاء ریاضی وجود دارند، از عاملان هوشمند و زبان، فکر و روش‌های آنها مستقل هستند
        \section{Determinism and Computation}
            \subsection{Turing Machines}
آلن تورینگ در سال‌های ۱۹۳۶-۱۹۳۷ دستگاه‌های تورینگ را به عنوان دستگاه‌های محاسباتی نظری و ابزارهای پایه برای بررسی مرزهای آنچه که محاسبه‌پذیر است و محدودیت‌های آن معرفی کرد. در ابتدا به عنوان "دستگاه‌های خودکار" شناخته شدند و برای محاسبه اعداد حقیقی طراحی شده بودند و در بررسی تورینگ درباره آنها، الونزو چرچ آنها را "دستگاه‌های تورینگ" نامید. امروزه، دستگاه‌های تورینگ یکی از مدل‌های بنیادین در علوم کامپیوتر و نظریه محاسبات هستند که تأثیرات قابل توجهی در درک طبیعت محاسبه و محدودیت‌های آن برای کامپیوترها دارند.
                \\
                \\
تورینگ دستگاه‌های تورینگ را در زمینه تحقیقات بنیادین ریاضیات معرفی کرد. به طور خاص، او از این دستگاه‌های انتزاعی برای اثبات این مطلب استفاده کرد که هیچ روش عمومی یا فرآیند موثری برای حل، محاسبه و محاسبه هر نمونه از مسئله زیر وجود ندارد:
                \begin{qt}
                    \textbf{Entscheidungsproblem} مسئله تصمیم برای هر گزاره در منطق نخستین نوع، تصمیم گیری اینکه آیا در آن منطق قابل استنتاج است یا خیر.\cite{sep-spacetime-bebecome}
                \end{qt}
نسخه اصلی مسئله، که در سال ۱۹۲۸ توسط هیلبرت و آکرمن ارائه شد، بر روی مفهوم اعتبار به جای قابل استنتاج متمرکز شده بود. با این حال، قضیه کاملیت گودل از سال ۱۹۲۹ نشان داد که اثبات وجود (یا عدم وجود) یک روش موثر برای قابلیت استنتاج نیز مشکل را حل می کند. برای پاسخ به این سؤال به درک دقیقی از چه چیزی یک "روش موثر" تشکیل می‌دهد، نیاز است که دستگاه‌های تورینگ مطرح شود. این دستگاه‌ها در سال ۱۹۳۶ به عنوان یک مدل انتزاعی از محاسبه توسعه یافتند و تعریف فورمالی یک روش موثر را فراهم کردند و به محققین اجازه دادند تا به مسئله Entscheidungs با دقت بیشتری بپردازند. در نهایت، کار تورینگ نشان داد که هیچ روش الگوریتمی برای تعیین قابل ثبوت بودن گزاره‌های دلخواه وجود ندارد، که به عدم تصمیم پذیری مسئله Entscheidungs منجر شد.
                \\
                \\
یک دستگاه تورینگ، یا همان دستگاه محاسباتی که به آن توسط تورینگ اشاره شده است، در اصل یک ماشینی را تعریف می‌کند که قادر به مجموعه محدودی از پیکربندی‌ها $q_1،\dots،q_n$ (توسط تورینگ به آن m-پیکربندی‌ها گفته می‌شود) است. این ماشین از یک نوار بی‌نهایت یک‌طرفه تشکیل شده است که به مربع‌هایی تقسیم می‌شود و هر مربع دقیقاً قادر به نگهداری یک نماد است. این نمادها با $S_0، S_1، \dots، S_m$ نشان داده می‌شوند، جایی که $S_1 = 0$ و $S_2 = 1$ هستند. در طول عملیات، دستگاه محتوای یک مربع روی نوار را اسکن می‌کند که می‌تواند یک نماد خالی ($S_0$) یا یکی از نمادهای دیگر را نگهداری کند.
                \\
                \\
دستگاه تورینگ به عنوان دستگاه خودکار یا a-machine طبقه‌بندی می‌شود، به این معنا که رفتار آن در هر لحظه به طور کامل توسط پیکربندی فعلی که در حال اسکن شدن است (که از حالت فعلی و نماد فعلی تشکیل شده است) تعیین می‌شود. این شرط به عنوان قطعیت شناخته می‌شود. دستگاه‌های a-machine از دستگاه‌های انتخابی متمایز می‌شوند، که برای تعیین حالت بعدی به تصمیم یک عامل خارجی یا دستگاه نیاز دارند (همانطور که در مقاله تورینگ از سال‌های ۱۹۳۶-۱۹۳۷: ۲۳۲ شرح داده شده است). دستگاه تورینگ قادر به انجام سه نوع عمل است:
                \begin{itemize}
                    \item Print $S_i$ move one square to the left ($L$) and go to state $q_j$.
                    \item Print $S_i$, move one square to the right ($R$) and go to state $q_j$.
                    \item Print $S_i$, do not move ($N$) and go to state $q_j$
                \end{itemize}
'برنامه' یک دستگاه تورینگ می‌تواند به شکل مجموعه متناهی از پنج‌تایی‌هایی با فرم زیر نوشته شود:
                \begin{equation}
                    q_i S_jS_{i,j}M_{i,j}q_{i,j}
                \end{equation}
تعریف اولیه آلن تورینگ از ماشین‌های محاسباتی از دو نوع نماد استفاده می کرد - شکل هایی که به طور کامل از ۰ و ۱ تشکیل شده بودند، و همچنین "نمادهای نوع دوم" که با استفاده از یک سیستم مربع های متناوب روی نوار تفکیک می شدند. دنباله‌ای که توسط ماشین محاسبه می شد در مربع های F قرار داشت، در حالی که مربع های E برای نشان دادن مربع های F و "کمک به حافظه" استفاده می شدند. با این حال، استفاده از مربع های F و E با مشکلاتی روبرو شد، همانطور که امیل ال پُست نیز به آن اشاره کرد.
                \\
                \\
دو جنبه مهم در خصوص تنظیمات دستگاه تورینگ وجود دارد. اولاً، نوار دستگاه به طور پتانسیل بی‌نهایت است، که به معنای حافظه دستگاه نیز پتانسیل بی‌نهایت است. در ثانی، یک تابع به عنوان Turing computable در نظر گرفته می‌شود اگر مجموعه دستورالعمل‌هایی وجود داشته باشد که منجر به محاسبه آن تابع توسط یک دستگاه تورینگ شود، بدون در نظر گرفتن زمانی که صرف محاسبه آن تابع می‌شود. این فرضیات به این معنی است که هیچ تابع قابل محاسبه‌ای به علت محدودیت زمان یا حافظه به کامپیوتر تورینگ-computable تبدیل نمی‌شود. با این حال، برخی از توابع Turing-computable ممکن است به دلیل محدودیت حافظه توسط کامپیوترهای موجود اجرا نشوند و برخی دیگر هیچگاه در عمل قابل محاسبه نخواهند بود زیرا به حافظه بیشتری نسبت به تعداد اتم‌های موجود در جهان نیاز دارند. زمانی که یک تابع به عنوان non-Turing computable نشان داده شود، این نتیجه بسیار قوی است زیرا به معنی عدم وجود کامپیوتری است که محاسبه آن تابع را انجام دهد.\cite{sep-turing-machine}
            \subsubsection{Turing's Universal Machine}
دستگاه تورینگ یکتا برای نشان دادن عدم قابلیت محاسبه برخی از مسائل توسعه یافته است. در اصل، یک دستگاه تورینگ است که هر چیزی را که دستگاه تورینگ دیگری محاسبه می کند می تواند محاسبه کند. با فرض اینکه مفهوم دستگاه تورینگ به طور کامل قابلیت محاسبه را در بر می‌گیرد (که منجر به صحت آستانه تورینگ می‌شود)، پیروی از آن است که هر چیزی که می تواند محاسبه شود، قابل محاسبه توسط دستگاه یونیورسال است. به عبارت دیگر، هر مسئله ای که نمی تواند توسط دستگاه یونیورسال محاسبه شود، غیر قابل محاسبه تلقی می شود.
                \\
                \\
                این ایده قدرتمند و تئوریک پشت دستگاه یونیورسال است - یک دستگاه فرمال ساده که قابلیت درک همه فرآیندهای ممکن در محاسبه یک عدد را داراست. همچنین، این یکی از دلایل اصلی است که آلن تورینگ اکنون به عنوان یکی از پدران بنیان علم کامپیوتر تجلیل می شود.\cite{sep-turing-machine}
            \subsection{Philosophical Issues Related to Turing Machines}             
                \subsubsection{Human and Machine Computations}
تطبیق اولیه آلن تورینگ بین اعداد قابل محاسبه و دستگاه‌های تورینگ هدفش نشان دادن این بود که مسئله Entscheidungsproblem یک مسئله قابل محاسبه نیست و به همین دلیل یک "مسئله کلی" نیز نیست. با این حال، سؤال باقی می‌ماند که آیا نظریه آلن تورینگ از مفهوم شناختی و اینتوییتیو خود در قابلیت محاسبه، تمام مسائل قابل محاسبه و انواع محاسبه را شامل می‌شود یا خیر. این یک مسئله اساسی در فلسفه علم کامپیوتر است.
                \\
                \\
باید توجه داشت که در زمان نگارش مقاله آلن تورینگ، کامپیوترهای مدرن هنوز توسعه نیافته بودند. بنابراین، تفسیرهایی که قابلیت محاسباتی تورینگ را با قابلیت محاسبه توسط کامپیوترهای مدرن مشترک می‌دانند، بیانی‌تاریخی درست از آستانه تورینگ نیستند. دستگاه‌های محاسباتی آن زمان مانند تحلیل‌گرهای دیفرانسیل و ماشین حساب‌های میزی، در قابلیت محاسباتی خود محدود بودند و به طور عمده در زمینه شیوه‌های محاسباتی انسانی مورد استفاده قرار می‌گرفتند. بنابراین، تورینگ در مقاله خود شیوه‌های محاسباتی انسانی را فرمالیزه کرده و مفهوم مسائل قابل محاسبه‌ی او، به مسائلی اشاره می‌کرد که توسط شیوه‌های انسانی قابل حل هستند.
                \\
                \\
در بخش 9 مقاله خود، تورینگ با تجزیه فرآیند محاسبات انسانی، توضیح داد که دستگاه‌های تورینگ به عنوان مدل طبیعی محاسبات انسانی استفاده می‌شوند. او یک "محاسب کننده" انسانی انتزاعی را به وجود آورد که شرایطی را که بر اساس محدودیت‌های انسانی برای محاسبه وجود دارند، برآورده می کرد (شامل محدودیت‌های سیستم حسی و ذهنی ما). این "محاسب کننده" اعداد واقعی را روی نوار بی‌نهایت یک بعدی با مربع‌های مشخص شده محاسبه می کرد، با برخی محدودیت‌ها. این محدودیت‌ها و محدودیت‌های دیگر توسط محققانی همچون گندی و سیگ بیشتر تجزیه و تحلیل شده‌اند.
                \begin{itemize}
                    \item \textbf{Determinacy condition D} "رفتار کامپیوتر در هر لحظه توسط نمادهایی که در آن نگاه می‌کند و "حالت ذهنی" او در آن لحظه تعیین می‌شود." (آلن تورینگ 1936-7: 250)
                    \item \textbf{Boundedness condition B1} "یک مرز B برای تعداد نمادها یا مربع‌هایی که کامپیوتر می تواند همزمان مشاهده کند وجود دارد. اگر او بخواهد بیشتر را مشاهده کند، باید از مشاهدات پیاپی استفاده کند." (آلن تورینگ 1936-7: 250)
                    \item \textbf{Boundedness condition B2} "تعداد حالت‌های ذهنی که باید در نظر گرفته شود، محدود است."
                    \item \textbf{Locality condition L1} "می‌توانیم فرض کنیم که مربع‌هایی که نماد آن‌ها تغییر می‌کند همیشه مربع‌های "مشاهده شده" هستند." (آلن تورینگ 1936-7: 250)
                    \item \textbf{Locality condition L2} "هر یک از مربع‌های جدید مشاهده شده، در داخل L مربع از یک مربع قبلی که به تازگی مشاهده شده است، قرار دارد." (آلن تورینگ 1936-7: 250)
                \end{itemize}
تجزیه و تحلیل تورینگ و مدل حاصل، در نظر بسیاری از متخصصان به عنوان بهترین مدل استاندارد قابلیت محاسبه امروزی در نظر گرفته می شود؛ به دلیل آنکه شرایط فوق الذکر می توانند به سادگی برای نحوه تولید دستگاه های تورینگ با تجزیه آن‌ها به عملیات ساده ای که به اندازه کافی ابتدایی هستند و نیاز به تقسیم بیشتری ندارند، استفاده شوند. جهت دیدن یک بیانیه قوی از این نقطه نظر، به Soare 1996 مراجعه کنید.
                \\
                \\
مهم است بیان کنیم که با اینکه تجزیه و تحلیل تورینگ در مورد محاسبات انسانی تمرکز داشت، شناسایی او بین محاسبه (انسانی) و محاسبه با دستگاه تورینگ اعمال شده در مسئله Entscheidungsproblem نشان می دهد که او به وجود یک مدل محاسباتی که امکانات محاسباتی انسان را پشت سر بگذارد و روش کلی و موثری برای حل مسئله Entscheidungsproblem ارائه دهد، فکر نکرده است. اگر او به احتمال وجود چنین مدلی فکر کرده بود، مسئله Entscheidungsproblem را قابل محاسبه تلقی کرده بود، نه غیر قابل محاسبه.
                \\
                \\تمرکز تورینگ بر روی محاسبات انسانی در تجزیه و تحلیل او از محاسبات، محققان را به گسترش دامنه تجزیه و تحلیل تورینگ برای شامل محاسبات توسط دستگاه‌های فیزیکی ترغیب کرده است، که منجر به نسخه های مختلفی از اصل چرچ-تورینگ فیزیکی شده است. به عنوان مثال، Robin Gandy تحلیل تورینگ را به دستگاه‌های مکانیکی گسسته توسعه داد و یک مدل جدید بر اساس مجموعه اساسی از محدودیت‌های قابل کاهش به مدل دستگاه تورینگ ارائه داد. Wilfried Sieg این کار را ادامه داد و چارچوب سیستم‌های پویا محاسباتی (Computable Dynamical Systems) را ارائه کرد. دیگران امکان وجود مدل های "معقول" در فیزیک را بررسی کرده اند که می توانند چیزی غیر قابل محاسبه با دستگاه تورینگ را "محاسبه" کنند. به عنوان مثال، Aaronson، Bavarian و Gueltrini (2016) نشان دادند که اگر خم‌های بسته زمانی وجود داشته باشند، مسئله تعطیلی با منابع محدود حل پذیر است. برخی همچنین مدل های جایگزینی را برای محاسبات پیشنهاد کرده اند که الهام گرفته شده از مدل دستگاه تورینگ هستند، اما برای گرفتن جنبه های خاصی از شیوه‌های محاسباتی فعلی، به آن‌ها تجهیز شده اند. یک مثال از این دستگاه‌ها، دستگاه تورینگ پایدار است که طراحی شده است تا فرایندهای تعاملی را دربرگیرد. با این حال، این پیشنهادات وجود "مسائل قابل محاسبه" را فراتر از آنچه دستگاه تورینگ می‌تواند محاسبه کند، نشان نمی دهند. برخی نویسندگان آن‌ها را به عنوان مدل‌های معقول محاسباتی تلقی می‌کنند که به نظر می‌رسد بیشتر از دستگاه‌های تورینگ محاسبه می‌کنند که منجر به بحث‌هایی در جامعه علوم کامپیوتر در مورد هایپرمحاسبه در اوایل دهه 2000 شد. برای بیان مواضع مختلف، به Teuscher 2004 مراجعه کنید.\cite{sep-turing-machine}
        \subsection{Causal Determinism}
من در این مطلب به طور کلی از عبارت "معینیت" استفاده خواهم کرد، نه "معینیت علتی"، به هماهنگی با شیوه فلسفی اخیر در تمایز دیدگاه‌ها و نظریات در مورد علت و معلولیت از نتیجه‌های موفق یا ناموفق بودن معینیت. با این حال، مهم است به یاد داشته باشیم که علت و معلولیت اغلب با درک ما از معینیت پیوند دارند، به عنوان مثال در ادامه بررسی خواهیم کرد. در تاریخچه، مفهوم معینیت به شکل‌های نامسق و نا دقیقی تعریف شده است که هنگام بررسی مفهوم معینیت در یک زمینه نظری خاص، مشکل ساز می‌شوند. برای جلوگیری از خطاهای اساسی در تعریف، می‌توانیم با یک تعریف کلی و باز تعریف شده از مفهوم معینیت شروع کنیم: معینیت به موقعیت فلسفی اشاره دارد که در آن هر رویداد، از جمله عمل‌های بشر، در نهایت توسط علل پیشین تعیین می‌شود که به وقوع رویداد مورد نظر منجر می‌شود. این بدان معناست که اگر ما تمامی عوامل و شرایط مربوط به یک رویداد را بدانیم، می‌توانیم با دقت پیش‌بینی کنیم که چه اتفاقی خواهد افتاد. معینیت یک مسیر ثابت و غیر قابل تغییر برای آینده را نشان می‌دهد، زیرا همه رویدادها تحت تأثیر رویدادها و وضعیت قبلی و پیشین قرار دارند که به وقوع رویدادهای فعلی منجر می‌شوند.
            \begin{qt}
معینیت: جهان تحت اختیار معینیت است، اگر و فقط اگر در یک زمان $t$ روش مشخص شده‌ای برای چیزها وجود داشته باشد، روش پیش برد رویدادها در ادامه به عنوان یک قانون طبیعی تعیین شده باشد.
            \end{qt}
ایده معینیت به اصل کافی برای توجیه هر چیز به ظاهر، بازمی‌گردد. این اصل به این معنا است که هر چیزی که وجود دارد، در اصل قابل توضیح است - هر چیز دلیل کافی برای بودن و وجود داشتن خود را دارد. لایبنیتز این ایده را اصل کافی برای توجیه نامید. با این حال، با پدیدار شدن نظریه های دقیق فیزیکی که به نظر معینی بودند، مفهوم معینیت به طور متمایز از این ریشه های فلسفی شناخته شد. در فلسفه علم، علاقه قابل توجهی برای تعیین معینیت یا نامعینیت نظریات مختلف وجود دارد، بدون آن که لزوماً از دیدگاهی در مورد اصل کافی لایبنیتز شروع کرد.\cite{sep-determinism-causal}
            \subsubsection{Natural Laws}
در طول رویکرد علمی به طبیعت، جهان را در قالب حالت‌ها و تحولاتی که آن‌ها ایجاد می‌کنند توصیف کردیم. هراکلیتوس گفت: 
                \begin{qt}
همه چیز در جریان است و هیچ چیز پایدار نیست؛ همه چیز در حال تغییر است و هیچ چیز ثابت نمی‌ماند. (نقل از هراکلیتوس)
                \end{qt}
در ادامه، ما شروع به تعریف این تحولات را در قالب معادلات ریاضی کردیم. در مکانیک کلاسیک، حالت یک سیستم می‌تواند با مشخص کردن موقعیت و جرم آن در یک لحظه خاص از زمان به طور کامل تعیین شود و تحولات بعدی آن توسط معادلات دیفرانسیل معینیت‌طلب تعیین می‌شود.
                \\
                \\
در مکانیک کوانتومی، با وجود اینکه معادله شرودینگر برای توصیف تابع موج یک سیستم معینیت‌طلب است، اما رابطه بین تابع موج و خصوصیات قابل مشاهده سیستم به نظر نامعینیت‌افزار می‌آید. تابع موج تمام اطلاعاتی را که درباره سیستم می‌توان داشت، فراهم می‌کند، اما توصیف کامل خصوصیات آن را نمی‌دهد. به جای آن، تابع موج احتمالات مختلفی را برای نتایج مختلف اندازه‌گیری‌هایی که بر روی سیستم می‌توان انجام داد، فراهم می‌کند. این نامعینیت ظاهری یک خصوصیت برجسته از مکانیک کوانتومی است که آن را از مکانیک کلاسیک متمایز می‌کند..\cite{enwiki:1092966327}
    \section{Is Quantum Mechanics Deterministic?}
اکنون با توجه به اینکه ساختار فرمال و فلسفی مورد نیاز به طرز قابل قبولی کامل شده است، زمان صحبت با مکانیک کوانتومی فرا رسیده است.
        \\
        \\
        این تئوری در تقریباً سه سال توسعه یافت، راه بسیار سریعی برای داشتن یک تئوری اساسی از طبیعت است (البته نظریه میدان و یا به طور پتانسیل نظریه یکپارچه را در نظر نگرفتم). نتایجی که مکانیک کوانتومی به دانشمندان داد، بسیار دقیق بودند، اگر یک مسئله را در نظر نگیریم..\cite{enwiki:1142268588}
        \subsection{Measurement is the Only Mystery}
تئوری مکانیک کوانتومی به طور کلی یک تئوری نامعینیت‌افزار است. در این تئوری، تابع موجی وجود دارد که از طریق تحولات مختلف (که به صورت عملگرهای ریاضی به نظر می‌رسند)، در فضا و زمان پخش می‌شود و تا زمانی که آن را اندازه‌گیری نکنیم، به صورت معینی تحول خواهد کرد.
            \\
            \\
دیدگاه عمومی در مورد رفتار سیستم های کوانتومی، معمولاً بر اساس سه مرحله اساسی است که شامل موارد زیر می‌شود:
            \begin{enumerate}
                \item \textbf{State:} یک سیستم در اساس خود یک حالت است که براساس محیط و توصیفات آن شرح داده می‌شود. برای شروع هر فرآیند کوانتومی، نیاز به یک سیستم اولیه با یک حالت اولیه داریم. حالت یک شیء یک شیء ریاضی است که به عنوان زیرنویس به صورت زیر نشان داده می‌شود:
                \begin{equation}
                    \left| \psi_i \right> 
                \end{equation}
                \item \textbf{Evolution:} تعامل‌ها و تحولاتی که یک سیستم را می‌گذرانند، با استفاده از عملگرها توصیف می‌شود. در واقع هر چیزی که بر حالت سیستم ما تأثیر می‌گذارد، یک عملگر است که بر روی آن اعمال می‌شود. به همین دلیل، آن را به صورت زیر نشان می‌دهیم:
                \begin{equation}
                    \left|\psi_f\right>  = \hat U \left|\psi_i\right>
                \end{equation}
                یا با استفاده از معادله شرودینگر توصیف می‌شود:
                \begin{equation}
                    -\hbar\frac{\partial}{\partial t} \psi = \frac{\hbar^2}{2m}\nabla^2 \psi + V(x)\psi 
                \end{equation}
                \item \textbf{Measurement:} فرآیندی که اطلاعاتی درباره سیستم به دست می‌آوریم، در اصل با تحول متفاوتی نسبت به فرآیند تحول سیستم همراه است. شاید فکر کنید که اگر ما به عنوان سیستم‌های فیزیکی با یک سیستم کوانتومی در تعامل باشیم، باید این تعامل با استفاده از یک عملگر روی آن توصیف شود. اما به نظر می‌رسد که این چیزی است که در اصل به صورتی اساساً متفاوت از تحولات معمول رخ می‌دهد.
            \end{enumerate}
فرآیند اندازه‌گیری به صورت زیر است. فرض کنید یک تابع موج دارید، با اندازه‌گیری شما یک پاسخ برای X دریافت می‌کنید. اگر سیستم را مجدداً (پس از اندازه گیری قبلی) اندازه‌گیری کنید، پاسخ یکسانی دریافت خواهید کرد. به عنوان یک پاسخ دقیق مشابه یک اندازه گیری کلاسیک که طول یک میز همیشه ثابت باقی می‌ماند، بدون توجه به اینکه چند بار اندازه گیری شود.
            \\
            \\
\ حال فرض کنید یک مجموعه از حالت‌های دقیقاً یکسان دارید، اندازه‌گیری هر کدام از آن‌ها پاسخ‌های مختلفی یعنی $x_i$ را به شما می‌دهد. بنابراین، برای پیش‌بینی اندازه‌گیری یک سیستم، مجبور به صحبت درباره احتمالات نتایج مختلف هستید. توزیع احتمال ظاهر شدن سیستم شما در منطقه فضایی $(a، b)$ به عنوان یک مثال به صورت زیر نشان داده می‌شود:
            \begin{equation}
                P = \int_a^b |\psi|^2 dx 
            \end{equation}
در فیزیک (تا قبل از اندازه‌گیری در مکانیک کوانتومی)، احتمالات زمانی استفاده می‌شود که شما نسبت به چیزی سردرگم هستید. اگر معادلات برای حل بسیار بیشتر و بسیار دشوار باشند، از احتمالات استفاده می‌کنید تا برخی پاسخ‌ها را بیابید و با معادلات سخت مبارزه نکنید. یا زمانی که نیازی به دقت بالای تعاملات وجود ندارد. به عنوان مثال در ترمودینامیک، ما به فشار، دما و چیزهایی که در سطح اتمی واقعیت خاصی ندارند علاقه داریم. نشان داده شده است که این ویژگی ها برای دید سیستم های بزرگ و تعریف‌های آماری دارند.
            \\
            \\
در مقابل، در اندازه‌گیری سیستم کوانتومی، یک رفتار شبه تصادفی در پایه سیستم وجود دارد. اما این مشکل نیست. مشکل اینجاست که هیچگاه چنین فرآیندی در توصیف فیزیکی طبیعت قبلاً دیده نشده است! اندازه‌گیری به صورت قطعی نیست. از یک تابع موج تحول یافته، نمی‌توان به حالت اولیه بازگشت کرد. اگرچه در هر فرآیند در فیزیک کلاسیک و کوانتومی (به جز اندازه‌گیری)، این کار را می‌توانید انجام دهید.
            \\
            \\
            بنابراین، می‌توان فرض کرد که مسئله اندازه‌گیری چالش توصیف نخستین پدیده غیرقطعی است که با آن روبرو شده‌ایم و تنها راز مکانیک کوانتومی است.\cite{ep-qm}
    \section{Summing Up}
    \subsection{Conclusion}
در اینجا، ما با پدیده‌های قطعی در حال توصیف هستیم و یک مدل برای توصیف آن‌ها داریم که پیش‌بینی‌ها و تاریخچه‌هایی را ارائه می‌دهد، اما در اساس ریاضیاتی که در حال استفاده از آن هستیم، تئورمی وجود دارد که بیان می‌کند که همه بیانیه‌هایی که ما در سیستم فرمال ایجاد می‌کنیم، می‌توانند دارای یک اثبات و یا یک مدل به صورت برگشت‌پذیر نباشند.
        \\
        \\
می‌توان فرض کرد که مدل‌های ریاضی که در حال استفاده از آن هستیم، نه تنها توسط ما ابداع شده‌اند، بلکه زبان رسمی خود دنیا هستند. انتزاعاتی که پلاتو توصیف می‌کند. جهان ممکن است یک زیرمجموعه از جهان انتزاعی ریاضی باشد و ما سایه‌های آن انتزاعات را داریم تماشا می‌کنیم. و سپس تئورم‌هایی می‌آیند که خود انتزاع را توصیف می‌کنند، ممکن است شامل نقص‌هایی باشند. نقص‌ها به معنای عدم کامل بودن هستند. اگر کسی قبول کند یا ثابت کند که جهان زیرمجموعه‌ای از جهان ریاضی و انتزاعی است، به طور ذاتی واضح است که چنین نقص‌هایی در جهان باید وجود داشته باشد و به همین دلیل در تئوری‌های رسمی که درباره جهان ارائه می‌دهیم، نیز وجود داشته باشد.
        \\
        \\
در اینجا، سعی می‌کنم به یک‌پارچه کردن دیدگاه‌ها و برقراری ارتباط بین آن‌ها بپردازم و برایتان استدلال کنم که توضیح دادن فرآیند اندازه‌گیری، به صورت اصولی غیرممکن است. نه به دلیل معادلات سخت و یا رفتارهایی که برای ما منطقی نیستند. بلکه به خاطر اینکه وجود بیانیه‌ها در جهان که قابلیت مدل کردن آن‌ها وجود ندارد، یک خصوصیت اصلی از خود طبیعت است.
        \subsection{Where to Go?}
"الآن برخی سوالات وجود دارد که باید پاسخ داده شوند تا در این ایده پیشرفتی حاصل شود."
        \begin{enumerate}
            \item \textit{"آیا جهان یک سیستم فرمال است؟"?} این سؤال نقطه شروع برای تحقیقات ما خواهد بود. ما باید راهی برای نشان دادن اینکه جهان یک سیستم فرمال است یا خیر، پیدا کنیم، زیرا به سادگی در نظر گرفتن واقعیت به عنوان یک انتزاع ریاضی، غیر ممکن است.
            \item \textit{"چگونه ناتمامی، غیرقطعیت و محدودیت‌های محاسباتی با هم ارتباط دارند؟"}باید یک مدل ریاضی قوی نشان دهیم که چگونه این مسائل با یکدیگر ارتباط دارند.
            \item \textit{آیا می‌توان اندازه‌گیری را به عنوان یک زیرمجموعه ناتمامی گودل در نظر گرفت؟} این سؤال مرحله بعدی است. ما باید نشان دهیم که در واقع اندازه‌گیری یک زیرمجموعه از قضیه ناتمامی گودل است.
            \item \textit{کاربردهای فرآیندهای غیرقطعی در دستگاه‌های محاسباتی}
            \item \textit{سیستم‌های فرمال نظریه‌های فیزیکی چیستند؟}
            \item \textit{چگونه ناقص بودن می‌تواند از طریق نظریه‌ها و پدیده‌ها به نمایش درآید؟}
        \end{enumerate}            
                \newpage
                    \bibliography{logicsources}
                    \bibliographystyle{plain}
            



\end{document}