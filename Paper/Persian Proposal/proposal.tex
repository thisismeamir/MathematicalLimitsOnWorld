\documentclass[9pt, twocolumn]{article}


\usepackage{amsmath}
\usepackage{amsfonts}
\usepackage{amssymb}
\usepackage{makeidx}
\usepackage[left=1.5cm,right=1.5cm,top=1cm,bottom=3cm]{geometry}



%Colors
\usepackage[dvipsnames]{xcolor}


\definecolor{black}{RGB}{0, 0, 0}
\definecolor{richblack}{RGB}{7, 14, 13}
\definecolor{charcoal}{RGB}{45, 67, 77}
\definecolor{delectricblue}{RGB}{93, 117, 131}
\definecolor{cultured}{RGB}{245, 245, 245}
\definecolor{lightgray}{RGB}{211, 216, 218}
\definecolor{silversand}{RGB}{190, 194, 198}
\definecolor{spanishgray}{RGB}{148, 150, 157}
\definecolor{darkliver}{RGB}{64, 63, 76}

\colorlet{lightdelectricblue}{delectricblue!30}
\colorlet{lightdarkliver}{darkliver!30}


%ColorDefines
\newcommand{\trueblack}[1]{\textcolor{black}{#1}}
\newcommand{\rich}[1]{\textcolor{richblack}{#1}}
\newcommand{\lightblack}[1]{\textcolor{charcoal}{#1}}
\newcommand{\lightrich}[1]{\textcolor{delectricblue}{#1}}


%Boxes
\usepackage{tcolorbox}
\newtcolorbox{calloutbox}{center,%
    colframe =red!0,%
    colback=cultured,
    title={Callout},
    coltitle=richblack,
    attach title to upper={\ ---\ },
    sharpish corners,
    enlarge by=0.5pt}

\newtcolorbox[use counter=equation]{eq}{center,
	colframe =red!0,
	colback=cultured,
	title={\thetcbcounter},
	coltitle=richblack,
	detach title,
	after upper={\par\hfill\tcbtitle},
	sharpish corners,
    enlarge by=0.5pt }
    
\newtcolorbox{qt}{center,
	colframe=delectricblue,
	colback=white!0,
	title={},
	attach title to upper,
	after upper ={},
	sharp corners,
	enlarge by=0.5pt,
	boxrule=0pt,
	rightrule=2pt}
	
\newtcolorbox{exc}{center,%
    colframe =red!0,%
    colback=darkliver!15,
    title={Excercise},
    coltitle=richblack,
    attach title to upper={\ ---\ },
    sharpish corners,
    enlarge by=0.5pt}
    
\newcounter{theo}
\newtcolorbox[use counter=theo]{theorem}
	{center,%
    colframe =red!0,%
    colback=cultured,
    title={Theorem \thetcbcounter},
    coltitle=richblack,
    attach title to upper={\ ---\ },
    sharpish corners,
    enlarge by=0.5pt}

\newcounter{defcounting}
\newtcolorbox[use counter=defcounting]{define}
{center,%
	colframe=darkliver!50,%
	colback=white!0,
	title={\textcolor{black}{\textbf{\textit{Definition}} \  \thetcbcounter  \ --}},
	coltitle=darkliver!50,
	attach title to upper,
	after upper ={ },
	sharp corners,
	enlarge by=0.5pt,
	boxrule=0pt,
	leftrule=2pt,
    rightrule = 0pt}

\newcounter{lemmacount}
\newtcolorbox[use counter=lemmacount]{lemma}
{center,%
    colframe=charcoal!50,%
    colback=white!0,
    title={\textcolor{black}{\textbf{\textit{Lemma}} \  \thetcbcounter  \ --}},
    coltitle=darkliver!50,
    attach title to upper,
    after upper ={ },
    sharp corners,
    enlarge by=0.5pt,
    boxrule=2pt}
    

\newcounter{examplecounter}
\newtcolorbox[use counter=examplecounter]{example}
	{center,%
    colframe =red!0,%
    colback=cultured,
    title={Example},
    coltitle=richblack,
    attach title to upper={\ ---\ },
    sharpish corners,
    enlarge by=0.5pt}

\usepackage{xepersian}
\settextfont{Nazanin}

\title{حدود ریاضیاتی بر نظریات فیزیک
\\ \large بررسی مسئله اندازه گیری و ناتمامیت گودل}
\author{امیرحسین ابراهیم نژاد}
\date{\today}

\begin{document}
    \maketitle
    \tableofcontents
    \newpage
    \section{مقدمه}
        مکانیک کوانتومی، نظریه‌ای که در ابتدای قرن گذشته با پیش‌بینی‌های بسیار دقیق خود. جایگاه خود را به عنوان دقیق‌ترین نظریه بشر، در رابطه با طبیعت بدست آورد. با این حال پس از گذشت صد سال از آغاز مکانیک کوانتومی، این نظریه همچنان رازآلود باقی‌مانده است. همانطور که در ادامه خواهیم پرداخت به صورت کلی میتوان تمام رازگونگی مکانیک کوانتومی را در مسئله اندازه‌گیری خلاصه کرد؛ هرچقدر که تجسم دوگانگی ذره‌و‌موج و برهمنهی و غیره و غیره برای شهود انسان چالش‌برانگیز باشد، ساختار کلی آنها عِلیت را زیر سوال نمیبرند. بدین ترتیب میتوان اینگونه پنداشت که چالش مکانیک کوانتومی تا قبل از مسئله اندازه‌گیری، تنها برای شهودی نبودن مسئله میباشد.
        \\
        \\
        در این پژوهش، مسئله اندازه‌گیری را از منظر جدیدی بررسی خواهیم کرد. سعی ما بر این خواهد بود تا اندازه‌گیری را نه بخاطر ضعف در مدل سازی و نه بخاطر وجود مشاهده‌گر. بلکه به عنوان یک ویژگی طبیعی که در جهان وجود خواهد داشت بررسی کنیم. ابتدا کمی از فلسفه های گذشته و دلیل موفقیت آنها در توصیف طبیعت به صورت فیزیک کلاسیک  خواهیم گفت. سپس به مکانیک کوانتومی و شگفتی‌های بوجود آمده از طرف آن خواهیم پرداخت. و در نهایت دو قضیه مهم در ریاضیات و منطق قرن گذشته را -قضیه گودل و نظریه محاسبه- بررسی خواهیم کرد؛ در ادامه سعی در توصیف شرایط حاکم در طبیعت با توجه به این قضایا خواهیم کرد. در آخرین فصل نیز به ارائه چند مسئله برای پیشبرد این دیدگاه خواهیم پرداخت.
    \section{چگونه میدانیم؟}
        \subsection{معرفت های ابتدایی}
            بررسی طبیعت از دوران های بسیار کهن، یکی از علایق بشریت به شمار می‌رفت. بشرِ آگاه، دیگر مانند موجودات دیگر درگیر چالش ها و خطر های روزمره قرار نمیگیرد، و در نهایت نمیتواند مانع ذهن خود در طرح پرسش های بنیادین بشود.
            \\
            \\
            روش های متعددی برای بررسی طبیعت در طی این سال ها بوجود آمده و از میان رفته است. حال آنکه از سرایش اشعار و خدایان تا سیستم های فوق پیچیده مهندسی امروز با این حال در جایی میان خدایان یونان و شتابدهنده های ذرات بنیادی بشر به نتیجه‌ای مهم دست یافت. طرح سوالات در قالب آزمایش!
            \\
            \\
            استدلال بسیار ساده ای پشت این نتیجه قرار داشت. چگونه میتوان دانست؟ برای آنکه باور خود را دانش بپنداریم، مهم است که چند صفت را برای آن قائل شویم. ابتدا باور ما باید صحیح باشد. در رابطه با صحت میتوان بحث های زیادی داشت اما بیایید فرض کنیم میتوانیم به هر گزاره که به باور ما تبدیل شده است یک ارزش صحیح یا اشتباه قرار دهیم. حال بدیهیست که باور هایی که ارزش صحیح دارند را به عنوان دانش بپذیریم. 
            \\
            \\
            اما اگر من به طور تصادفی تعداد زیادی حدس در رابطه با موضموعات مختلف بزنم. احتمالا چندتایی از آنها درست از آب در بیایند. پس این سوال مطرح میشود: آیا من در رابطه با حدس های درست خود دانش داشته ام؟ جواب ما منفی است. هنگامی که از دانش حرف میزنیم لازم داریم تا برای داشتن آن باور و البته برای درست دانستن آن باور دلیل بیاوریم و دلیل داشتن آن باور را توجیه کنیم. بدین ترتیب بخش حیاتی از دانش وجود توجیه میباشد.
        \subsection{تجربه گرایی}
            در علم توجیه درستی یک نظریه عموماً به کمک تجربه انجام میشود. بدین صورت که ما یک نظریه فیزیکی را در شرایط گوناگون استفاده میکنیم و یک دسته از نتایج آن نظریه که تنها اگر آن نظریه درست نباشد بوجود خواهد آمد را دریافت میکنیم. در ادامه با آزمایش کردن در آن شرایط خاص یا در جایی که آن نتایج ممکن از دیده بشوند از صحت نظریه داده شده مطمئن میشویم. نظریه های متفاوت فیزیکی در قرن های گذشته به همین طریق بررسی، رد یا پذیرفته شده اند. ساختار آزمایشات فیزیکی پیچیده تر از گذشته شده اما یکی از عناصر مهم فلسفی که پیشفرض تمام این آزمایشات بوده است به شرح زیر است.
            \begin{qt}
                اگر شرایط اولیه در آزمایش را به دقت تنظیم کنیم، برای هر تعداد بار میتوانیم نتیجه مشابه بگیریم.
            \end{qt}
            بسیار خوش‌شانس بوده‌ایم که هنگامی که علوم ما تازه پا به عرصه حضور گذاشته اند، طبیعت با مهربانی این پیشفرض مارا برایمان مهیا کرده بود. چراکه در ادامه این پژوهش خواهیم دید که چگونه این پیشفرض اشتباه میباشد. در فلسفه طبیعی کلاسیک، فیزیک کلاسیک و حتی در جهان بینی ماقبل قرن گذشته قطعیت حرف اول را میزد. اگر برای شما یک سیستم فیزیکی را در حالتی که هست و اندرکنش هایی که دیده توصیف میکردم، شما به راحتی میتوانید تمام اندرکنش هایی که گفته‌ام را وارونه کنید و حالت اولیه سیستم را بدست بیاورید. این را علیت میگوییم، و در ادامه به آن بیشتر خواهیم پرداخت.
        \subsection{قالب ریاضیاتی فیزیک}
            علاوه بر علوم طبیعی ریاضیات نیز از دوران کهن توجه انسان را به خود جلب کرده بود. بررسی موجودات انتزاعی که در سیستم های ریاضیاتی مدل میشوند، حل معادلات و روابط مختلف بین اعداد و موجودات هندسی، بر خلاف ظاهر آشفته طبیعت، آراسته و منسجم‌تر به نر می‌رسد. با این حال اگر به پیشفرض مهم آزمایش که در بخش قبل به آن پرداختیم دقت کنید، دور از انتظار نخواهد بود که سعی کنیم تا تکرارشدگی آزمایش را با کمک ریاضیات توصیف کنیم. بدین صورت که
            \begin{equation}
                \hat H f = Y
            \end{equation}
            یک معادله کوچک که میتواند هرچیزی را توضیح دهد. در این معادله ما گفته ایم موجودی به نام $f$ که خصوصیاتی دارد تحت یک پروسه $\hat H$ به موجود نهایی $Y$ تبدیل میشود. نمیخواهم بگویم که تمام فیزیک به همین یک معادله میرسد، اما بخش بزرگی از فیزیک بررسی همچین معادلاتی است که به عنوان معادلات دیفرانسیل شناخته میشوند. البته که هر کدام معانی خود را در پی دارند و با تغییر هر یک از این سه موجود ممکن از به معادله ای جدید دست پیدا کنید که شاید معنایی حقیقی در رابطه با آزمایشی خاص داشته باشد.
            \\
            \\
            پس میتوان گفت تصویر برای فیزیک کلاسیک این چنین بود:
            \begin{itemize}
                \item آزمایشات در صورت داشتن شرایط اولیه یکسان همواره جواب نهایی یکسان در پی خواهند داشت که به نشانه صحت یا عدم صحت حدس اولیه ما خواهد بود. 
                \item میتوان از ریاضیات برای توصیف وقایع طبیعت استفاده کرد.
            \end{itemize}
    \section{ناتمامیت گودل}
        بیایید صحنه را بنا کنیم. در میانه قرن نوزدهم میلادی، سوالاتی در زمینه بنیان ریاضیات شروع به بررسی شدند. بدلیل پیشرفت های هندسه، حساب دیفرانسیل و نظریه مجموعه ها. ریاضی دانان و منطق دانان سعی در ساخت یک اصول موضوعه برای ریاضیات، و به طور خاص حساب بودند. سیستم های متنوعی در طی زمان ارائه شد که در هرکدام خطا و تناقض نماهایی مشاهده میشدند.
        \\
        \\
        در کنگره بین المللی ریاضیدانان، یک جلسه در پاریس در سال 1900، دیوید هیلبرت یک لیست 23تایی از مسائلی که ریاضیدانان باید اقدام به حل آن بکنند ارائه کرد. یکی از آن مسائل اینگونه نقل میشود:
        \begin{qt}
        ... اما در ابتدا میخواهم در ادامه یکی از مهم ترین سوالاتی که میتوان در رابطه با اصول موضوعه (در حساب) پرسید را مطرح کنم: اثبات اینکه اصول موضوعه ما متناقض نیستند، به این معنا که، تعدادی قدم منطقی که بر اساس این اصول موضوعه برداشته میشود به تناقض منجر نشود.
        \end{qt} 
        به بیان دیگر، هیلبرت ریاضیدانان را برای ساخت سیستمی بدور از تناقضات به چالش کشید.
        \\
        \\
        در ابتدای قرن 20 ام سه مکتب بزرگ در ریاضیات وجود داشت. افلاطونی ها، باور داشتند که موجودات ریاضیاتی مستقل از ذهن انسان وجود دارند، بدین ترتیب کار یک ریاضیدان کشف رمز و راز این موجودات بود.
        \\
        \\
        شهودگرایان، باور داشتند که ریاضیات باید محدود به ساختار های محدود و عملیات های مستحکم بشود، از آنجایی که بیشتر جنبه های ریاضیات مدرن وابسته به متد های نا متناهی میبود آنها ادعا داشتند که باید ریاضیات 3000 سال گذشته کنار گذاشته شود تا جایی که بتوان آنها را با متد های متناهی اثبات کرد.
        \\
        \\
        هیلبرت از ادعای شهودگرایان منزجر بود. بنابراین فرمال‌گرایی را بنانهاد. فرمالیست ها ادعا دارند که ریاضیات چیزی بجز سرو کله زدن با علائم بی معنا تحت قوانین خاص نبود. از دید آنها نشان دادن نبودن تناقض بین اجزای ریاضیات چیزی بجز این نبود که نشان دهیم تحت اصول موضوعه خود نمیتوانیم ترکیبی از این علائم بیمعنا را رقم بزنیم.
        \\
        \\
        هیلبرت برنامه ای برای اثبات اشتباه شهودگرایان ارائه کرد. او میخواست تا اثبات کند، که با روش های متنهای که مورد تایید شهودگرایان بودند میتواند نشان بدهد که تمام ریاضیات کلاسیک، بدور از تناقض است. اما در نهایت همه شوکه زده شدند.
        \\
        \\
        در کنفرانس شناخت شناسی در علوم دقیق در کونیگزبرگ، آلمان، یک ریاضی دان 24 ساله به نام کرت گودل ادعا کرد که میتواند گزاره ای  درست را نشان دهد که در فرمالیزم ریاضیات کلاسیک قابل اثبات نیسد.
        \\
        \\
        اثبات این ادعا که به قضیه ناتمامیت اول گودل شهرت دارد در کنار قضیه ناتمامیت دوم، که ادعا دارد وجود تناقض در یک فرمالیزم را نمیتوان در داخل خود فرمالیزم اثبات کرد. ارائه شدند. ریاضیات که قرن ها نقش دقت و اعتبار را بازی میکرد ناگهان با چنین قضایایی روبه رو میشود که فرمالیزم کلی آن را تحت تاثیر قرار میدهد.
        \\
        \\
        در ادامه بیایید اثبات کوچکی از قضیه ناتمامیت اول را بررسی کنیم.
        \subsection{اثبات}
            برای داشتن هر نوع ریاضایات یا فرمالیزم ابتدا به داشتن یک زبان نیازمندیم. بنابراین بیایید تعاریف ریاضیاتی خود را مستحکم کنیم.


\end{document}