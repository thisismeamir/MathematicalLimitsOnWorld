بررسی طبیعت از دوران های بسیار کهن، یکی از علایق بشریت به شمار میرود. بشرِ آگاه که دیگر به اندازه موجودات دیگر درگیر زندگی روزانه نیست و خلاقیت بینظیری را از خود در مواجهه با چالش ها بروز میدهد نمیتواند مانع پرسش هایی بشود که طرح آنها تنها از ذهن خلاق او بر می آید. بدین ترتیب همواره به طرح پرسش هایی از قبیل، چرا اینجا هستیم؟ آیا این جهان حقیقت است یا توهم؟ اجزای جهان چطور کار میکنند؟ و… دست میزنیم. 
\\
\\
در اصل علم همیشه یک سؤال ساده وجود دارد که باید به آن پاسخ داده شود. این سؤال نیازمند پرسیده شدن و بررسی دقیق است، پیش از هر گونه پیشرفت علمی. این سؤال این است: "چطور از دانشی که در اختیار داریم اطمینان داریم؟" حدود علم برای نگه‌داشتن چنین سؤالی خیلی کوچک هستند، زیرا در واقع خودشان نتیجه آن هستند.
\\
\\
معرفت شناسی شاخه ای از فلسفه است که خود را مشتاق پاسخگویی به چنین سوالاتی میداند. کلمه Epistemology از لغاط یونانی epistéme و logos تشکیل شده که به همراه هم به معنای مطالعه دانش است. برای شروع این فصل ابتدا لازم است سه سوال زیر را طرح کنیم
\begin{enumerate}
     \item دانش چیست؟ و وقتی میگوییم میدانیم به چه معنا است؟
     \item منبع دانش چیست؟ چگونه میتوان اطلاعات قابل تکیه پیدا کرد و آنها را دانش نامید؟
     \item آیا بدست آورد کمال دانش ممکن است؟ اگر نه حدود ما برای دانستن چیست؟
\end{enumerate}
سوال اول به نظر مسئله تعریف است، اما پرسیدن "دانش چیست؟" نقش مهمی برعهده دارد. اهمیت این سؤال از آنجاست که با تعریف نادرست دانش، ممکن است نادرست را با حقیقت ترکیب کنیم، و این آخرین اشتباهیست که در هنگام طبقه بندی گزاره ها بخواهیم مرتکب شویم. علاوه براین، اگر دانش را به صورت بی دقتی تعریف کنید، مشکلاتی برای ارائه استراتژی های خوب و منابع، و حتی پیدا کردن محدودیت های واقعی دانش پیش می آید.
\\
\\
سوال دوم مربوط به روش‌هایی است که با آنها اطلاعات (پیش زمینه های درست یا نادرست) درباره هر چیزی به دست می‌آوریم. چه چیزی باعث می شود یک روش قابل اعتماد باشد و دیگری نباشد. این سوال شامل مسئله قدیمی از این قبیل است که چرا باید به علم اعتماد کنیم؟ با بررسی این مسئله، من سعی خواهم کرد نشان دهم که علم، و به ویژه فرآیند آزمایش، به عنوان روشی پایدارتر جهت تولید دانش شناخته شده است.
\\
\\
آخرین سوال هدف پروژه‌ای است که در پیش رو دارید. این سوال دعوت به مطالعه دقیق منبع دانش است و برای بیشتر وضوح، نشان می‌دهد که آیا هیچ محدودیتی دارد یا آیا یک تونل بی‌پایان از دانش مداوم است. ممکن است ما بخواهیم در مورد این موضوع بحث کنیم و یا به عبارت دقیق‌تر، فلسفی شویم. اما در نسخه‌های بعدی (یا فصل‌های بسته به جایی که این را می‌خوانید)، ملاحظات مهم این سوال در مورد منطق جهان، محاسبات، و دیدگاه ریاضی درباره طبیعت و تجربه مورد بررسی قرار می‌گیرد.
%%%%%%%%%%%%%%%%%%%%%%%%%%%%%%%%%%%%%%%

\begin{qt}
     حدود ساعت سه صبح بود که نتایج نهایی محاسباتم پیش رویم بود. عمیقاً جا خورده بودم و چنان تشویش داشتم که خوابم نمیبرد. خانه را ترک کردم و به آرامی در تاریکی راه رفتم. روی صخره ای مشرف به دریا در کنج جزیره رفتم. منتظر ماندم که خورشید برآید. \textit{ورنر هایزنبرگ}
\end{qt}
مکانیک کوانتومی در قرن گذشته در نزدیکی زمانی که کلوین، فیزیکدان بزرگ، فیزیک را کامل میدانست پا به عرصه گذاشت. شگفتی های طبیعت بعد از یافتن قوانین کلاسیک حرکت، جاذبه، اپتیک و الکترودینامیک. چهره جدیدی به خود گرفته بودند. نکته حائز اهمیت دررابطه با فیزیک کوانتومی مسئله اندازه گیری در آن است که موضوع اصلی بررسی این پژوهش بوده.
Sure, I can translate the paragraphs into Persian and use \subsection{} for each number. Here's the translation:

\subsection{توابع موج}
وضعیت یک سامانه کوانتومی با تابع موج $\Psi(x,t)$ توصیف می‌شود که شامل تمام اطلاعاتی است که می‌توان درباره سامانه بدست آورد. تابع موج یک تابع پیچیده واقعی است که به موقعیت $x$ و زمان $t$ بستگی دارد. طبق معادله شرودینگر، تکامل زمانی تابع موج به صورت زیر است:
\begin{equation}
i\hbar\frac{\partial}{\partial t}\Psi(x,t) = \hat{H}\Psi(x,t)
\end{equation}
که در آن، $\hat{H}$ عملگر هامیلتونی را نشان می‌دهد که مجموع انرژی کل سامانه را نمایش می‌دهد.

\subsection{مشاهدات}
مشاهدات، مانند موقعیت و جرم، به وسیله عملگرهایی که بر تابع موج اثر می‌گذارند، ریاضیاتی نمایش داده می‌شوند. عملگر موقعیت $\hat{x}$ و عملگر جرم $\hat{p}$ به صورت زیر تعریف می‌شوند:
\begin{equation}
\hat{x} = x,\qquad \hat{p} = -i\hbar\frac{\partial}{\partial x}
\end{equation}
این عملگرها با هم قابل تبادل نیستند، بدین معنا که نحوه اعمال آن‌ها به تابع موج حائز اهمیت است.

\subsection{اصل عدم قطعیت}
طبیعت احتمالی مکانیک کوانتومی به این معناست که نتایج اندازه‌گیری با اطمینان پیش‌بینی نمی‌شوند، بلکه تنها با یک احتمال خاص قابل پیش‌بینی هستند. اصل عدم قطعیت این را می‌گوید که حداکثر دقتی که می‌توان در هنگام شناوری بین جفتی از مشاهدات را همزمان داشت، محدود است. به عنوان مثال، موقعی

\subsection{جمع بندی}
ساختار کلی مکانیک کوانتومی که در قرن گذشته معرفی شد به صورت زیر است.
